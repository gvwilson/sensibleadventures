\documentclass[12pt]{report}

\oddsidemargin 0.0in 
\textwidth 6.5in

\begin{document}

\newcommand{\UL}{\ship{Unshadowed Land}}
\newcommand{\aemott}{\emph{{\ae}mott}}
\newcommand{\citer}[1]{\emph{#1}}
\newcommand{\ship}[1]{\emph{#1}}
\newcommand{\word}[1]{\emph{#1}}

\begin{titlepage}
  \begin{center}
  \begin{huge}The Voyage of the {\UL}\end{huge}
  \\
  \begin{Large}Greg Wilson\end{Large}

  \vspace*{4cm}

  Copyright {\copyright} Greg Wilson, 2014.  This material may not
  be reproduced in any form without the author's prior permission.

  \end{center}
\end{titlepage}

\newpage

\chapter{The Return}

On the fourth of Carnelian, YS 1140, a fisherman northwest of
Jalkelainan saw a ship emerge from the early morning fog.  History did
not record his name, but his blood-sworn affidavit survives in the
archive of the Standing Committee's inquiry:

\begin{quotation}
Her timbers were weathered gray, and her sails torn and patched to
motley.  I saw no sign of life aboard her, nor dolphins at her
bow.  I thought her the ghost come back from the time of the
Rebellion, and so made haste to harbor to sound the alarm.
\end{quotation}

Despite her battered appearance, the ship was no ghost.  She was
instead the {\UL}: the last survivor of an expedition that had set out
six years previously to raid Bell Prison on the Salt Coast, and the
first vessel to circumnavigate Cherne without the aid of magic since
the end of the Age of Heroes.  As she made her weary way toward
Jalkelainan harbor, she carried eight of her original Ruudian crew, an
ex-slave from Thind, a Gifted chimpanzee, and the Ara\~{n}ese
missionary Costanjila di Xueres, whose prison letters are the other
principal record of that epic voyage.

Jalkelainan's harbor master at the time was a practical, hardheaded
man named Kupoleva's Aardi.  Nothing is known of his personal life or
political beliefs, but a portrait does survive, showing a man with
hair cropped down to stubble, a neatly-trimmed beard, and absolutely
no sense of humor.  Born in YS 1099, he would have grown up with
stories of the rebellion that finally ended five centuries of unalive
rule in Ruuda.  It is therefore not surprising that upon hearing the
fisherman's description of a ``ghost ship'', he immediately took action.

Two lean cutters, each crewed by twenty marines and a magician, were
immediately sent to intercept the newcomer, with strict orders not to
board.  A third, larger, vessel was pressed into service and
dispatched in their wake.  This ship carried a harbor pilot, a
doctress, another magician, and another squad of marines, along with a
catapult and as many barrels of pitch as the harbor master could
commandeer in half an hour\footnote{In a subsequent suit against the
port authority, a warehouse owner claimed that two barrels of dark rum
were commandeered as well.  Unlike the pitch, they were apparently
never returned.}.

The next three hours must have been tense ones for K.'s Aardi.
Jalkelainan's mayor would have undoubtedly descended upon him en
masse, demanding information that he didn't have; rumors that the Pale
Remainder were returning to harvest their former subjects would have
spread through the town like wildfire.  One imagines the harbor master
standing on the seawall, telescope clenched in his hand, secretly
wishing that the Standing Committee's rules did not prevent him from
recruiting a few whales to act as messengers or watchgen, as was the
norm in other parts of Cherne\footnote{Many Gifted animals had no
qualms about working for the Pale Remainder: since their flesh and
blood couldn't be used to keep the Remainder ``alive'', they were never
required to pay the flesh-tithe.  As a result, most Ruudians viewed
the Gifted as spies and collaborators.  The hardline factions that
came to power in the Fifth Rebellion's turbulent final days banished
most Gifted from Ruuda, and severed relations with both the Parliament
of Whales and the Trollthang.  Many informal contacts persisted, but
it is unlikely that an official in the harbor master's position would
have dared to openly violate the Standing Committee's directive.}.

The third of the ships sent by K.'s Aardi reached the {\UL} in
mid-morning.  With the fog gone, its crew could clearly see what the
fisherman had not: the vessel was very solid, and her crew unafraid of
direct sunlight.  The two cutters paralleled her course as the third
ship came alongside her.  The harbor pilot's entry in Jalkelainan's
port log is almost comically understated:

\begin{quotation}
Boarded twin-spire {\UL} (acting capt. Costanjila di
Xueres for Cn. Tomonainan's Petta, deceased, out of Ruuda-in-Ruuda
by way of diverse other harbors) third hour after dawn, 4
Carn. 1140.  No evidence plague or unalive.  Harbor fee waived.
\end{quotation}

The {\UL}'s exhausted crew gladly relinquished their vessel to the
harbor master and his marines.  There was a brief discussion of
whether it would be better to tow her into harbor, or let her come in
under her own sail.  According to local folklore, the harbor master
decided on the former, but was then persuaded by the doctress that the
latter would be a less ignominious end to what everyone involved must
have realized was an epic achievement.  Over the course of six years,
the {\UL} had sailed almost a hundred thousand gallops: past the
dragon's isles, Thind, and Ini Bantang to raid Bell Prison, then
around Cap di Per\c{c}alle and the Regimental Kingdoms to return to
Ruuda.  For the first time in almost a thousand years, the inhabitants
of northern Cherne had made direct contact by mundane means with
peoples they knew of only as legends.

What is more, her cargo was enough to pay for the entire expedition.
The three other ships she sailed with may have been lost, along with
most of her original crew, but in her hold the {\UL} carried medicinal
pearls from the Undja Delta, gem corals from Ini Bantang, a
twentyweight of diamonds from Bell Prison, and many other treasures
and curiosities.  More importantly, her crew had brought back more
noon-grade dayglass than existed in the whole of Ruuda at the time.
Scarred and salt-starved though they might be, her crew were
potentially among their nation's wealthiest citizens.

The little convoy reached Jalkelainan shortly after noon.  A handful
of townspeople had fled in terror, but most of the rest flocked to the
seawall to witness the {\UL}'s arrival.  We can only hope that some
cheered as she lay alongside the pier and made fast, for if her crew
had known what lay before them, they might well have cast off and set
sail once again.

\chapter{Our One True Ally}

In many ways, the {\UL}'s arrival in Jalkelainen on that end-of-summer
day in 1150 was a direct result of two previous arrivals of even
greater note.  The first took place more than six centuries earlier,
and plunged Ruuda into a long night that shaped the whole history of
northern Cherne.  The second occurred within the memory of many people
still living in Ruuda when the last survivors of the Salt Coast
expedition returned.

\begin{center}
* * *
\end{center}

Yearagain Eve, YS 478.  With the moon a dark sliver in the night sky,
Ruuda's major cities---Jalkelainan, Pohjoinen, Etela, and
Ruuda-in-Ruuda---throbbed to the beat of thousands of drums.  People
wrapped in winter furs thronged the streets, laughing, drinking, and
setting fire to scraps of paper on which they had written prayers for
the coming year.  Most would have included the customary plea for the
Uncertain Angels to return and care for the world, as the north was
solidly Subservient at the time.  Many others would have mentioned the
hero Uws, ruler of the eponymous empire that lay on the other side of
the Helada Mountains.  After a decade fighting bandits and monsters on
his southern border, he had turned his attention northward, and was
probing Ruuda's defenses.  Nearly fourteen palms tall, he was still
protected by the Angelic sword that never left his grasp, but that
wouldn't have stopped the fishergens, tradesfolk, and minor
aristocracy of Ruuda's city-states from hoping.

Two watchmen were dicing in Jalkelainen's lighthouse that evening.
According to legend, they began rolling nothing but nines.  Each
accused the other of cheating.  Fuelled by drink, they quickly came to
blows, and it was only when one chased the other onto the upper
platform that they noticed a double handful of ships gliding into
harbor.  Black, with black sails, they slipped silently through the
ice-choked water, showing no lights, and sounding no trumpets to
signal their arrival.

The watchmen's first thought was that Uws was taking advantage of the
celebrations to invade\footnote{Exactly how they thought an Uwsian
fleet could have traveled nearly a thousand gallops down the length of
the Ruudian coast without being detected is not recorded.  One
suspects drink may have played a role in their reasoning.}.  The
watchmen began beating the warning drums, but their signal was lost
amid the hubbub of celebration below.  They could only watch as eight
of the nine black ships made fast at the city's piers.  The nine stood
off at the mouth of the harbor, her sails hanging slack despite the
harsh easterly wind.

That much is legend.  What happened next was recorded in several
diaries, and in letters carried over the Heladas by survivors pleading
for assistance or refuge from Uws, the Darpani, or anyone else.  Some
four hundred gens disembarked with an orderly haste born of long
practice.  They moved quickly to secure the harbor, taking control of
the gate that separated the docks from the rest of the town and
putting archers on the roofs of several warehouses.  The handful of
revellers in the dock district were bludgeoned into silence, or
quickly and efficiently killed.

Alarm began to spread as the invaders pushed through the crowd toward
Jalkelainen's central square, where the mayor-elect and other
dignitaries were, as was customary, waiting on tables at a midnight
feast for the area's poor and indigent.  People began shouting or
screaming as they realized the town was under attack.  A few fought
back, including a fishing boat captain named Asmenaila's Urvi:

\begin{quotation}
I saw two men lay hands on one of the creatures, which drove its sword
into the first's belly and then broke the second's arm without
breaking stride.  I drew out my scaling knife and stabbed its neck as
it passed.  It flinched, but then drew the blade out and threw it back
at my feet and walked on.
\end{quotation}

A.'s Urvi had just discoverd what the rest of Ruuda would learn in the
days and years that followed.  The invaders, who called themselves the
Pale Remainder, were not truly alive, and so could not be slain by
mundane weapons.  Instead, their bodies were a patchwork of pieces cut
from the living flesh of human beings and stitched together with
strands of moonlight.  Nothing short of complete dismemberment could
end their unnatural existence.

Nothing, that is, except the direct light of the sun.  For reasons
that are still unknown\footnote{The Pale discouraged investigation of
their magic during their reign, often violently.  The prohibition is
now largely customary; nations routinely accuse one another of
violating it when tensions between them escalate.}, the magic the Pale
Remainder used to hold their bodies together could not withstand
exposure to direct sunlight.  Reflected or magical light was not
enough; nor was light released from anything less that the finest
noon-grade dayglass, which at the time was as rare in the north as
giants' hair.

Estimates vary, but most scholars agree that the Pale Remainder
numbered fewer than three thousand in total.  Their magicians were
powerful, but there were humans in Ruuda at the time who could match
them spell for spell.  What could not be matched was their inhuman
strength, and the fact that they never wearied.  A Pale swordsgen
could fight without rest for three days straight.  The only limits
were the need to sleep, and to replace any body parts that started to
go off.

The popular image of the invaders has been shaped by who the Pale were
at the height of their power, or later romanticizations, such as the
aristocratically-posed figures in polished black armor in Vardishav
the Elder's famous \citer{Night Falls on Ruuda}.  The reality was
probably very different: the Pale were in fact a raggle-taggle band of
refugees.  The four hundred or so creatures who descended upon
Jalkelainen while their compatriots attacked Ruuda's other major
cities would have worn sea-stained rags and scraps of boiled leather
harness salvaged during the probing raids they had made on islands
lying off Ruuda's coast in the preceding two weeks.  Many were
probably missing a few non-essential body parts, such as ears and
noses; a few might even have been short a limb or two, though if they
employed the same tactics as they used later in their reign, they
would have addressed that in the field by harvesting from those they
killed.

But where were they from?  And why had they come to Ruuda?  Dozens of
theories have proposed, some more fantastic than others.  The most
credible is that the Pale Remainder were a leftover from the era of
the Uncertain Angels.  Several sources from the Age of Heroes refer to
``bandits'' living in the caverns of southern Praczedt after the fall of
the Uncertain Angels.  The most complete, the Szestetelmeny Chronicle,
describes them as ``{\ldots}feasting upon the unwary or unwise as might
a great cat upon an ox.''  Like so many others afterward, the
chronicle's anonymous authors may have mistaken the Pale's use of
others' bodies to renew their own for cannibalism.

If this hypothesis is correct, then it is believable that the
eruptions of Mount Narjemczy in YS 461 and 474 could have triggered a
series of catastrophic underground floods, which would have driven the
Pale up to the surface.  There, they would have been faced with a
difficult choice.  They could go south, into the jungles of Thind and
Ini Bantang; west, onto the Great Plain; north, to Uws; or norther
still, to Ruuda.  Thind or Uws would mean challenging the largest and
most populous realms in Cherne on their home ground.  Wresting control
of the Great Plain from the Darpani tribes and clans would have been
less of a challenge, but establishing a reliable supply of the body
parts the Pale needed to sustain themselves would have been a
significant challenge.

A second theory is that the Pale were not driven to the surface
against their will, but had instead been planning such a move for
centuries.  The Pale leader Bokchang Tzen Urpatnati made several
references to his compatriots' near-infinite patience during the
Pale's first post-conquest embassy to Uws in 577.  Effectively
immortal, the Pale could well have decided to wait out the chaos that
accompanied the fall of the Uncertain Angels, and the churn of
magic-fueled empires that characterized the Age of Heroes.

A third possibility, highly relevant to present-day events, is that
the Pale were ``invited'' to leave their caverns by the dragon Sulk.
The two islands now known as Sullair Major and Sullair Minor lie
directly opposite the area in southern Praczedt from which the Pale
most likely came.  They, the waters around them, were off-limits to
human beings for two hundred years while Sulk brooded on her three
eggs.  Shortly after they hatched in YS 240-41, she launched a series
of devastating attacks against nearby towns and magicians, ruthlessly
eliminating any possible human threat to her progeny.  While she would
obviously not have been able to pursue the Pale into their underground
refuge, her mundane and magical strength would have posed a
significant risk to the Pale.  If in fact they were a holdover from
Angelic times, it would have been entirely in keeping with immortal
custom for negotiations over their relocation to take several hundred
years\footnote{There is some evidence that the Szestetelmeny Chronicle
was originally conceived as an \emph{aide memoire} to help human
beings keep track of the state of play during decades-long negotations
with Sulk.}.

Patient or not, the Pale Remainder could be as fast as lightning when
necessary.  It probably took them less than half an hour to secure
Jalkelainen's dock.  The two watchmen in the lighthouse (if they
actually existed) would have been killed along with the harbor guards
and any passers-by.  If taken by surprise, their deaths would have
been delivered with a single knife thrust up into the skull through
the soft tissue under the jaw, or by strangulation, two methods
favored by the Pale because they did so little damage, and spilled so
little precious blood.

The next step was Jalkelainen's main square.  Like most large
settlements in Cherne at the time, Jalkelainen had originally been the
center of an Angelic estate.  While their architecture was as varied
as their physical form, the Uncertain Angels's palaces tended to be
three stories of stone, coral, or magically-hardened wood around a
courtyard large enough for a thousand or more to gather without
feeling crowded.  In Jalkelainan's case, this basic plan was augmented
with four diagonal ``spokes'', each a hundred strides long, that stepped
down in sections to two stories, then one, before turning into
open-sided galleries.  Entrance to the square itself was through
two-story arched gateways lying between the spokes.  The northern one
lay less than a gallop from the docks; the other three opened onto
major roads leading east to Pohjoinen, south toward the Heladas, and
west to peter out among small fishing villages and forest.

Leaving fifty or so of their number to guard the harbor, the remainder
of the Pale force moved quickly toward the square.  As they approached
it, they split into three forces.  The first, numbering perhaps two
hundred, pushed through the crowd toward the eastern arch, while
another fifty or so headed for the northern and western.  The southern
arch was deliberately left open: as in the ``show hunts'' later staged
by the emperors of Thind, the Pale's aim was to drive their victims
like cattle.

The mayor of Jalkelainen, Vuonemaima's Saardu, was serving salt fish
soup to orphaned beggars when word arrived that the harbor had been
attacked.  Ladle in hand, and undoubtedly a little bit drunk, he
dismissed the first report, saying that the town watch ``{\ldots}well
knew how to deal with brawlers.''  But then shouts and the din of metal
against metal were heard near the northern gate.  A wainwright with an
arrow in his arm began bellowing, ``Demons! Wraiths and demons!''
People began screaming, racing to and fro, tripping over the long
leather bootlaces that were in fashion that year.

V.'s Saardu was nearly sixty at the time.  He had inherited control of
his family's fur-trading business while in his early twenties, and
parlayed a warehouse and three ships into two full piers and full or
part ownership of eighteen sturdy vessels.  He may or may not have
been the same ``V.'s S'' who helped defend a trading post upriver from
Jalkelainen from an Uwsian raid while still a young man, but he
certainly saw action in the summer of 453, when tensions between
Pohjoinen and Jalkelainen's merchant houses turned bloody.

Laying about himself with his ladel, he pushed through the crowd to
the high table where many of the town's other dignitaries had been
waiting upon a motley collection of beggars, whores, and dancing
masters.  He climbed up on the table and began shouting orders in a
vain attempt to make order out of the growing chaos.  Suddenly, an
arrow plunged out of the night sky and into his right foot.  His
attendants tried to pull it out, but were unable to, leaving V.'s
Saardu effectively nailed to the heavy pine table.  ``Then our line
needs form here,'' he is reputed to have said, ``For if I cannot move,
then neither shall anyone else.''

Unfortunately for the people of Jalkelainen, the Pale Remainder had
other plans.  With the northern and western entrances to the square
sealed off, the larger force to the east began pushing into the
square.  The Ruudians fought back with whatever makeshift weapons came
to hand, but their carving knives, skewers, broken bottles, and
cobblestones had little impact on their unalive foes.  If a lucky
stroke did manage to cut a hamstring or put out an eye, the damaged
Pale would simply bludgeon someone unconscious and drag gens behind
the advancing line.  It took only a few minutes, sometimes less, to
remove the required part from the hapless victim and bind it into
place of whatever had been injured.

And then there was blood.  The Pale Remainder bled from their wounds
like anyone else, and like anyone else, if they lost too much, they
first grew weak, then lapsed unconscious, then died.  But unlike the
truly living, their bodies could not replenish what they lost.
Instead, they had to take living blood and inject it directly into
their own veins.

Today, over a thousand years later, we are inured to these unnatural
acts.  We can only imagine the horror that must have come over the
people of Jalkelainen as they saw their neighbors and loved ones
drained and dismembered just a few strides away from them.
Khodormeneneko's Ijtvan's \citer{Remembrances}, written a generation
later, tells of a magician who saw first her father, then her husband,
and finally her son harvested, until she stood face to face with a
Pale warrior whose face was an amalgam of everything she had loved in
the world.  She traded her memory of that love for a spell powerful
enough to blast her foe to ash, only to be slain by the one behind
him.  Another story describes a young man pushing a pile of masonry
over the edge of a roof onto a Pale below, crushing him, then being
caught and drained to revive the enemy he thought he had defeated.

Faced with what seemed an unstoppable force, the crowd panicked.  They
poured through the southern arch of the square, trampling their own on
cobblestones made slick by frost, spilled rum, and blood.  The Pale
pressed against them relentlessly.  With the harbor under their
control, there was nowhere for Jalkelainen's people to run.  The roads
were choked with midwinter snow---it would be four months before a
horse or camel could get as far as the mountains, and another month
after that before anything earthbound could cross them.

Ignoring his commands, and the blows they received from his ladle,
V.'s Saardu's bodyguards picked up the table he stood on and carried
it toward the gate.  One was felled by an arrow; a townsman took his
place.  Two others were cut down by a squad of Pale who had pushed
ahead of their comrades; again, people stepped in and lent their
shoulders.  Like a magician standing on a river, or an actor leaving a
stage, V.'s Saardu was borne south onto Jalkelainen's main street.

There, if legend is to be believed, he was met by the two watchmen who
had first seen the Pale arrive.  Armored, with swords in hand, they
had been mistaken for Pale by several in the crowd until they
discarded their helmets.  ``My lord, what shall we?'' the first cried.

``Draw out this arrow from my foot,'' the mayor commanded.  The two
watchmen pushed the townspeople aside and did as they were commanded.

``Now give me your blade,'' V.'s Saardu said.  When one of them handed
him his sword, the mayor cut his palm, crying, ``This I swear by my
blood, that I will not from here 'til these are vanquished.''

A sudden gust of wind swept through the crowd---the blood oath had
taken.  That same gust blew a Pale arrow from its path so that it
fetched home in V.'s Saardu's neck.  He fell to the ground, instantly
dead.  A few moments later, when the Pale drove the last of the crowd
away from the table on which the mayor had been standing, they found
his ghost standing there, arms crossed, defying them to pass.

It would be six hundred years before that ghost was finally laid to
rest.

\begin{center}
* * *
\end{center}

The same events played out elsewhere in Ruuda that Yearagain Eve with
only minor variations.  Jalkelainan was both the smallest of the major
northern city-states, and the furthest west, so the Pale Remainder
only sent nine ships against it.  Pohjoinen was attacked by fifteen,
Ruuda-in-Ruuda by either twenty-eight or thirty\footnote{Even the Pale
Remainder's own histories do not agree on this point.  Several
scholars have attempted to square this circle by speculating that
there were originally 30, but two were later erased from the records
during one of the Pale's internal feuds.  This is, however, purely
speculation.}, and at Etela---only a fraction larger than Jalkelainan,
but close to the border with Uws---forty-one ships brought almost two
thousand unalive invaders ashore.

Everywhere they landed, the Pale followed the same strategy: secure
the harbor and major potential rallying points, then drive a
substantial portion of the local population into the midwinter snow.
It was brutally effective; while the true death toll will never be
known, at least a quarter of Ruuda's population were dead by Peridot
of 478.  The remainder were scattered, disorganized, hungry, and
leaderless.

Word of the invasion reached Uws within days.  As usual, the hero was
sunk in melancholy in his winter palace in Vnir.  When a courtier
approached him to say that a Gifted eagle had arrived with urgent
news, Uws reputedly asked him, ``It it new, or just news?''  Thinking
that the eagle's report was exaggerated, and that in fact Etela had
been attacked by a well-organized band of pirates, the courtier
apologized for intruding and left his king to brood.

It wasn't until Chrysoprase, nearly two months after the invasion,
that Uws roused himself.  Donning his seven-gallop boots, he strode
across the Sibor Plain toward Etela, shaking the earth with every
step.

A trio of Pale magicians met him near a small stone fort just north of
the border\footnote{The fort was destroyed and rebuilt several times
over the next few centuries.  The author was able to visit it during
the writing of this book; its only modern occupants are a pair of
faded ghosts and some badgers.}.  Shrouded for protection against the
sun, they could easily have been mistaken for crows, or for the
shadows of things not present.  They presented Uws with a gold ring, a
ram, and an unstrung fiddle---the same three gifts that the mayor of
Etela had sent south as a token of peace every year for the past two
decades.

Uws thanked them for their gifts, and asked after his ``friends'' in the
north.  ``They are well, or not,'' one of the Pale replied.

``And if I were concerned to know which?'' Uws inquired.

``Then we would counsel you not to concern yourself.''  With that, the
Pale magicians bowed and vanished, leaving the hero with a scrap of
brass, a musty tag-end of wool, and a broken stick in his hands.

Uws returned to Vnir that evening.  Coronel Szarkos ard Niczolu
recorded in his journal that, ``[Uws] shows such joy as chokes the
court with terror.''  Having served under him for several generations,
the nobility in Vnir knew that only thing could make their ruler so
happy: the prospect of battle.  Blocked in the south by the dragon,
and to the west by two mountain ranges and the Herd of Trees, Uws
would probably have found a way to break his oath of peace with the
Ruudians eventually.  By invading, the Pale Remainder had save him the
trouble.

Despite being a hero, Uws was no rash fool.  He spent the spring of
478 gathering, equipping, and training his army.  At the same time,
refugees from Etela and elsewhere were collected at the border; those
who knew something useful were taken to Vnir for interrogation, while
the rest were sold into slavery or driven south toward Praczedt.

As poets later wrote, each new revelation lengthened the odds, and
brightened the smile on Uws's face.  Ruuda had been invaded by roughly
five thousand well-organized unalives, who had taken control of its
major urban centers, and were quickly strengthening their grip on the
surrounding countryside.  Resistance was fierce, but uncoordinated:
one by one, smaller towns and villages were taken, and their counts
and colonels either harvested or driven off.  Many starved, unable to
find food in the end-of-winter cold.  Others banded together and
counter-attacked, only to be cut down.  A few led their families and
retainers into the Heladas, or fled west around Cape Grind and the
Herd of Trees to seek refuge in Derway and Bruyere.

The sudden appearance of so formidable a foe on his doorstep snapped
Uws out of his decades-long stupor.  The portraits of his wife and
children that covered the palace walls were not taken down, but for
the first time since he slew them in a murderous rage, Uws paid more
attention to the present than to the past.  New generals were
appointed; those who could not keep up with their lord's pace were
quickly replaced.  An embassy was sent to Darp to buy horses, taking
with it a substantial portion of the royal treasury.  Another made its
way down the coast to Timorcze (then the principal city of northern
Praczedt) to tell its duke that Uws's preparations were not directed
at him.  We have no record of how the duke reacted to this
unlooked-for reassurance, though the presence of several regiments of
Praczny archers in the Uwsian army during the subsequent
war\footnote{The fact that these troops were under Uwsian command was
later cited by Sarkoszy chroniclers as proof that Praczedt had at the
time been a province, or at least a protectorate, of Uws.  It is much
more likely, however, that Uws simply hired them, as he and his
mercenary band had often been hired in the days before he stumbled
across the cache of Angelic treasures that started him on the road to
kingship.  The persistence into modern times of several Praczny family
names in northeastern Uws may signal that not all of those soldiers
returned home when the fighting was over.} may signal that for once,
Praczedt's rulers were able to put aside their interminable squabbles
in the face of an external threat.

The first blow in the struggle to reclaim Ruuda was not struck by Uws,
however.  That honor fell to a sea captain from Pohjoinen named
Loyhkata's Uurvo, known to history as Uurvo the Foul for her love of
``ripened'' squid\footnote{Ripened squid is prepared by marinating and
smoking finger-thick slices of tentacle, then burying them in sealed
jars for a year or more until the surface of the meat begins to
deliquesce.  The liquid is decanted, and the jellied remainder spread
on toasted flatbread.  Its consumption is banned in Seyferte and
Leyselle, though it is frequently used there as a pesticide.  Many
other regions forbid its sale to pregnant women.}.  Uurvo had beached
and buried her double-masted \word{laiva} at a fishing encampnent some
forty gallops northeast of Pohjoinen at the start of winter in order
to conduct repairs.  Under normal circumstances, she and her crew
would have carved new planks and beams for their ship during the
winter months, then refloated her in the spring.

When the first handful of refugees arrived with word that Pohjoinen
had been taken by monsters, the villagers told them to move along.
They only had stores laid in for so many people; more mouths would
have meant hunger or starvation.  As the trickle grew to a torrent,
though, Uurvo took charge.  The \word{laiva} was dug out of its nest
in the sand and refloated in the ice-clogged harbor.  As her crew
worked round the clock to refit her, small boats were sent east and
west along the coast to gather food and intelligence.  The local
whales were enlisted as well: in exchange for a ten-year increase of a
tenth share of the catch, they drove school after school of winter cod
into the villagers' nets.  Some refugees undoubtedly did starve, or
succumb to malnutrition or despair, but many survived.

By the middle of Heliodor, four months after the invasion, Uurvo was
ready.  She had a rough idea of how the Pale Remainder's magic worked;
she would also have known how vulnerable the Pale were to sunlight.
She therefore settled on the same tactic that the Pale's enemies would
use for the next six centuries: sneak in as close as possible without
being detected, find a small group of Pale, and attack just after dawn
on a cloudless day.

At first, their attack seemed to be a small but unqualified success.
With two Gifted seagulls as lookouts, Uurvo threaded her \word{laiva}
through the jumble of forested islands off Ruuda's northern coast to a
point some twenty gallops from Pohjoinen.  There, she split her crew
into three troops: one to stay aboard, a second to draw the Pale in,
and a third, the largest, to fall upon them.

They did not have long to wait.  That very evening, one of their
seagulls reported a Pale patrol moving toward them along the coast
road.  Songs and poems still record the disgust the Ruudians felt when
she learned that a handful of alives were riding with the three
Pale\footnote{Despite the protests of those enamored of folk tunes,
there were certainly not the ``hundred-strong troop'' of the traditional
song \citer{Ballad of the Bright Buccaneers}.}.  Some were convicts who
had been given a reprieve, but others were military gens who had no
trouble accepting the change of power.  In Ruuda, as elsewhere, the
millenium-long civil war among the Uncertain Angels had instilled a
convenient degree of moral flexibility in their human chattel; the
rigid nationalism so characteristic of modern Ruudians had not yet
arisen.

The Pale found shelter in a farmhouse just before dawn, setting their
human servants on guard.  As the sun cleared the tops of the nearby
pine trees, the first group of Uurvo's raiders fell upon them, slaying
two or three and then retreating into the forest.  That was the signal
for the second, larger group to attack.  The farmhouse and its
outbuildings were set on fire; when the Pale emerged, they were cut
down and stripped naked.  Stolen skin and muscle fell off their bones
in the sunlight like dry mud off a boot.

The Pale's alive servants were treated less kindly.  Setting a pattern
for centuries to come, Uurvo's crew hacked them to pieces, making sure
to put the major muscles and organs beyond use.  No unequivocal record
survives, but it is likely that in at least some cases, this was begun
while the victim was still alive.  The struggle against the Pale
Remainder was already beginning to harden the Ruudian soul.

The Ruudians returned to their ship in high spirits.  Their enemy
seemed much less formidable than they had feared: true, they had lost
five of their own number, but that seemed a small price for three of
the invaders.

But then the sun set, and the moon rose, and Uurvo's crew discovered
that Ruuda would not be won back that easily.  An unnamed member of
Uurvo's crew told the villagers what happened next\footnote{Taken from
the \citer{Deed of Corlum Early}, ca. 930 (?).  In the \citer{Deed},
King Corlum meets a hermit in the Herd of Trees who is cursed to tell
his tale to everyone he meets until Ruuda ``bathes in sunlight''.  While
such an encounter could have taken place (particularly in the depths
of the Herd, when the king was searching for the key to his true
love's heart), it seems more likely that the chronicle is paraphrasing
a report passed down over several centuries by a survivor from the
village where Uurvo's \word{laiva} had wintered.}:

\begin{quotation}
Ae the moon its blue light fell upon them, all thay had lain metal
upon the foulers were siezed upon b\^a the unruly ghosts of them as
thar slain and driven to tae up thay its metal again each the other.
All thay cunning bloody to venge the foulers sauf uns cut thay down
uns own, and nar stand bot a few.
\end{quotation}

Whatever magic held the Pale Remainder together acted after their
death on whoever slew them.  Once exposed to moonlight, anyone who
believed ge had struck a death blow against one of the Pale was driven
to kill those around gar.  Unlike the spell known as Orran's Last
Laugh, though, those afflicted were not driven into a murderous
berserker rage.  Instead, the spells' victims remained completely
capable of planning and dissimulation.

Two thirds or more of the men and women aboard Uurvo's \word{laiva}
had been at the farmhouse.  They quickly overwhelmed the rest of the
crew, then turned the ship toward the village where it had been
wintering.  Uurvo herself was probably among the slain: in keeping
with a tradition going back to Angelic times, she would have stayed
with the ship while her gens were on shore.  During the centuries that
followed, leaders of rebellions against the Pale made several attempts
to ask her spirit for advice or a blessing, but none succeeded.  We
will never know what she thought when her crew turned on her; we can
only hope she felt it was worth it.

Dawn came, and with it, shock and bewilderment.  The Pale Remainder's
vengeance spell held no sway in daylight: all the crew knew was that
many of them were wounded, and that twenty or more of their number
were missing.  The two Gifted gulls who had been guiding them were
gone too, frightened off (though the survivors could not know this) by
the carnage they had seen on board.

Sensibly, but erroneously, the Ruudians assumed that the Pale had
somehow found them.  They laid on as much sail as they dared in the
windswept northern spring and raced back to their village.  There,
they find that half a dozen more ships had arrived in their absence.
They disembarked and gave their version of events, warning the
assembled alives that someone or something was still pursuing them.

An hour later, with the sun down and the moon up, they were dead, and
the village was in flames.  The thousand or so Ruudians left alive by
the tragedy boarded whatever was still seaworthy and fled west, taking
with them the bitter knowledge that the fight to reclaim Ruuda would
be longer and harder than anyone had imagined.

\begin{center}
* * *
\end{center}

Word of Uurvo's attack would not reach Uws until the following year.
Even if he had heard the story earlier, the events of that summer
might well have played out as they did.  Confident and re-energized,
the hero was pressing ahead with his plans almost recklesslly.  Uws's
strategy was driven by food and mud: he could not move his troops away
from their dwindling stores until the spring potato crop was harvested
in early Topaz, and there was little point trying while the spring
rains were turning the roads north into shallow muddy rivers.

As the moon lightened toward its mid-summer gold, his troops drilled
endlessly under the unforgiving eyes of Uws's trusted lieutenants.
Like the whirlwind he had wrestled with in his youth, Uws himself
travelled from one side of his domain to the other, sometimes covering
two hundred gallops in a day in order to oversee details of drill and
provisioning.  Many of the nobility who had risen to positions of
power in the long years since his family's death were retired, and
younger gens promoted into their places.  Boots, arrowheads,
canvas---everything was counted and checked.

On the nineteenth of Topaz, YS 478, Uws left Vnir at the head of a
force of some 15,000 gens.  Mounted on a pure white camel, surrounded
by a ten-strong bodyguard of giants, and equipped with the boots, ax,
and mask that were the source of his power, he must have looked like a
force of nature.

He gathered the rest of his troops on his way north, along with two
dozen heavy ballistae he had ordered be constructed.  By the time he
reached the Kravriye River, his army had swollen to 25,000.  As was
usual for the time, he divided them into a mounted vanguard, three
columns (the central of which he led himself), and a rearguard, which
also contained the artillery and engineers.  Advance troops had
already scouted and prepared campsites, so the whole force was able to
cover 20 gallops or more each day.

Uws did not expect to surprise his opponents.  The only way to move
that many men north was on the Great Northern Road, a legacy of
Angelic times that was only just beginning to show signs of wear after
nearly five centuries without maintenance.  Forty strides wide, with
waist-high conical stone markers every hundred and twenty strides, it
paid only condescending attention to the shape of the land.

Forty gallops past the Kravriye, near a hamlet called Leikikalu, the
Great Northern Road passed through a broad 'V' that had been cut out
of a hill.  The Pale Remainder attacked his vanguard there just after
dark on the second of Carnelian.  Casualties were light: the Pale
managed to conceal themselves from the Uwsian magicians, and were able
to pin their opponents against a dense stand of forest, but did not
have time to press home their advantage before the approaching dawn
forced them to withdraw.

Fearful of a larger trap, Uws ordered the vanguard to wait for the
main body of the army to catch up before advancing.  It was late
afternoon when the first column reached them, and nearly dark when the
central column---Uws's own---arrived.  With night fast approaching,
Uws ordered the army to make camp, and posted sentries.  He then
invited several officers from the vanguard to his tent to discuss the
previous night's battle.

Coronel Szarkos's granddaughter Martta transcribed her father's
account of what happened next\footnote{Here as elsewhere, I use the
Ebrentennen translations of the Szarkosy's dynastic chronicle, rather
than those officially incorporated into the Barsadov dynasty's
records.  Despite regular protestations to the contrary, the evidence
of political bias in the latter is overwhelming.}:

\begin{quotation}
The Old Bear (note: Uws) called for cider and sweets, and plied the
gens with questions about the damned's (note: Pale Remainder's)
tactics and valor.  When the stewards pulled aside the tent flap to
bring in what had been ordered, moonlight fell upon three of the
cavalrymen, who on the instant began to complain that the air was
close.  The Old Bear ordered the all outside, at when the whole party
who had been in the vanguard put hands to whatever weapons they could
and threw themselves at him and us with no heed for their own lives.
\end{quotation}

It was a repeat of what had happened to Uurvo's crew.  Of the three
hundred gens in the vanguard, perhaps a quarter had struck one of the
Pale to the bone and survived.  Throughout the camp, those seventy-odd
gens calmly or madly went about the chore of killing as many of their
fellow soldiers as they could.  Cries of ``Treason!''  filled the air;
men who were untouched by the curse attacked one another, each fearing
the other's drawn sword.  It was Orran's Last Laugh writ large.

As the camp descended into chaos, the Pale attacked, firing the tents
on three sides of the perimeter to drive the Uwsians east toward boggy
ground.  Once again, they attacked in lines; any Pale who fell was
dragged to safety by those in the rear line to be patched up with
blood and muscle harvested from Uwsian casualties.

The army's other two columns had camped a gallop or so ahead and
behind Uws's.  Reinforcements from those two camps began arriving
within minutes of the start of the battle.  In response, the Pale
Remainder's lines bent in on themselves to form squares, which fought
their way clear of the Uwsian troops before breaking into the tireless
jog-trot that allowed the Pale infantry to cover ground like seasoned
cavalry.  They carried as many of their truly-dead with them as they
could for later resurrection.

By itself, the attack would not have been enough to check Uws's
advance.  While the effect on morale of him losing a battle---even a
small one---for the first time in over a century cannot be
underestimated, his physical losses were actually relatively small.

The real damage only sunk in slowly.  A squad of Pale had penetrated
all the way to the heart of Uws's camp.  Their bodies lay on the
ground near his tent, each head neatly severed by Uws's magic ax.  He
had slain Pale: what would happen to him the next time moonlight fell
on him?

``We'll know when we know,'' was his famous answer.  As dawn approached,
he ordered his gens to resume their advance.  He himself moved to the
first column to march beside his troops in a plain brown leather
cuirass.  Laughing and joking, he must have seemed his twenty-year-old
self once again.  We can only guess whether his jokes became forced as
the shadows began lengthening around him.

When the army made camp, Uws commanded Coronel Szarkos to shackle him
wrist-and-ankle.  His giant bodyguards formed a ring around him, each
holding a club padded with layers of canvas and blankets.  Darkness
fell; the whole camp held its breath.

Uws suddenly began laughing.  Telling his followers that his mask or
ax must have protected him from the curse, he ordered them to unchain
him\footnote{According to legend, one of the magicians in attendance
protested, turning herself into a puff of feathers to be carried away
by the wind when the assembled nobility refused to listen.  This is
said to be the origin of the expression ``blow to feathers'', meaning
``flee to anywhere''.}.  He then sauntered back to his tent, picked up
his ax, and fell upon his men.

Over the next twelve hours, Uws killed or wounded almost two thousand
of his own soldiers.  Most fled; those who tried to fight back were
restrained or beaten back by their comrades, many of whom were slain
by Uws in reward.  Once again, troops from the other camps came
running at the sound of battle.  This time, however, there was nothing
they could do: Uws's Angelic weapons made him both lethal and
untouchable.

The next five days were a living nightmare for the king and his gens.
Each day was a forced march for the border; each night, the Pale
Remainder fell on them like hawks on a tide of lemmings.  There was
nothing Uws could do but weep each time the sun crept toward the
horizon, and his giant bodyguards re-attached his manacles.  So proud
of being ``just another soldier'', he was now a danger to his own
beloved army.

As Uwsian losses mounted, Coronel Hradcy ard Eszten volunteered to
lead a counter-attack early on the second day of the retreat (6
Carnelian 478).  The Gifted birds who were serving as scouts had
reported that the Pale were taking refuge from the sunlight in heavy
canvas tents.  Even if they couldn't be slain, Hradcy reasoned,
destroying their shelter might force them to abandon their pursuit.
He was probably also mindful of how desperate his gens were for a
victory---any victory---and of how vulnerable the Pale's human aides
would be without their unalive masters to protect them.

The Pale's camp formed a broad arc along the edge of a pine forest.
It would have been a suicidal position for a living army, but it made
perfect sense for the unalives, as it gave them shadows to retreat
into if attacked during the day.  Their alive servants were positioned
more conventionally in a single large camp laid out on the classic
``square and tee'' pattern\footnote{The Uncertain Angels used the
``square and tee'' pattern for military camps from one end of Cherne to
another.  Interestingly, it is only in Barra Bantang and Ini Bantang,
where Angelic rule was most tenuous, that it was adopted as a layout
for permanent settlements.} at the northern end of the arc.  Coronel
Hradcy therefore concentrated his attack at the southern end, ordering
his cavalry to fire as many tents as they could as his infantry and
magicians used swords and spells on anything that moved.

Casualties in the Battle of the Shadowy Forest were probably actually
rather light on both sides: Hradcy's gens were afraid to penetrate too
deeply into the forest, while the Pale Remainder were unable to
venture out of it.  Its most important effect came that afternoon,
when the Uwsians reconnected with their main force.  After presenting
his report, Coronel Hradcy drew his sword and offered it to the king.
When asked why, he said, ``Because if you do not slay me while 'tis
day, my lord, I stand at risk of being traitor when 'tis night.''
Despite their attempts to burn the Pale Remainder in their tents, and
avoid them otherwise, Hradcy and several of his gens had actually
slain two directly\footnote{In Bardessalen Yeramowcsza's ``speculative
biography'' of Hradcy, the coronel consciously decides to incur the
Pale curse as a way of forcing Uws to realize what needs to be done.
As noble as this sounds, readers must keep in mind that Bardessalen's
works are called ``speculative'' for good reasons{\ldots}}.

Understandably, Uws refused, but the coronel was unrelenting.  As far
as he was concerned, he had fallen in battle; the fact that he was
still walking and breathing was an irrelevance.  If Uws would not kill
him and his men, they would have to kill one another, and Uws would
still be left with the task of finishing off the last
one\footnote{This may be the first evidence of the emergence of
post-Angelic strictures against self-killing.}.  And there wasn't time
for debate: the eastern horizon was already bruised.

Heavy-hearted, Uws embraced the gens one by one, swearing by his blood
that they would be remembered as heroes, and that their families would
be taken care of.  They then knelt in a line on the cold, wet ground.
Taking up his ax, Uws swung it seven times, then let it fall.
Wordlessly, his giant bodyguards carried the bodies away to be
immolated on heroes' pyres as their king lay down in his tent to be
chained hand and foot once more.

The army that crossed the Kravriye on 15 Carnelian was a tattered
shadow of its former self.  Of the 25,000 gens who had followed Uws
north at the end of Topaz, at least five thousand were dead---two
thousand of them at their king's hand.  A roughly equal number had
slipped away during the retreat, less afraid of trolls, cave lions, or
the Pale Remainder than they were of their moon-maddened king.

Word of the disaster reached Vnir long before the army.  Szarkos ard
Martta later wrote that her grandfather had assumed effective control
on the journey, and that when Uws paid attention to his followers'
questions and pleas at all, he simply told them to ask the coronel
what to do.  Allowing no one near him but his much-loved giants, Uws
locked himself in his chambers, alternately berating those who dared
disturb him, and pleading with them tearfully to forgive him.  It was
a repeat of the events of a half-century previously, but this time
Uws's magicians were unable to find a way to break the curse.

Uws disappeared on Yearagain Eve, YS 480/481, along with his mask, ax,
and boots.  There are those who claim he is still alive somewhere in
the Herd of Trees, foraging by day, chaining himself by night with
locks that even he cannot pick or break before the sun rises.  As far
as history is concerned, though, he no longer mattered.  Like the
Angels whose magic had given him his superhuman strength, the Old Bear
had finally fallen, and taken with him the only chance the living had
of quickly defeating the Pale Remainder.

\begin{center}
* * *
\end{center}

The next six centuries were no harder for Ruuda than they were for
many other parts of Cherne, but it is not polite to say that within
earshot of Ruudians.  While kingdoms and empires rose and fell
elsewhere, the Pale Remainder's grip on the land north of the Helada
Mountains was as cold as strong as steel and as cold as ice.
Accustomed as they were to heroes with Angelic powers, the gens of the
time simply had no conception that ``mere mundanes'' could defeat a
magical foe.

Many scholars have overlooked this last point, failing to
recognize\footnote{See for example Ld.\ Armenda Denys Cal\c{c}aere's
\citer{On Immediate and Extraneous Causes in History}.  Her argument
that the poetry and song of the 500s and 600s shows a modern
conception of capability is disputable, since the first recorded
versions of those poems and songs date from the early 1000s, and we
may reasonably believe that they have altered over time.  We may also
discount testimony to the contrary from the handful magicians and
cursed gens who have been alive since that time, as most are mad,
forgetful, dishonest, or Praczny.} how differently gens viewed the
world at the end of the Age of Heroes.  For countless thousands of
years, humanity and the Gifted lived every day in the shadow of the
Uncertain Angels, whose intellect and power no mortal being could
possibly match.  Those who siezed control after the Angels'
fall---Janbinder the Great, Uws, the Brass Admiral, and others history
would rather forget---may have been born mundane, but used leftovers
from Angelic times to transform themselves into something greater.
Simply put, the Ruudians of the 400s had been trained for hundreds of
generations to believe that it was their destiny to be governed by
creatures other, and more powerful, than themselves.  The fatalism
with which they initially accepted Pale rule should therefore be
neither surprising, nor criticized.

The form that unalive rule would take was clearly heralded in the
second year of the Pale Remainder's reign.  In the spring of 479, the
Pale ordered isolated landholders to abandon their farms and move to
larger towns and villages.  A policy of communal responsibility was
strictly enforced: each gen owed duty to a ``gathering'' defined by the
Pale, rather than to gar family or lord-commander.  Those who dragged
their heels saw their homes and possessions burned; those who resisted
were harvested.

At midsummer, the Pale harvested again: anyone too infirm to work in
the fields, or accused of malingering by gar gathering-mates, was
forced to ``volunteer'' a filch of blood (enough to fatally weaken a
young child).  There was scattered resistance, but the shock of the
previous year's conquest had not yet worn off, and the refugees who
had fled around Cape Grind to Derway, or south to Uws and Praczedt,
were as yet unable to offer any assistance.

The Pale harvested again in the autumn, after the crops were in.  As
at midsummer, bands of half a dozen to twenty rode from village to
village at night, taking shelter in houses whose windows had been
securely sealed during the day.  Each stop saw the same sequence of
events: a careful examination of each gathering's roll
book\footnote{These were not yet bound with leather made from human
skin; that customs seems only to have arisen later.}, followed by
interrogation of a few selected individuals, and then, as the moon
rose, the harvest itself.  Blood was mixed with tincture of olymanden
and stored in sealed glass jars; skin, muscles, and organs were
grafted to the Pale who needed them right then and there.

Hungry, dispirited, and forbidden to travel for any except the most
pressing of reasons, an entire generation of Ruudians sunk into a
stupor that even the poorest Thindi \word{doi} would have scorned.
Comparing pre-invasion tax rolls to the surviving Pale roll books
shows that population decreased by a third in most areas, and more
than half in some.  The countryside around Jalkelainan was
particularly hard-hit: its shorter growing season and colder winters
meant that the loss of too many able bodies could doom an entire
community to starvation.

Despite the harsh conditions, gens did find ways to communicate.
Gifted animals and birds carried messages from one town to the next,
as did tinkers and other itinerant tradegens; the alives who served
the Pale as outriders could sometimes be bribed or shamed into passing
word along; and magicians sometimes managed to send dreams to one
another.

The most famous surreptitious channel, though, was that used by
Suirenami's Iervo, a feller whose pine orchard stood beside the
Kalastava thirty gallops above Jalkelainen.  He devised a way to
remove strips of bark from trees, paste a waxed envelope containing a
message to the wood, and refit the bark so that only the closest of
inspections would reveal it had been tampered with.  The first
recipient was his lover, a furniture maker in Jalkelainen itself, but
the technique quickly spread, carrying with it the hope that the
Ruudians could free themselves.

The First Rebellion (YS 507) was a poorly organized fiasco.  Inspired
by stories of what life had been before the invasion, apprentices in
Ruuda-in-Ruuda and Pohjoinen turned their adzes and saws on whatever
Pale ``pets'' came to hand, rather than on the Pale themselves.  The
support they had been promised by the Regency Council in
Uws\footnote{Coronel Szarkos ard Niczolu formed the Council early in
YS 481 to govern the kingdom ``until its monarch shall have regained
hismelf''.  By the time of the First Rebellion, 26 years later, it was
little more than the Szarkosy family's court; the other major
coronelcies had established \emph{de facto} independence that would
last until the reign of Alyczandr II Szarkos (YS 701-717).}  never
materialized; the ``fleet'' promised by the refugees who had settled in
Derway got as far as Cape Grind before being beaten back by storms.

The First Rebellion marks the earliest recorded appearance of the
{\aemott} who would play such a large part in the subsequent history
of Ruuda.  The word is a contraction of \emph{{\ae}n am otta}, or ``one
for another''.  Inspired by the story of the Uwsian Coronel Hradcy,
each {\aemott} swore to turn gar weapons on geself after killing as
many of the Pale as they could.  With swordplay, archery, and other
martial arts forbidden under pain of the offender's entire gathering
being harvested, the {\aemott} demanded complete secrecy: they wore
masks in meetings, used handsign instead of speaking to conceal their
voices, and concentrated on mastering weapons that were easily
concealed.  After the collapse of the First Rebellion, they abandoned
any hope of a mass rising.  Instead, each {\aemott} was to sieze
whatever opportunities came gar way, in the hope of eventually
whittling the Pale Remainder down to a defeatable size.

As distant as it seemed, that hope was the only one the Ruudians had.
By the time the First Rebellion ended, the Pale Remainder's strategic
weakness had become clear: they were unable to create more of
themselves.  Each time a Pale's bones were burned, the ranks of those
who husbanded Ruuda's people like the Darpani husbanded cattle were
reduced by one.  In the words attributed to S.'s Iervo\footnote{Like
most modern scholars, this author believes the ``Suirenami Missives'' to
be a forgery from the 1000s or even later, rather than a transcription
of an earlier original.}:

\begin{quotation}
It is therefore a simple race: shall we reduce their numbers
sufficiently to make possible victory before we forget what it is to
live without their yokes on our backs and their knives at our throats?
\end{quotation}

Five hundred years is a long time, even for creatures who cannot die.
Through conquest, alliance, and appeals to past greatness, the
Szarkosy family extended its control of the Sibor Plain outward from
the royal palace in Vnir until all of Uws's former territory was
reunified under their rule in YS 713.  Further south, Praczedt
suffered an interminable plague of monsters, as one abomination after
another crawled out of Hrstil Canyon to lay waste to whatever had not
been burned, eaten, or cursed by its predecessors.  And on Cherne's
west coast, from Cape Grind south to the Cansado Mountains, the
Regimental Kingdoms slowly coalesced in the ``New Territories'' settled
in the Angels' last days by Ealx and Heot's followers.

Thousands of refugees from the Pale invasion settled in the
northernmost of these kingdoms: Bruyere, Ensworth, and especially
Derway.  Regimental and Ruudian families often lived side by side, but
there seems to have been little intermarriage; as the saying goes, the
two communities ``sang different songs''.  While the Regimentals were
hewing new homes and farms out of the vast forests that still
blanketed the western seaboard, the Ruudians look north and east to
the land that had been stolen from them.  Periodically, a fashion for
decorative scarring or self-mutilation ``for sake of remembrance'' swept
through the exiles.  More frequently, young gens snuck into Ruuda in
small boats, made the hazardous trek through the mountains, or
disguised themselves as tinkers, traders, or trappers to see for
themselves what life was like under the Pale's rule.

Like other regions of Cherne in this time, Ruuda only had significant
contact with its immediate neighbors.  The dragon Sulk destroyed ships
that came anywhere near her nest on Sullair Minor, which effectively
closed the Gulf of Szigor\^u and the Inner Yr to navigation.  In the
southwest, the Pesa Sadilla seems to have been much larger than it is
today: accounts of the time speak of people as far north as the
Cansado Mountains, and well out to sea, losing their reason.  And
overland, the Hett who infested Avaunt were too obsessed with their
dark magics and unnatural machineries to allow the Flying Mountain to
be used as ``Cherne's biggest camel''.  As a result, the lands north and
south of the Great Plains were effectively cut off from one another,
and spent most of the Age of the Same turning each other into legend.

To the continued disappointment of Ruudians both at home and in the
diaspora, the rest of Cherne therefore regarded Ruuda's plight as a
local matter.  The coronels of northern Uws began exchanging embassies
with the Pale Remainder as early as YS 500; by YS 540, Pale emissaries
were being courted by all sides in Thind's interminable dynastic
struggles.

The Second and Third Rebellions (YS 786 and 855) had little impact on
these realities.  Both were initiated by hotheads in the Ruudian
diaspora; neither had significant support among ``native'' Ruudians
(although those few who joined in are still remembered in local songs
and legends for their courage and folly); and neither had any lasting
impact on day-to-day life.

What did have an impact in those years was the {\aemott}.  Once a
year, more or less, a quiet fanatic---often someone who had waited
years for the right opportunity---managed to bring down one of the
Pale Remainder.  The resulting reprisals were always horrific, but by
the late 800s, the realities of the occupation were clear to all.  In
the long run, the Pale Remainder could only lose.

If it is impolite to say that the Age of the Same was no harder on
Ruuda than it was on many other parts of Cherne, it is simply
dangerous to point out that the arts flourished under Pale rule---so
much so that scholars in Thind, Ara\~na, and elsewhere sometimes speak
of a ``golden age''.  The Major Triad---calligraphy, conversation, and
bas-relief sculpture---were refined by Pale masters like Chezen
Ortopalti, Yuwen Cmo Alptni, and the ``Moonlight School'' of Pohjoinen.
Ortopalti in particular broke new ground: his widely-imitated
\citer{Hands Reaching Out of the Plain} was the first sculpture to
incorporate incised calligraphy, while his ``bronze dreams'' (a series
of small pieces done between YS 750 and 850) suggest calligraphic
characters that seem to somehow hover in the instant that precedes
recognition.

The Pale Remainder did not generally pursue the Minor Triad---law,
architecture, and mathematics---but when they did innovate, others
took notice.  The Pale's decision in YS 504 to reinstate the biennial
examinations used by the Uncertain Angels to select their human
servants was copied within a decade by the Empire of Thind (where they
had earlier been abolished by Janbinder the Great).  Similarly, when
the Szarkosy dynasty's Uniform Regulations of YS 738-40 granted
citizenship to Gifted animals who had given at least eight years of
military service, they were directly imitating the Pale Remainder's
proclamation of YS 620.

These examples, and others, inspired the Learned Jizelle uy-Armaq's
argument that other nations' imitation of the Pale Remainder reflects
the era's yearning for a return to the certainties of Angelic rule.
To quote a representative passage\footnote{See for example her
\citer{Lectures Given at Ensworth in Honor of the Royal Accession}.}:

\begin{quotation}
The chaos of the Age of Heroes produced a yearning among all the
peoples of Cherne for the comfortable certainties of Angelic rule.
Those who felt this yearning were unaware of its strength, as our
mouths are unaware of the taste of water, but in the Pale
Remainder---almost immortal, almost invulnerable, and able almost
effortlessly to release their subjects from the unaccustomed burden of
self-direction---the small child that hides within each of us found
the image of an ``adult'' in whom to put its trust.
\end{quotation}

Were we to adopt the Ld.\ Jizelle's standards of reasoning, though, we
could equally well argue that her school's preoccupation with the
effects of Pale rule on political developments in other regions of
Cherne is an ``unaware'' attempt to avoid the fact that we know almost
nothing about Pale politics itself.  The chronicles of the time refer
to Bokchang Tzen Urpatnati as the ``king'' of the Pale Remainder, but
their exact organization remains a mystery.  While Ruudians referred
to the Pale conclaves in the major cities as ``Midnight Courts'', we
still do not know whether these were social gatherings, a forum for
debate, or attempts to find spells able to create more of their kind.
We \emph{do} know, from the careful records kept by the {\aemott},
that some of the Pale who went into these ``courts'' never came out, but
the same can be said of inns, bath houses, and universities.

Then came the catastrophe of YS 966.  Over the course of ten weeks
(17-18 Peridot to 26 Citrine, with minor aftershocks reported as late
as Amethyst), four great caverns beneath the lands of the western
coast collapsed, drowning Plangent and half of Fourette.  The shocks
shook the whole of Cherne, rekindling the fires of Mount Narjemczy and
causing sympathetic cave-ins as far away as Barra Bantang.  Half a
million gens drowned as sea levels dropped a double handspan,
disturbing the migratory patterns of fish, birds, clouds, and other
creatures for decades.  The devastation in the World Below may have
been even worse; we will never know.

In Ruuda, the land beneath the upper reaches of the Saarumeva River
gave way on Malachite 19 and 20.  The steel and brass mines beneath
its headwaters were cut off from the outside world, and in them, the
{\aemott} finally made their move.  There, safe from the sun's light,
the Pale Remainder should have been undefeatable, but the miners and
slaves who rose against were able to triumph against the odds.  They
had practiced and prepared for such a day for over a hundred years;
life and death were as one to them; and they had a new weapon:
dayglass so pure that its stored light could dissolve the spells that
held the Pale Remainder together.  There were no more than a dozen
pieces of the precious material in the whole of Ruuda at the time, but
that was enough: at a cost of hundreds to one, the {\aemott} were able
to clear the mines and proclaim an independent Ruuda once again.

It took the Pale Remainder three years to crush the Fourth Rebellion.
By the time they were done, a fifth of Ruuda's population had been
slain, or had died in the forced famines inflicted on suspect regions.
The Saarumeva mines were never completely cleared; the harder the Pale
and their human followers pushed, the deeper the rebels retreated.  On
at least two occasions, Pale war parties surfaced to find that they
had passed completely beneath the Heladas and into the Herd of Trees,
some hundred and fifty gallops from their starting point.

The Ruudians in the Regimental Kingdoms were elated.  Now making up
almost half the population of Derway, they pressured its marshal-king
into assembling a fleet to attack Jalkelainen in the spring of YS 967.
Troops landed five gallops from the harbor, and pressed forward to lay
siege to the city, but were beaten back and forced to retreat.

While the Pale Remainder had tolerated nuisance sorties from the
Regimental Kingdoms, an invasion of this magnitude demanded a
response.  It was not long in coming: with the Rebellion reduced to
embers by the winter of 970-71, the Pale mounted a two-pronged attack
of their own.  In Sapphire 971, a fleet of some fifty ships made its
way north around Cape Grind, sailing far out to sea so as to avoid any
magical, mundae, or Gifted patrols the Derwers might have mounted.  As
they swung back in toward the shore, thirty of those ships made for
Connomenaer, Derway's ``second city'', which they burned to the ground.
The rest swept through the islands that dotted Derway's northern
coast, destroying every small fishing village they could find.

The marshal-king of Derway, Lyam the Occasional, hastily negotiated a
peace.  It was no time to be weakened by a foreign war: his southern
neighbors had quickly gobbled up what was left of Plangent, and were
sniffing at the broken remains of Fourette.  The Ruudians in the
Parledoux who had pushed for the invasion were exiled or beheaded, and
the situation apparently returned to the status quo ante.

But beneath the surface, everything had changed.  The Pale Remainder
could be beaten by mundane means; Ruudians did not have to wait for a
``new age of heroes'' to win their freedom.  The question was no longer
``if'', but ``when'', ``how'', and ``by whom''.

For many gens, the last of these was crucial, as three different
factions were vied to take charge of the next uprising.  The first was
the diaspora in the Regimental Kingdoms, which was still, despite
Marshal-King Lyam's purge, numerous and influential.  While still
ethnically Ruudian, they had adopted much of the lifestyle of their
host nations over the centuries; the language they spoke was at best a
cousin to that spoken in central and eastern Ruuda, and many had
adopted Regimental naming, marriage, and funeral practices as well.

On the other edge of the rainbow were the {\aemott}.  After centuries
of struggle in which every victory meant the death of the victor at
the hands of gar most trusted companions, ``their'' war had become
deeply spiritual.  {\aemott} families were effectively a second
culture within Ruuda; in camps called \word{paetakyla}\footnote{A
contraction of a phrase meaning ``a safe place to flee to''.}, hidden in
the Helada Mountains or on the fringes of the Herd of Trees, the
{\aemott} gathered every dawn to welcome the sun's return, thanking it
for being their one true ally in the struggle against their unalive
oppressors.  Unsophisticated, often unlettered, the {\aemott} felt
contempt for the ``softness'' of the diasporan Ruudians, who in turn
made jokes about the crudeness of their cousins (in part, no doubt, to
conceal the fear that the {\aemott}'s fanaticism inspired).

The third and final player on the Ruudian side of this complex game
was the people of Cherne's northeastern corner, between Ruuda proper
and Uws.  From the early 700s onward, a growing number of poets and
noblegens had proclaimed that the region centered around Etela, which
they called Vaarda, was not part of Ruuda at all.  Quoting evidence
both scholarly and spurious, they claimed that the Uncertain Angels
who had resided in Etela had been enemies of those in Ruuda, and that
the only reasons the world thought them Ruudian were that Uws hadn't
conquered them, but the Pale Remainder had.  According to a
``Declaration of Amity'' that was circulating in Uws as early as YS 850,
the goal of those seeking to overthrow the Pale should not be a united
Ruuda, but rather a confederation of regional states on the Regimental
model, each with one of the major cities (Jalkelainen, Ruuda-in-Ruuda,
Pohjoinen, and Etela) as its capitol.

Opposition to this plan was one of the few things diasporan and
{\aemott} Ruudians agreed on.  The idea of Ruuda free and whole was
the only thing that united the former; in an era in which
long-distance travel was as rare as giants' hair, they simply didn't
understand that centuries of living next door to Uws had changed the
``Vaardians'' almost as much as living in the Regimental Kingdoms had
changed them.  As for the {\aemott}, the idea that anyone but them
could have any right to shape the future of a Pale-free Ruuda had
become unthinkable.

\begin{center}
* * *
\end{center}

Pale rule in Ruuda was also changed by the rebellion.  The first sign
of this was Bokchang Tzen Urpatnati's institution of a spring
inspection tour.  Popularly referred to as a ``king'' or ``general'',
Urpatnati's actual role among the Pale Remainder seems to have been
more ambiguous.  The records that survive, and can be read by mundane
eyes, use terms like ``eldest'' and ``first'', although it is not clear
whether these are meant literally, or simply honorifics.

Urpatnati's first tour in YS 972 was both a show of strength and a
punishment.  Accompanied by more than four hundred of his fellow Pale,
he spent three months sailing from Muteletta (a small port near the
Uwsian border that marked the southern extent of the Pale's domain) to
the fishing villages northwest of Jalkelainen.  At each stop, the
Pale's living servants rounded up hundreds of their fellow Ruudians,
sometimes riding inland over two hundred gallops, and bringing victims
back by the cartload.  There were no inspections, no attempt to
separate the guilty from the innocent, just wholesale harvesting on a
scale that dwarfed even the aftermath of the original invasion.

Heliodor 8th, 1091: the fleet arrived in Jalkelainen harbor an hour
after sunset.  After two decades of ``graveyard peace'', the Pale were
confident that only the embers of rebellion remained.  The fleet,
which had steadily shrunk from its 972 size, numbered a mere fourteen
vessels: five of the dark ships in which the Pale had originally
arrived, and nine newer vessels ranging in size from double-masted
\word{laiva} to sturdy three-masted \word{rahda} capable of carrying
150 men, 50 horses, and a portable forge.  Between them, the ships
carried no more than 80 of the estimated 2800 Pale still left in
Ruuda.

As was traditional, Jalkelainen's Pale lady, Boknan Tzur Pelludidar,
was waiting on the sea wall.  As the fleet came into the harbor, a
blinding ray of light stabbed downward from the lighthouse---the same
lighthouse from which, according to legend, two watchmen had first
seen the Pale fleet centuries before.  Pelludidar didn't even have
time to scream: when the light struck her, the spells that held her
stolen flesh together dissolved, and she simply fell apart.

The dozen or so Pale standing near her suffered the same fate.  The
rest fled, shouting for their servitors to bring them weapons as the
light swept across the incoming ships and sent Pale after Pale to gar
true death.  The {\aemott} had finally found the weapon they had been
searching for, and the Fifth Rebellion had begun.

Even today, many commentators insist that the \word{skenren lans} must
have some magical component.  Some (such as the Popular Learning
school around Yorye Guliao \'{e} Gauro) go so far as to excoriate the
{\aemott} for keeping the supposed ``spell'' secret, and spin dark webs
of conspiracy in which some dark cabal, power-hungry or fanatical,
wishes to ensure that Ruuda will always need them.

The truth is much more prosaic.  Some time around YS 1000, an unknown
craftsgen in Barra Bantang discovered that properly-shaped slivers of
glass called ``lenses'' could bend light, much as it is bent as it
passes between air and water.  By YS 1040, there are records of
double-lensed devices being used by ship captains to view distant
objects.  Within a decade, the secret of their manufacture had spread
inland along caravan routes to the oasis states of the Karaband, and
particularly to Ossisswe, where almost all of Cherne's dayglass was
and is mined.

The practice of ``mirroring'' dayglass (i.e., applying a thin coat of
electrum or pewter to one side) in order to increase its brilliance
was already known.  Some time before 1090, someone combined a mirror
piece of noon-grade dayglass with a series of lenses to produce the
first \word{skenren lans}: a purely mundane device capable of shooting
a beam of sunlight two hundred strides or more.

We may never know who built the first one, or how the {\aemott}
managed to smuggle them into Ruuda undetected---they \emph{do} keep
some secrets\footnote{At least as interesting as ``how'' is ``who paid
for it''.  Scholars have suggested the Ruudian diaspora, the Society
for Inoffensive Conversation, or the Barsadov dynasty, who had
recently extended their control over the whole of Uws.}.  What is
beyond dispute is that at the start of Heliodor, YS 1091, there was at
least one \word{skenren lans} in every major town and city in Ruuda.

Like a flood pouring across a field after a retaining dyke is
breached, the sunlight of revolution swept over Ruuda in a matter of
weeks.  Dozens of {\aemott} threw themselves at the disorganized Pale
who fled from Jalkelainen harbor, killing them and then kneeling to be
thanked and beheaded by their comrades.  In Pohjoinen, a diasporan
magician named Nastryla's Villepartu traded his ability to eat and
drink for a spell to shield a cart from fire, rot, and catastrophe;
the {\aemott} mounted their \word{skenren lans} on it and raced from
battle to battle through the city's main thoroughfares.

In Ruuda-in-Ruuda, meanwhile, the ``Valiant Seamstress'', Unnegantha's
Polininya, deliberately allowed herself to be captured.  Knowing that
most of the city's senior Pale would be certain to attend her trial,
the {\aemott} concealed their \word{skenren lans} inside a telescope
mounted on a building some hundred strides from Flensing Square and
told the local constabulary that they intended to sell views of
Polininya being harvested.  Thirty-one Pale died in the purified
sunlight that swept across the square that night, chief among them the
same Chemche Angbod Ulpatati who had harvested the rest of Polininya's
family a generation before.  Miraculously, Polininya escaped, only to
be cut down two months later when the Pale's living loyalists retook
the city (discussed below).

Events played out very differently in Ruuda's fourth city, Etela.  As
described earlier, the {\aemott} had never been as strong there, where
they had to compete for loyalty with self-described Vaardians.  On
Heliodor 14th, a band of Vaardians stole the \word{skenren lans} that
the {\aemott} had smuggled into the city, hid it on a barge under a
load of dried squid\footnote{Or cheese mold, or undyed blarthings---as
Kurtitina observed in \citer{A History of the Ruudian Rebellions},
every Vaardian's grandmother was there, and each one hid the
\word{skenren lans} under something different.}, and took it south to
Turnaj\"{o}ki, a fortified port just twenty gallops north of the
border with Uws.  Turnaj\"{o}ki had already been cleared of Pale by
the time the second \word{skenren lans} arrived; using the two
together, the Vaardians cleared the land all the way to the border,
where a startled Uwsian Capitan-Earl was the first outsider to hear of
the founding of the independent Commonalty of Vaarda.

In contrast to its successes in Ruuda's cities, the rebellion's first
wave was much less successful in the countryside.  More than have of
the Pale Remainder who were unalive in 1091 lived on rural estates
called \word{maatilaso}.  Each \word{maatila} was a self-sufficient
unit, typically comprising a small farming village with a mill and
forge built along a single-lane road that connected the local Pale's
mansion house to the rest of the world.  Some were owned and ruled by
a single reclusive Pale; others were home to up to a dozen, whose
parties might last an entire season (and require the supporting
village to be repopulated in their wake).

The Pale Remainder who chose to live in \word{maatilaso} had always
been easier targets for the {\aemott} than their urban counterparts.
As a result, they had evolved a lifestyle that is frequently compared
to the ``genteel paranoia'' of the Thindi aristocracy, and their
dealings with their alive chattel were even more ruthless than their
peers'.  Paradoxically, the people who lived in \word{maatilaso}
villages also tended to be more loyal than city dwellers, some going
so far as to worship the Pale Remainder as ``Angels come again''.  This
is hardly surprising when one considers that some of these villages
had been virutally cut off from the rest of the world for several
centuries, but it meant that those where the {\aemott} had never
gained a foothold viewed the rebellion as a betrayal of the natural
order.

After beating back scattered individual attacks on their estates, the
rural Pale rallied and counter-attacked.  The Commonalty of Vaarda (at
this time just five ridings around Turnaj\"{o}ki) managed to hold them
off, in large part thanks to help from ``volunteers'' from northern Uws.
The Barsadov clan ruled half of the kingdom at this point, and another
fifth followed their blazon thanks to a carefully-crafted web of
marriages and alliances.  Coronel Barsadov ard Yuriy (posthumously
Marshal-King Yuriy I) instantly saw the political coin to be minted by
helping ``liberate'' Ruuda, and ``allowed'' some five thousand of his army
to renege their oaths and go north to join the fight.  A series of
bruising battles, one fought just an hour's ride from the ruined fort
where Uws himself first encountered the Pale Remainder, temporarily
secured the Vaardians' gains.

Further north, things did not go so well for the living.  Pohjoinen
was besieged in mid-Chalcedony by an army containing four hundred
Pale, and twenty times that number of their living followers.  Day and
night they dug tunnels and trenches in the rocky soil outside the city
walls; at least once a week, the Pale fleet assaulted the harbor,
hoping to provoke the city's inhabitants into using their
\word{skenren lans} long enough for the Pale's magicians and artillery
to get a fix on its location.

The Pale Remainder were not so patient with Ruuda-in-Ruuda.  Their de
facto capitol was twice the size of any other city north of Vnir; the
symbolic cost of losing it was immense.  After a rushed assault was
beaten off on Carnelian 21-22, the Pale assembled an army twice the
size of the one deployed against Pohjoinen, pulling troops away from
Jalkelainen and other centers to the north to strengthen it.  Their
first attack on the first of Chalcedony captured the Trnemaia Bridge,
the first crossing point on the Kypsyva River below the harbor.  From
there, the army was able mount a two-pronged attack from the south and
west simultaneously.  On the fourth, they breached the walls of the
Old City; on the fifth, they destroyed the rebels' \word{skenren
lans}.

The massacres that followed were no doubt intended to terrorize the
rest of Ruuda into submission.  Strategically, however, it was a grave
miscalculation.  In the weeks that followed, Gifted birds and other
witnesses spread word of the bloodbath throughout Uws and the
Regimental Kingdoms.  Popular opinion was roused as never before.
Songs, plays, and popular rallies demanded that ``the living must aid
the living''; up and down the west coast, bewildered ``Ruudians'', whose
families had last seen Ruuda six centuries before, became overnight
heroes.  Young gens clamored to take up arms for the cause, and when
Capitan-Earl Fraederiq iye Var\c{c}ennes-Lligar announced in the
Parledoux of Seyferte that he would ``walk through the Herd of Trees
barefoot if need be'' to join the fight, the aging Marshal-King,
Etienne III, took off his boots and handed them to the young
capitan-earl himself.  No one knew exactly what the gesture meant, but
everyone agreed it was magnificent.

The winter of 1091-92 was one of the coldest ever recorded.  The Ocean
froze from the mouth of the Evacsza River in central Praczedt all the
way to Fidditch on the west coast.  A huge storm in the first week of
Chrysoryl buried the besiegers of Pohjoinen in eight strides of snow.
Alive and unalive alike were forced to retreat to whatever shelter
they could find, but the rebellion did not stop.  Necessity's cramped
quarters made the Pale Remainder for the {\aemott} still lurking in
their ranks, while bandits from the Helada Mountains, more accustomed
than lowlanders to the harsh conditions, were able to raid isolated
\word{maatilaso} almost at will.

The winter of 1091-92 was not just cold; it was also long.  Pohjoinen
harbor was still frozen solid at the end of Chrysoprase, fully a month
after the ice would normally have started to break up.  Incredibly,
the city was still free, although it had suffered heavy losses:
already low on supplies when winter set in, many of its people had
succumbed to cold and starvation\footnote{Plays and novels about the
siege, particularly those written in Praczedt, sometimes imply that
Pohjoinen's inhabitants resorted to cannibalism during the siege.  It
must be emphasized that there is absolutely no evidence to support
this: having had their own bodies used as raw material for six
centuries, Ruudians consider eating human flesh, even in extreme
situations, an unbreakable taboo, one which unfortunately extends to
medical procedures such as flensing.  Praczny authors' portrayal of
mothers ``accidentally'' roasting themselves so that their children can
eat therefore tells us much more about the people of Praczedt than it
does about the events in Pohjoinen.}.  Their only consolation was that
the same was happening to the Pale armies shivering on their doorstep.

Then, on the first of Peridot, 1092, something close to a miracle
occurred.  Unknown to all but a handful of {\aemott}, a few survivors
of the rising in the Saarumeva mines a century before had made lives
for themselves in caverns far deeper than human beings normally dared
to venture.  Trading speech for darksight, they lived lives as simple
as those of ungifted animals, dressing in woven snakeskin and only
rarely venturing near the surface.

Some time during that winter, the {\aemott} of Jalkelainen struck a
deal with the Saarumevi.  If the roads were blocked, and ships could
not sail, then Jalkelainen's army---all eighteen hundred of
them---would travel underground.

The story of Dark March is too well known to be recounted
here\footnote{See for example the third volume of Lemmuelen's
\citer{Exploits of the Valorous of Jalkelainen}, or Maatenala and
Urgo-Aedie's \citer{Five Dark Weeks} for a less scholarly, but
eminently more readable, account.}: cave-ins, vapors, toothed worms,
flash floods, wingless bats, suspicious grandmothers, a seam of pure
gold as thick as a man's two legs, a mad hermit who may or may not
have been Uws, or Uws's son, or perhaps just someone who liked to
hoot{\ldots} Of the eighteen hundred who started, only twelve hundred
survived.  But when those twelve hundred came out of the ground half a
gallop behind the Pale Remainder's principal camp southeast of
Pohjoinen early on that Peridot morning, they struck like a smith's
hammer.  Almost a hundred Pale fell in their first assault, and ten
times as many of their alive servants.  Pohjoinen was saved, and with
it, the Fifth Rebellion.

It took another eight years for Ruuda to free itself completely from
unalive rule.  For much of that time, it seemed that a ``light and
dark'' solution would emerge: the Pale Remainder would continue to
govern a much-shrunken domain centered around Ruuda-in-Ruuda, while
Jalkelainen, Pohjoinen, and Etela would form a federation of
city-states, much as they had six centuries before.  Uws and the
Regimental Kingdoms both favored this, as it would forestall the
emergence of a new northern rival for power.  Herkko's Tuure, who
appointed himself Last Defender of the Commonalty of Vaarda in 1094,
also argued for ``light and dark'', and not just because his Uwsian
backers told him to.  Publicly claiming it would give the Ruudians a
chance to regroup for a final assault at some unspecified future date,
he privately felt that a Pale buffer state between ``his'' Vaarda and
the ``fanatics'' further north would be very useful.

Those fanatics---the {\aemott} who now ruled Pohjoinen, Jalkelainen,
and growing swathes of countryside---were in no mood for compromise.
Most were uneducated, or even unlettered; few had any experience of
governing.  Faced with chaos, they improvised, often harshly; anyone
they believed had collaborated with the Pale was imprisoned or
executed, frequently without any chance to defend themselves.  The
Gifted, who had never been subjected to the red harvest, were treated
especially harshly.  No one knows how many cats, horses, bears, and
other thinking creatures died during this period, but most of those
who could run, fly, or swim to safety did so.

The worst incident in this period was the panic in 1095-96, when it
was discovered that several Pale had managed to live in hiding in the
cellars and caverns beneath Pohjoinen for three years.  During that
time, they had rebuilt themselves from harvested parts so cunningly
that their assembly scars could be passed off as war wounds, and
established identities for themselves in the city's burgeoning
criminal underworld to excuse their preference for midnight meetings
and darkened rooms.  Suddenly, the Ruudians' hard-won freedom seemed a
very fragile thing: any stranger in a tavern could be a Pale in
disguise.

An {\aemott} faction called Pure Light emerged from the purges that
followed as the undisputed governors of Pohjoinen.  They divided the
city into \word{maatilaso}, each of which was subdivided into blocks
and ``families'' (membership in which was defined haphazardly).  Each
unit was responsible for policing itself; lapses were punished
collectively.  ``Greeting the sun'' (i.e., rising at dawn to stand in
the sunlight with one's neighbors in order to prove oneself) became
mandatory, and the use of informants became widespread.

The ``light and dark'' question came to a head in 1097, when H.'s Tuure
announced that he was opening negotiations with the Pale.  Two hours
later he was dead, his body swollen to three times its natural size by
the poison one of his chambermaids had painted on his favorite pen.
Having spent several centuries refining the arts of patience and
assassination, the {\aemott} were not about to let anyone take ``their''
rebellion away from them.

A combined army of Ruudians, Vaardians, and Uwsians crossed the
Kuumineva on the ninth of Heliodor, 1098.  Two days later, a larger
force of native and diasporan Ruudians---poorly equipped and barely
trained, but passionately committed to victory---marched over the
Hanging Bridge near the mouth of Saarumeva Gorge.  Bypassing the
remaining \word{maatilaso}, which were by this point heavily
fortified, they drove relentlessly toward Ruuda-in-Ruuda.  Underground
patrols kept pace with them to ensure that the Pale Remainder did not
turn the living's own tricks against them, while a fleet of over two
hundred vessels (many crewed by freebooters who had been promised a
share of the spoils when Ruuda-in-Ruuda fell) sealed off a seaward
escape.

After a series of fierce delaying engagements, the Pale Remainder
retreated behind the city walls on Citrine 23.  Their attackers
settled in for a long siege.  They knew that surrender was not an
option for the Pale; they would keep fighting ``{\ldots}until they had
stripped the last scrap of flesh from the last of their traitorous
followers.''\footnote{See Kurtitina, op cit.}

On 20 Chalcedony, Boelwe's Ulzen's troop, the Mongrel Hundreds,
breached the city's Southeast Wall.  Three days later a Thindi
magician named Patchandouliander persuaded a tower in the Southwest
Wall to shiver itself to pieces.  On the first of Malachite the first
case of scribbles was reported in the besieging army, but it held firm
as gen after gen shook garself to death, gar body covered with the
plague's strange runes.  Siege engines threw the rubble from the
breached walls at the Pale's last stronghold in the warehouse district
next to the dock.  It was only a matter of time.

On the last day of Malachite, the surviving Pale made a break for
freedom.  A fierce storm, obviously magical in origin, scattered the
fleet guarding the harbor mouth.  As the last of their living
followers launched a suicidal attack against the Ruudian line in
Cobblemaker Street, approximately two hundred Pale crowded onto their
three remaining ships and set sail.  The first ship was sunk by a
lucky shot from an on-shore catapult\footnote{Whose crew reportedly
never had to pay for a drink again in their entire lives.}; the second
was rammed, boarded, and sunk, taking five of the besiegers' ships
with it; but the third escaped into the storm.

It was over---or nearly.  Victory would not be complete until the last
\word{maatilaso} in the Powrm Valley fell in Chalcedony of 1100.  Even
then, a handful of die-hard {\aemott} held that the living would not
truly be safe until that last ship, the so-called \ship{Damned Dark
Bird}, was found.

Most Ruudians were too busy to care.  Their country was in ruins.  For
the first time in six hundred years, they were masters of their own
destinies.  How should they govern themselves?  Should they be one
state, or two, or many?  Should gens who had neither collaborated with
the Pale, nor actively supported the living, be tried?  If so, by
whom, and under what rules?  All the passion that had gone into
overthrowing Pale rule was now thrown into these questions.  As
Queren\c{c}ennes-Cuenstans wrote in his memoirs:

\begin{quotation}
To bid a gen ``good morning'' was to invite a disquisition on the
relative merits of written and dramatic examinations.  To ascertain
whether the eels on one's plate were fresh required that one clarify
one's position on what rights Wise Beasts [note: the Gifted] ought
have in the new order of things.  And to carry through a seduction,
one needed nuanced constitutional proposals more than heart-plucking
ballads.
\end{quotation}

Then, in 1103, word reached the north that the \ship{Damned Dark Bird}
had resurfaced.  The Pale Remainder's last ship had sailed halfway
around the world to the Salt Coast---the most inhospitable territory
in mainland Cherne.  Its crew had wrested control of the diamond field
known as Bell Prison from the Bantangui pirates who had controlled it,
and reinstituted the red harvest.  Living human beings were once more
being taken apart to maintain the Pale's unnatural existence.

It was intolerable.  It was an insult to the memory of every
Ruudian---nay, every Chernese, no matter what their nation---who had
died in the struggle to rid the world of such abomination.  Something
had to be done---but what?

In Pohjoinen, a young bookster named Friida's Ryutaanan thought she
had an answer.

\chapter{Well Tarred and Truly Masted}

Friida's Ryutaanan was born in Pohjoinen on Heliodor 8th, 1081, ten
years to the day before the start of the Fifth Rebellion.  That, at
least, is what she later claimed; as her family perished during the
rebellion, and most of the city's records were burned during the
siege, it is impossible to know for sure.

The first surviving mention of her is a despatch dated Tourmaline
1091, which commends a message runner named ``F's R'' for her courage.
In her autobiography\footnote{\citer{The Light of Recollection},
dictated in 1151-3, and extensively edited by Ryutaanan's secretaries
during her exile in 1155-65.  Selections were published in Ensworth
upon her death in 1165 to raise money for her funeral pyre; the
manuscript is stored there in the university's archives.}, Ryutaanan
matter-of-factly recollected the incident:

\begin{quotation}
I was gathering window moss with Haldi and Gurgi (note: two older
girls) when a catapult stone came crashing down on a bakery the
Generous (note: a name the {\aemott} used for themselves in this
period) were using as a guard house.  Several were killed outright,
and many others wounded.  The guard captain told Haldi and Gurgi to
fetch help, but they were too frightened, and ran off.  I told him I
would do it, and did.
\end{quotation}

Ryutaanan was adopted after the siege was lifted by the Tyt{\ae}rs, a
marriage of furriers that had produced several {\aemott} over the
centuries, but which had somehow survived the Pale's purges.  At the
time the family consisted of four husbands, six wives, and some two
dozen children, of whom nine (including Ryutaanan) were adoptees.  For
the next eight years, her days were filled with work, more work, and
study.  She spent three days a week, and the mornings of two more,
working as a brickmason's apprentice\footnote{Like most Ruudian
merchant families, the Tyt{\ae}rs reserved the family business for their
blood children.} to pay off her debt to her adoptive parents.
Orangeday and Bluesday afternoons were set aside for lessons.  In an
unheated single room classroom in an attic on Coppersmith's Street she
mastered reading, arithmetic, and argument so quickly that she was
soon giving lessons to the other children while the school's bookster
nursed a succession of hangovers.

In 1102, squabbling in the conclave that governed Pohjoinen spilled
onto the street.  Having fought ``in silence, in shadow, and in secret''
for five hundred years, many of the {\aemott} found it impossible to
get used to the light of public scrutiny.  The general populace didn't
even know who was a member of the largest faction, Pure Light; it made
its proclamations by snatching passersby off the street, hectoring
them, and releasing them with armloads of letters to distribute.

Other factions, such as Clear Dawn and Bright Reflection, were quicker
to adjust.  Borrowing a verse from the governments of the Regimental
Kingdoms, they held a mock-Parledoux every Purplesday at which anyone
with three silver pennies to gar name could offer up a petition for
debate by the faction's leading members.  Several proved themselves
skilled orators, and during the spring and summer of 1102 their
sessions were the most popular entertainment in Pohjoinen.

Matters came to a head in Malachite, when Pure Light condemned the
debates as a ``scurvy foreign invention with no natural place in Ruuda,
imported solely for the purpose of sewing [sic] dissension among the
Generous''.  That Purplesday, speaker after speaker rose to challenge
the proclamation.  Who were Pure Light to challenge the patriotism of
others?  Did anyone know for a fact whether they had actually fought?
Or---whisper it---was there any proof they were actually \emph{alive}?

At 21, Ryutaanan was what a contemporary described as ``the clearest of
the clear''.  She did not doubt Pure Light's right to govern, or to
govern as they had fought.  Like many of the faction's advocates, she
argued that remaining secret would prevent the government from
succumbing to the ``pomp and pride of kings and marshals'' (as a
contemporary catchphrase put it).  It was also a form of insurance: if
the Pale Remainder ever reappeared, the {\aemott} would be ready.

Then, on Malachite 19, the Niemenin brothers broke ranks.  First
secretary and chief advocate of Clear Dawn respectively, they
announced to a stunned gathering that they were also members of Pure
Light.  They claimed they had been ordered to use the most extreme
language they could in their criticism of Pure Light in order to
provoke public sympathy for it.  They also revealed that Pure Light
had been stockpiling weapons in two granaries under Apteraalo Bridge
in case ``more persuasion than voice alone can provide'' was needed.

To this day, no one knows if the Niemenins were telling the truth, or,
as the next morning's pamphlets claimed, playing a double game.  After
two days of increasing tension, though, Bright Reflection echoed Clear
Dawn's demand that Pure Light reveal themselves and take part in
public debates to decide the future of Pohjoinen.

``How shall it all end?'' Ryutaanan agonized in her journal.  The answer
was not long in coming.  On Malachite 24, Niemenin's Buurlo's throat
was cut while he was sitting in the privy.  The shock of the
assassination drove crowds into the streets, demanding that Pure Light
give up whoever was responsible for the crime.  Pamphlets ordering
calm appeared on the morning of Malachite 25, but no one paid them any
heed.  This killing had not been for the protection of Ruuda: it had
clearly been a political murder, and it shook Pure Light's support to
its foundations.

On Malachite 27, three wheelwrights announced that they were members
of Pure Light, and publicly apologized for their faction's role in the
death of Niemenin's Buurlo.  A day later another five gens stepped
forward; a day after that, fifteen.  The trickle quickly turned to a
flood: by month's end, over six hundred people had publicly proclaimed
that they belonged to Pure Light.  Of those, two hundred also said
that they were withdrawing from it in protest over its actions.

The events of that fall showed just how fragile the new Ruudian state
was.  Without a common enemy to hold them together, it seemed, the
native and diasporan factions might tear the country apart.  ``What was
needed,'' Ryutaanan later wrote, ``Was some great purpose akin to that
we were accustomed to, something seemingly impossible that we might
strive to achieve.''

Many thought that ``great purpose'' should be the integration---by force
if necessary---of Vaarda into Ruuda.  Thanks largely to its proximity
to Uws, Vaarda had started rebuilding even before the rebellion was
over.  By the winter of 1102-03, it seemed that every major building
in Etela was encased in scaffolding.  The common people of Pohjoinen
and Ruuda-in-Ruuda, many of whom still counted a second bowl of
porridge in the day a luxury, muttered darkly about their southern
cousins' ostentation, and about how much of the reconstruction was
being paid for by ``incomers''.

In Heliodor 1103, however, the news that the Pale Remainder had taken
over Bell Prison, and reinstituted the red harvest, reached the north.
Overnight, there was only one topic of debate: what should be done?
Pure Light and other {\aemott} factions called for an immediate attack
to wipe the unalive scourge from Cherne once and for all.  Perhaps
surprisingly, their call was echoed by moderates in the diasporan
community and Vaarda.  While this may have been a ploy to appear
``purer than Pure''\footnote{Kurtitina (op cit) is the most prominent
exponent of this interpretation.  Citing some ambiguous entries in the
personal diaries of Vaardian councilors, she argues that their
``disagreements'' over the makeup and aims of an expedition were
carefully calculated to keep the debate churning, while giving them a
pretext to begin construction of a navy.  Others (including the
present author) feel that this gives the squabble-prone Interim
Council too much credit---as Ld.\ Cal\c{c}aere tartly observed, ``Any
group capable of such subtlety and subterfuge would likely not have
locked themselves out of their own meeting chambers on so
distressingly regular a basis.''}, it is also true that, whatever
disagreements these groups may have had with the {\aemott} about how a
free Ruuda should be governed, their hatred of the Pale Remainder was
in no way dilute.

But how exactly was such an attack to be mounted?  If Ruuda was the
upswept wing of continental Cherne, then Bell Prison was its
underbelly, four thousand gallops away as the duck flew, and three or
four times as far by sea.  No one had ever made such a journey without
magical aid.

Of course, that didn't prevent every ``hay bale sailor'' in Ruuda from
putting forward a plan.  ``We approach the matter directly,'' wrote one
in an anonymous pamphlet that circulated that summer:

\begin{quotation}
The Fleet shall be made up in equal parts of vessels from each major
region, and shall assemble in the mouth of the Kravriye in the spring.
Thence, it shall sail to Nevy Rav to be provisioned, from which it
will make all haste through the Gulf of Szigor\'u to the great Empire
of Thind.  Reprovisioned, it shall proceed south around the tip of
Barra Bantang, confident that none of those waters' infamous pirates
would have the audacity to attack such a force.  Once past the
Whirlpool, the fleet shall turn north, striking directly across deep
water at its target so as to ensure complete surprise.  If any survive
uncursed, they may retrace their path to return to heroes' collars
they will have justly earned.
\end{quotation}

In just one hundred and fifty words, the author reveals how little
most Ruudians knew about the world after almost six centuries of
isolation.  Never mind where those ships were going to come from; the
Pederov family who governed Nevy Rav had been the Barsadovs' greatest
rivals for most of the preceding century, and it was unlikely they
would allow a fleet crewed by Ruudians (whom they viewed as the
Barsadovs' clients) to make landfall, much less provision it.  It was
even less likely that the dragon, Sulk, would allow a war fleet to
pass through the Gulf of Szigor\'u, or indeed come within a thousand
gallops of her nest on Sullair Minor.

The next challenge was Thind, which viewed all other nations on Cherne
as wayward vassals.  They might allow a Ruudian expedition to pass
unmolested; they might equally well enslave the ships' crews, or have
the vessels painted gold in honor of the emperor's nameday.  As for
Ini Bantang and Barra Bantang, the ``infamous pirates'' so blithely
dismissed by the pamphlet had long since joined forces to create the
Twin Admiralcies; by the early 1100s, they were de facto rulers of the
city-states of Antharwaddy, Teberjaya, Pejangorian, and Yanaunchang,
and were pressuring the southern Thindi port of Yadanapore.  Any fleet
the Ruudians might assemble would have been a twig in a forest
compared to those of the Twin Admiralcies.

Finally, the Whirlpool was no longer impassable, as it had been during
the Age of Heroes, but it travel around the tip of Barra Bantang was
still extremely hazardous for anyone not familiar with it.  So was
crossing the two thousand gallops of open ocean that separated it from
the Salt Coast: while Ruudians sailors were no strangers to sudden
storms, they would never before have encountered the region's giant
eels.

Other voices in the debate of 1103 therefore favored an overland
attack, though this was if anything more problematic.  Crossing Uws
from the Heladas to the northern arm of the Brumosos would be
straightforward, providing the politics could be worked out.  But what
then?  The direct route---over the Brumosos, across the Great Plains,
and through the Karaband---would put an army at the mercy of one
Darpani tribe after another, only to have to find a way through
Cherne's largest desert.  The indirect routes---through Praczedt,
Thind, and Barra Bantang to the east, or the Regimental Kingdoms to
the west---were just as daunting.

Of course, there was always a third option: magic.  In speech after
speech, Ruuda's booksters and magicians put forward ever-more-fanciful
schemes.  Find Uws, cure him, and let him figure out what to do; find
Uws, slay him, take his boots, and have their new wearer carry the
army south one gen at a time; use opals, dragon scales, a magic mirror
(the construction of which would be a quest in its own right), or have
every magician in Ruuda cast a single great spell of persuasion over
some Darpani tribe and let them sort it out.  Yes, this would leave
those magicians witless and incontinent, but that was no greater
sacrifice than countless thousands had already made{\ldots}

In retrospect, the most interesting aspect of these proposals is how
seriously they \emph{weren't} taken.  In the first centuries after the
Uncertain Angels destroyed themselves, matters of state had routinely
been decided by magic.  Whole nations---Thind, Uws, Praczedt, and the
Regimental Kingdoms---had magical origins; it would therefore have
been natural for people to look to magic for a solution to ``the Pale
problem''.

Instead, all serious discussion focused on mundane options.  This may
reflect the fact that contemporary Ruudians were less familiar with
magic than most Chernese (since it had been tightly controlled during
the Pale supremacy), or that Pale rule may have made Ruudians
distrustful of magic in general.  Alternatively, the fact that they
had overthrown the Pale Remainder without the aid of a hero in the
classical mode may have given Ruudians a confidence that no other
Chernese people had.

\begin{center}
* * *
\end{center}

Ryutaanan was inducted into Pohjoinen's Pure Light in the summer of
1103.  Under Pale rule, her duties would largely have consisted of
acting as normally as possible, so as not to arouse suspicion while
working her way into a position of trust.  Earlier than most,
Ryutaanan realized that ``acting normal'' would not help advance Pure
Light's interests in this new era of openness.  Faction members
\emph{had} to draw attention to themselves in order for their views to
be heard.

Ryutaanan therefore began accompanying progressive faction members,
such as Aarbi's Perguuran and Daanimo's Daanima, to the city's
Purplesday debates.  Her fearlessness made her a formidable debater:
contemporaries recorded that she would take on anyone and any subject,
no matter how loudly the crowd clapped.  She quickly mastered the
rhetorical tricks imported by the diasporans flooding into Pohjoinen,
many of whom had cut their teeth in civic councils in the Regimental
Kingdoms.  She also displayed a talent for research that was rare
among the relatively unsophisticated Ruudians; time and time again, it
seems, she trounced her opponents by rattling off facts and figures in
a way that made at least one ask whether, ``No longer governed by the
cursed, we are now to be governed by the magical?''

Much to Ryutaanan's chagrin, Pure Light did not select her to be one
of its representatives at the Great Debates held in Ruuda-in-Ruuda
between Heliodor and Topaz of 1104.  There, over the course of eleven
weeks, the two hundred members of almost twenty factions argued,
bargained, threatened, and cajoled.  Some were seduced; others
poisoned or bewitched, and at least one turned out to be a Praczny
merchant whose deafness and heavy accent had mistakenly given
onlookers the impression of sagacity.  Everyone understood that the
issues being debated were fundamental to the future of their nation.
Was it to be one nation, or two?  Or a loose confederation of
city-states, as some argued was ``natural''?  And who was Ruudian?
Those who had lived under Pale rule, certainly, but what of diasporans
whose families had resided in the Regimental Kingdoms for centuries?
Should pirates who had preyed upon Pale shipping for their own gain,
brather than as an act of rebellion, be enrolled in the new state (or
states)?  Should the increasingly harsh treatment of the Gifted, whom
many viewed as Pale collaborators, be formalized in law?  And what, if
anything, should be done about Bell Prison?

Over the course of the Debates, that last question became a seed
around which different opinions pearled.  On one side were the
``Diplomats'', so-called because their favored overland attack would
require Ruuda to negotiate treaties with its immediate neighbors.
Most diasporans fell into this camp, as did the Vaardian delegation
(many of whose members had put their bloody thumbprints on just such a
treaty with Uws the year before), and many of the better-educated
``native'' Ruudians.

On the other side stood the ``Admirals'', who preferred a seaborne
assault.  This, they argued, would not require permission or
assistance from anyone.  Mistrustful of their neighbors' intentions,
Admirals believed that Ruuda should become a major naval power.
Building a fleet capable of attacking Bell Prison would be either a
step toward this, or proof that it had been accomplished.  Almost all
rural {\aemott} were Admirals, despite (or because of) never having
been to sea.  Many urban {\aemott}, and a scattering of fanatical
diasporans, made up the rest of the party.

The dividing lines between the two sides were never as clear in
practice as they seem in retrospect.  Bright Reflection's leaders, for
example, advocated a seaborne attack \emph{and} negotiating treaties
of support with Uws, Thind, and the Twin Admiralcies, but were unable
to carry enough of the faction's spear-and-shield membership with them
to make their ``honey and whisky'' strategy viable.  Equally, many
Vaardians were privately in favor of building up a strong navy, not
least because Etela was fast becoming Ruuda's major commercial center,
and was already suffering the depredations of ``renegades'' who were as
happy to prey on the living as on the unalive.

Miraculously, by the time the Great Debates adjourned at the end of
Topaz, a handful of major decisions had actually been made.  First,
Ruuda was divided into five regions, corresponding to the four major
cities and the Saarumevan underworld.  Delegates to the following
year's debates would represent regions, rather than factions; exactly
how each region was to choose its representatives was left
unspecified.  Second, the peoples of Jalkelainen and Etela were to
establish good relations with Derway and Bruyere to the west, and Uws
to the southeast, respectively.  At the same time, every fifth ship
constructed in Ruuda's harbor was to be ``fitted for war'', though
again, exactly what this meant was left unspecified.

And finally, after numerous minor pronouncements on taxation,
censorship, and the re-institution of biennial examinations, came the
now-infamous ``Declaration on the Demonstration of Loyalty''.  Those
whose ``collateral relations'' had not ``shewn strong and loyal resolve
in the recent struggle for liberation'' would not be allowed to speak
at future debates, and could not henceforth acquire property, though
they would retain title to whatever they already owned.  Everyone
understood that this was aimed solely and squarely at the Gifted.
When copies of the debates' decrees reached Pohjoinen, Ryutaanan seems
to have been among the few who realized that the Declaration would tip
the scales in favor of the Admirals: given the influence of the Gifted
in Bruyere, Derway, and western Uws, the Declaration made it
politically impossible for their rulers to form too close a
relationship with the new Ruuda\footnote{The Declaration was amended
in 1105 to apply only to birds and landgoing animals, so as not to
alienate various parliaments of whales whose goodwill was essential to
the operation of Ruuda's fisheries.  This gesture actually seems to
have made relations with the Regimental Kingdoms worse, as it removed
the last shreds of ambiguity behind which apologetic Ruudians had
sheltered.}.

Given her loyalties, Ryutaanan should have sided with the Admirals,
but as she explained in her memoirs, ``My head heard all the reason in
their arguments, but my heart yearned for faraway lands.''  In her
early twenties, unmarried, and increasingly influential, it is hardly
surprising that she would see advantages in closer ties with such
``exotic'' places as the Regimental Kingdoms, Uws, Praczedt, and Thind.
By 1104, chocolate was being imported over the Helada Mountains from
the Flying Mountain's northern stopping point, Normous Berth.  Silk
had appeared too, most famously at the Turning Moon Ball of 1105,
where the tight fit of the attendees' dresses and breeches scandalized
and fascinated the whole of Etela.  Magic, which had been so tightly
regulated by the Pale Remainder, had become fashionable among the
well-to-do, and a national duty among the {\aemott}.  More than a few
``dandies and die-hards'' lost their wits, their teeth, or their ability
to sing in tune as they tried to master spells that they hoped would
allow them to see what was happening at Bell Prison, half a continent
away\footnote{Lemmuelen (op cit) lists over two hundred {\aemott} who
made debilitating or disastrous bargains with the Infinite toward this
goal.  It wasn't until the Society for Inoffensive Conversation's
\citer{Guide to the Persistent} began circulating in 1110-11 that
Ruudians learned of Lady Kembe's proof of the inverse relationship
between the distance and accuracy of scrying.  Lemmuelen goes on to
argue that the false visions given by the handful who ``succeeded'' in
seeing Bell Prison had a significant influence on the planning and
execution of the First Expedition.  However, Kurtitina argues equally
that the contradictions between these visions led Ruudians to distrust
them all.}.

Ryutaanan gave up her teaching position in the spring of 1105 to
devote herself to politics.  She was by this time a ``lieutenant'' in
the East Wall \word{maatila} in Pohjoinen, a district of perhaps 2000
gens.  Five blocks long and two blocks wide, the \word{maatila} was
bounded by the swift-flowing Potteleva, furriers' and tanners' shops,
the wall after which it was named, and, to the north and west, the
sturdy four-story mansions of Pohjoinen's emerging merchant elite.
Ryutaanan's duties spanned the range from organizing the
reconstruction or removal of buildings damaged during the Rebellion (a
process that would not finally be completed until thirty years later),
to overseeing charity for the poor and invalid.

She was also responsible for the education of the young.  One of the
Debates' more prosaic resolutions required every
\word{maatila}\footnote{``Or any other governed or regulated body of
similar size and intent,'' which gives an indication of how chaotic
Ruuda's governance was in practice at the time.} to arrange tutelage
for anyone wishing to sit for the biennial examinations.  These were
divided along the Regimental model into an examination of general
literacy and simple mathematics, which was usually taken at the age of
sixteen, and a craft-specific examination taken in one's early
twenties.  Ryutaanan herself never sat either, but worked tirelessly
to ensure that as many of her \word{maatila}'s children as possible
had the opportunity to do so.  Contemporaries recorded that if a
bookster's lessons were not up to her exacting standards, she would
take the lectern herself, telling the hapless gen to ``sit, listen, and
learn'' along with gar pupils.

Her direct approach to improving instruction generated many
complaints, some of which are still in Pohjoinen's city archives.  It
also won her a loyal following among both young and old in her
\word{maatila}.  Despite several petitions, though, Pure Light passed
her over once again when the time came to choose delegates for 1005's
Great Debates.  Her youth, and the fact that she hadn't actually
fought in the Fifth Rebellion, both counted against her; so too,
undoubtedly, did her Diplomatic stand on the Expeditionary Question.

Ryutaanan was disappointed, but not as much as she might have been.
During that winter, her working relationship with Aarbi's Perguuran
had blossomed into something more.  Ten years older than Ryutaanan,
and half a head shorter, his small eyes and unfashionably narrow jaw
would later lead his political enemies to caricature him as ``half man,
half pig''.  He was, however, one of the most widely read gens in
Ruuda, and his correspondents spanned the breadth of the continent.
Ld.\ Wo\"uter the Elder described him as, ``{\ldots}able to split an
argument in half with one blow, as would a gemsmith a diamond,'' while
Coronella Barsadov ard Innu, whose instinct for advantage had as much
to do with her family's ascendancy in the early 1100s as her son
Yuriy's successes on the battlefield, once opened a letter to
Perguuran with, ``Ld.\ sir\footnote{Perguuran was not actually awarded
the title ``Learned'' until shortly before his death in 1147, but never
corrected those who applied it to him.}, having read the remarks in
your latest [note: an argument in favor of imposing quotas on North
Ocean fishing], I am grateful that you have no quarrel at law with my
family.''

Unsurprisingly, Pure Light chose Perguuran to represent Pohjoinen in
1105's Great Debates.  This created a dilemma for Ryutaanan: should
she travel to Ruuda-in-Ruuda with him, or continue her work in the
East Wall \word{maatila}?  To quote her memoirs once again:

\begin{quotation}
I had made up my mind to do the responsible thing, and remain behind,
but then Daanima [note: Perguuran's debating partner, Daanimo's
Daanima] was stricken with ptyche.  As a place for him aboard the
\word{laiva} carrying the delegation to the capitol had already been
paid for, Perguuran pressed upon me that it would be wasteful to do
other than accompany him.
\end{quotation}

A simple case of salt deficiency---probably brought on by the heavy
drinking that later led to Daanima's expulsion from Pure Light---was
therefore the hinge on which so much of subsequent Ruudian history was
to turn.

\begin{center}
* * *
\end{center}

Ryutaanan, Perguuran, and two dozen others left Pohjoinen on the first
of Citrine, 1105.  It was Ryutaanan's first time at sea; half a
century later, the scene was still fresh in her mind:

\begin{quotation}
We departed under a brisk wind that lent the ship such speed that a
double wave curled back from her prow.  The crew busied themselves
with ropes and sails and strapping down odd-ends of cargo that seemed
secure enough to our landlubberly eyes.  The harbor gulls followed us
out a half gallop from shore, squawking for scraps, until the captain
shouted that she was carrying passengers, not fish, at which point a
Gifted among them shouted back some good-natured abuse and led his
companions away.  Perguuran leaned over the side to catch some spray
in a tankard and offered it to me as my first draught of ``real'' water.
It was so fresh and cold that I almost choked on it.
\end{quotation}

They reached Ruuda-in-Ruuda two weeks later, having stopped twice
along the way to gather more delegates.  They were met outside the
harbor by a small cutter flying the twelve-rayed sun that would, that
summer, become Ruuda's new blazon.  Ryutaanan recorded the exchange
between its customs inspectors and her ship's captain with some
amusement:

\begin{quotation}
{\ldots}at which the captain expostulated, ``By d--n, whelp, I didn't
fight nineteen years just to hand over good brass to a headscratcher
in a starched collar!''
\end{quotation}

Pure Light's delegation took rooms on Tinsmith Street, a brisk
ten-minute walk from the converted auction house where the Debates
were held.  Ryutaanan was immediately caught up in a whirlwind of
preparation.  While Perguuran and others argued, cajoled, and railed,
she buried herself in the city archives, a dayglass lantern clipped to
her broad canvas shoulder belt, scraps of paper (still a precious
commodity in the north at that time) and a stub of hardened charcoal
in her hands.  Bright Reflection would support Pure Light's motion to
incorporate surviving bands of irregulars into regional militias, but
only if Pure Light would back a two-pence reduction in the salt tax:
what revenue would be lost?  And how would the adjusted tax compare to
those of Uws and Derway?  An independent delegate representing three
\word{maatilaso} in the Saarumeva Valley claimed that villages had
jurisdiction over tree planting and harvesting before the Pale
Remainder invaded---was she right?  How much land would be affected if
that rule was restored?  And---whisper it---was there anything in the
archives, even so small as initials scribbled beside an informer's
report, that could be used to smear this delegate or that one?

Ruudian politics, Ryutaanan quickly realized, was turning ugly.  The
division between Diplomats and Admirals was widening, not narrowing.
What to do about Bell Prison, how closely Ruuda should involve itself
with its neighbors, and whether the new state should have a strong
central government, or be a federation of semi-independent regions,
were no closer to settlement than they had been eight months
previously.  When Tellervo's Maarit arrived on Citrine 29 at the head
of her troop of battle-hardened veterans\footnote{The ``Red Knees'', who
earned their name from Ugli's Tellervo's statement after the first
battle of Partle's Bridge that they had ``stood knee-deep in their own
blood'' to hold the ford below the bridge while the rebels retreated.
Tellervo's Maarit took command of the troop in 1097, after which it
spent as much time suppressing bandits in the southeastern Heladas as
it did cleaning up the last few Pale \word{maatilaso}.}, no one
believed her claim that she had just wanted to show her ``friends'' the
big city's lights.  A few of Ruuda-in-Ruuda's delegates responded by
sending their families to the countryside; again, no one believed them
when they said it was in case any of the debaters had brought plague
with them.

The second Great Debate opened on the third of Topaz, three days later
than scheduled.  The sessions were stormy from the start.  As their
first act, the debaters adopted a twelve-pointed sun, gold on white,
as Ruuda's blazon.  When a motion to allow regions to amend it with
their own sigils to it was narrowly defeated, though, the Vaardian
delegation announced that they would hang one of their own
devising---a stylized pine tree on a white-over-brown background, with
the sun rising behind it---as well.  And when a slim majority raised
their hands in favor of a new tax on the fishing fleet, to be put
toward construction of a standing navy, several prominent Diplomats
publicly renounced seafood, and called upon others to do the same.

Watching from the sidelines was Derway's ambassador, Majeur Callum apt
Connomenaer.  A seasoned observer of courts and parledoux in the
Regimental Kingdoms, he was alternately amused, inspired, and appalled
by the confusion of the Debates.  ``They have as little discipline as
squabbling children,'' he confided to his wife\footnote{From a letter
reproduced in Ld.\ Ernaest Guillaume \'{e} Kristen's \citer{Confident
in Themselves Alone: The Life of a Derwegian Noble Family
1047-1221}.}, ``Yet upon an instant, may reach such heights of noble
intelligence in their arguments as to put our grand collegians to
shame.''

A chance meeting at a rat fight led the majeur to offer his services
as an advisor to the contingent from Jalkelainen, who were struggling
to balance the needs of region and faction.  Much to his surprise,
Majeur Callum was soon pressed into a greater service: from the second
half of Topaz onward, he found himself lecturing to an audience of
booksters, debaters, and others on the theory and practice of
delegatory government.  ``They have made a d---ned scholar out of me!''
he complained good-naturedly to his private secretary\footnote{Enna
Gwydion \'{e} Laurael, who bore him two illegitimate children before
being imprisoned for being a Seyfertois spy in 1117.  Majeur Callum
petitioned to have her freed ``to care for her young''; his use of
phraseology normally reserved for rearing animals obviously did not
help their relationship, as she immediately took service with a cousin
whom he reputedly loathed.}, though he must have realized how much
influence this gave him over the direction of the Debates.

Ryuataanan was taken to one of Majeur Callum's lectures (she uses the
term ``dragged'') toward the end of Topaz.  She quickly became a regular
attendant, scribbling summaries to give to Perguuran and arguing over
how applicable the Regimental experience was to Ruuda.  Most of the
gens she argued with were, like her, junior delegates or seconds who
had taken up residence in the archives.  As the summmer wore on, they
found that they increasingly had more in common with each other than
they did with their ``superiors'' deadlocked in the debating chambers.
``A government's first responsibility is to govern,'' Majeur Callum
repeatedly reminded them.  By the time the Debates broke up in
acrimony and name-calling at the end of Carnelian, Ryutaanan and her
``archivist'' colleagues were inclined to agree.

Ryutaanan worked harder than ever that winter, teaching, organizing,
and writing letter after letter to her new-found comrades from
Ruuda-in-Ruuda's archives.  She copied long passages from Bolkov's
Uwsian translation of di Juenez's \citer{Treatise on the Employment of
Law} to circulate among her peers, scribbling last-heartbeat thoughts
in the margins.  Many of those letters have survived, some in their
original form, but more as copies passed hand-to-hand in chocolate
houses during that long, loud winter.

Despite her youth (she was still only twenty-four years old),
Ryutaanan's letters quickly earned her a reputation as one of Ruuda's
most incisive political thinkers, and the chief advocate for a new
approach to the questions that had deadlocked the First and Second
Debates.  Yes, there should be a strong central government, but only
of regions that chose to take part; any that didn't (meaning Vaarda)
should be allowed to go their own way, ``{\ldots}lest we throw off
occupation only to become occupiers ourselves.''  And yes, Ruuda owed
it to those who had fallen to eradicate the last of the Pale
Remainder.  ``But who among us has ever seen the verdant jungles of
Thind, or passed by the Sea of No Dreams [note: Cap di Per\c{c}alle]?
Who among us then has it within gar competence to decide, shall we
sail or shall we march?''

Above all, she wrote, Ruuda must decide \emph{how} it was to govern
itself.  Should rank be earned purely through competitive examination,
as it had been in Angelic times, and still was in some parts of
Praczedt?  Or should there be a hereditary element as well, as in Uws
and the Regimental Kingdoms?  Should representation be factional,
regional, professional, or, as Daanimo's Daanima believed, some mix of
the three?

It was this question that led to the break with Perguuran that had
been brewing since the summer.  Like many veterans, Perguuran believed
that those who had fought hardest against the Pale Remainder had
thereby earned the right to govern Ruuda.  Moreover, he said (loudly,
publicly, and regularly), no one else could be trusted with the
task---certainly not ``foreigners'' from the Regimental Kingdoms, ``Who
call themselves Ruudian, but can scarce speak the language,'' or ``Their
soft-palmed bootlacers, too overawed by flowery phrases.''

In a series of increasingly heated attacks, Perguuran and others who
had been faithful to the {\aemott} cause in the long, dark years
before the Fifth Rebellion singled Ryutaanan out as a symptom of what
was going wrong with ``their'' victory.  Ryutaanan fought back by
challenging her adversaries to justify their hold on power.
Frustrated by how little the Great Debates had accomplished, and no
doubt feeling that \emph{they} deserved a greater voice in Ruuda's
governance, many of Pohjoinen's well-to-do rallied to Ryutaanan's
cause.

The picture of Ryutaanan that emerges from diary entries, letters, and
other records is of a courageous but stiff-necked woman, always sure
of her opinions even when they were changing as often as the weather.
Tall, with her hair cropped short to show off a tall brow and striking
gray eyes, she seemed to contemporaries to be constantly in motion,
sometimes carrying on two conversations while annotating a letter or
checking over receipts from some building project in the East Wall
\word{maatila}.  ``She sleeps,'' wrote a contemporary wryly, ``But only
as does the clockwise petrel---on the wing.''  Her appetite for
chocolate was legendary: she and Daanima often worked through the
night, she growing increasingly agitated under the influence of her
favorite drink while he became increasingly morose under the influence
of his.

``You must plan your campaigns for lecterns at the Debates as you would
plan any other battle,'' Majeur Callum wrote to her and her fellow
archivists in a circulated letter dated Sapphire 5, 1106.  Ryutaanan
took his advice to heart.  Early in Chrysoprase, she renounced her
membership in Pure Light, declaring that ``factionalism'' could not meet
the needs of the new Ruuda.  Clear Dawn and Bright Reflection both
courted her, but she held fast: henceforth, she would speak for the
East Wall \word{maatila}, and no one else.

Perguuran led the charge against his ``wayward charge''.  At the
Purplesday Debate on Sapphire 12, he rose on a point of procedure,
pointed dramatically at Ryutaanan, and demanded to know what right
``that interloper'' had to be present on the debating floor?  Ignoring
the uproar that followed, Ryutaanan calmly rose to her feet and held
up the petition her students had quietly circulated throughout East
Wall during the preceding month.  ``This is my right,'' she announced,
passing it to a page to be taken up to the debate's gaveleer for
inspection.  ``Where, colleague, is yours?''

On cue, three other debaters---Eirika's Juuso, Kylliki's Aatu, and
Rauha's Terhenaar---rose and passed forward petitions of their own.
In a clear, strong voice, Ryutaanan delivered what she later described
as the most important speech of her career.  Ruudians did not govern
Ruudians by right of conquest, she declared.  They governed because
the Ruudian people had chosen them to govern:

\begin{quotation}
Thus it was when the Uncertain Angels held the world in their care;
thus too it was in the years after, before the blight of unlife fell
upon us.  Let our grandchildren, or theirs, choose other if they
would---humbly, I submit that we have too little practice in governing
ourselves to choose other now than emulation of those great days.
\end{quotation}

With that appeal to ancient tradition, Ryutaanan and her colleagues
set in motion the great innovation in Chernese politics since the end
of the Age of Heroes.

\begin{center}
* * *
\end{center}

Predictably, Pure Light's elder statesmen\footnote{States\emph{men},
because all of the senior members of Pure Light in Pohjoinen at the
time were male.  Several writers have suggested that this oddity
biased them against the position put forward by Ryutaanan and her
allies, though no one has advanced a convincing explanation of why or
how.  On the other hand, Perguuran's attacks were undoubtedly rooted
in the end of his romantic relationship with Ryutaanan.} reacted with
scorn.  ``We have governed ourselves for six hundred years!'' thundered
Anssi's Ilmari.  ``We have made laws, raised taxes, and passed
judgment.  That we did so in hiding is no fault of ours.  That some
who shed no blood to win this nation's freedom would forget those
centuries is most certainly a fault of theirs.''  Lieutenant of the
Dockside \word{maatila}, and the only surviving member of Pohjoinen's
first \word{skenren lans} crew, Ilmari was an instinctive brawler; his
appeal to ``the blood we shed'', and the thinly veiled threats that
accompanied it, would have been expected.

Perguuran's attacks were all the more forceful for being less emotive.
Ryutaanan and her fellow ``Consenters'' (as they quickly became known)
were hopelessly na\"ive, he said---any government that depended on the
consent of the governed would be no better than anarchy.  ``If a bandit
says, 'I do not recognize your authority,' should then the sheriff
halt the chase and wave him away?'' Perguuran asked.

Clear Dawn's chief spokesgen in Pohjoinen, Pa\"ivi's Aatu, was equally
hostile initially.  Three weeks after Ryutaanan's speech, however, she
changed tack, having been instructed by her factional superiors in
Jalkelainen that Clear Dawn was going to recognize the Consenters'
right to lecterns at that year's Debates.  This was not simply a
tactical move: Clear Dawn's leadership had been deeply influenced by
Majeur Callum's lectures on governance, and proved fertile soil when
reports of Ryutaanan's speech reached them.

A show of hands on Chrysoprase 6 confirmed lecterns for Ryutaanan,
E.'s Juuso, K.'s Aatu, and R.'s Terhenaar.  Unwilling to concede
defeat, Perguuran demanded to know how the quartet would travel to
Ruuda-in-Ruuda, ``Since they are most certainly not welcome among
veterans.''  Ryutaanan responded by passing a sack around her East Wall
\word{maatila}.  In just two days, she delivered ``two hundred thirty
rings three quarterings and miscellaneous loose metals'' to P.'s Aatu
(about nine hundred rings in today's money).

Two months later, in early Heliodor, the Consenters boarded the
\ship{Cloud}, a Derwegian \word{spas\'arth\'ach} bound for Etela.
Once again. Ryutaanan's joy at being on the Ocean---even in spring,
with ice floes still evident---is palpable.  Her memoirs contain
fresh, vivid accounts of weather and wildlife, and sharp observations
on her fellow passengers.  She describes a fishing smack that
paralleled their course for a while, its three-gen crew yelling
friendly insults at the \ship{Cloud}'s Derwegian crew in such thickly
accented Ruudian that she had to ask her fellows to translate.  A
black-and-red sail on the horizon sent the ship's magician up the mast
while the crew donned helmets and limbered their crossbows, but the
pirate decided to seek easier game elsewhere.

And at some point during the two-week voyage, Ryutaanan made one of
the few impulsive decisions in her long career.  She, E.'s Juuso, K.'s
Aatu, and R.'s Terhenaar decided to marry.  A year of working
sock-in-boot with one another undoubtedly contributed to the decision;
so too did the need to make the most public declaration possible that
they intended to stand together, come what may.  They were also
entering the summers of their lives: while their close intellectual
companionship was undoubtedly rewarding, it is hardly surprising that
they craved the physical as well.

``No sooner did we tie ribbons around each other's necks,'' Ryutaanan
remembered wryly, ``Than the entire crew, the captain no less, crowded
'round E.'s Juuso to wish him the health and vigor he would need, now
being one with three wives.  With his blushes and laughter, I had
never seen E.'s J. so happy.''

The wedding took place aboard ship at noon on Heliodor 22.  In front
of a mixed stew of Derwegian sailors and debaters from Jalkelainen,
Pohjoinen, and sundry points in between, the foursome swore in blood
to cherish and defend one another so long as the marriage should last.
None of their blood-oaths proved binding, but no one let that sour the
mood.  After a brief, but noisy, charivari, Ryutaanan and her new
wives and husband retreated to the captain's cabin, which he had
generously given over for the night.

By the time they stumbled onto deck the next morning for the
traditional cold bath and hot drink, the coast starboard of the
\ship{Cloud} was dotted with small farms.  They had sailed past a
fishing fleet during the night, Ryutaanan was told, and its whale
pilot had told them that they were a scant thirty gallops from
Ruuda-in-Ruuda.  The \ship{Cloud} had made excellent speed.  As E.'s
Juuso washed and braided his wives' hair, they discussed plans for the
coming Debates.  There could be as many as twenty in their ``faction'',
if all had gone well, and they would be able to count on support from
perhaps three times that number on several key issues.  It was still
well short of a majority---the Third Debate was to comprise three
hundred and one lecterns---but it would give them considerable
influence.  As K.'s Aatu wrote in a letter several years later:

\begin{quotation}
It was glorious, to be us, and then, and full well we knew it.
\emph{This} would be our Hanging Bridge, our leap from cliff to cloud.
We had outwitted war, famine, disease, and now our elders---how, we
asked Fate, could we fail at this turn?
\end{quotation}

The answer was not long in coming.  While the \ship{Cloud} had been at
sea, the Clear Dawn chapter in Ruuda-in-Ruuda had decided to break
with its leadership in Jalkelainen over the issue of Consent.
``Overland!''  their hired criers shouted on street corner after street
corner.  ``Overland for vengeance, overland for justice, overland to
victory!''

Perguuran, who arrived in Ruuda-in-Ruuda two days before the
\ship{Cloud}, was delighted.  Pure Light was still firmly in the
Admirals camp; anything that split the opposition could only aid their
cause.  He toured the shipyard at Kypsyva Mouth on Heliodor 24 to see
the keels that had already been laid for three new \word{taistelaso}.
Longer and leaner than the \word{laiva} that had plied the northern
Ocean for hundreds of years, each twin-masted \word{taistela} would
have a crew of fifty, and carry either a hundred and fifty marines, or
fifty cavalry.  In company with many of his fellow debaters, Perguuran
made a great show of forswearing chocolate and wine (though not beer
or spirits), and of donating the money he might have spent on them to
the ``ship fund''.  ``It is not enough,'' he wrote in a widely-read
circular, ``Not nearly.  But it is a start.''

Many other people toured the shipyards that spring and summer as well,
including Derway's Majeur Callum and the newly-appointed Uwsian
ambassador, Coronel Barsadov ard Nitisza.  Both were troubled by what
they saw: unlike a \word{laiva}, a \word{taistela} had no function
except war.  While it would be two or even three years before they
would be seaworthy, the mere fact of their existence was enough to
upset the balance of power in northern Cherne.

The situation that greeted Ryutaanan and her spouses when they stepped
onto the dock on Heliodor 25 was therefore even more tense than it had
been the previous summer.  There was still a month to go before the
official opening of the Third Debate, but the air was already thick
with accusations of bribery, seduction, magic, and slander.  An
independent debater representing three \word{maatila} in the
southeastern Heladas was found dead outside a brothel on Heliodor 26;
every faction immediately blamed her ``murder'' on some other.  When the
coroner delivered a verdict of misadventure (citing the amount the
debater had drunk, and the strenuous nature of the exercise in which
she had been engaged just prior to her death), he was accused on all
sides of whitewashing rotten boards.

And once again, it seemed, some factions were prepared to use force,
or at least were preparing for someone else to do so.  Tellervo's
Maarit and her Red Knee troop had wintered in Ruuda-in-Ruuda,
nominally to supplement the city's militia (which, at the time,
doubled as its constabulary).  Her membership in Pure Light was an
open secret; so too was her animosity toward the Oxen In Harness,
another rebellion-era troop that had publicly declared for the
Diplomatic cause\footnote{Ld.\ Otnampatelleli, a Bantangui bookster
who visited Uws and Ruuda in the 1150s, claimed to have been shown
letters written by senior officers of the Oxen In Harness to leaders
of Clear Dawn and Bright Reflection, offering the troop's support ``in
all eventualities'' if the factions would support the troop's right to
a lectern at the debates.  Without doubting the Learned's claim, it
seems likely that he misinterpreted what he read.  Representation by
profession, rather than region, class, or family, may be the norm in
Ini Bantang and Barra Bantang, but is unknown elsewhere in Cherne; it
seems improbable that it would have been proposed so many thousands of
gallops away from its source.}, and had taken up residence in the
Brickyard district south of the Kypsyva's second bend.

Assassination was inevitable.  For six centuries, it had been the only
tactic Ruudians had; for those same six centuries, would-be assassins
had known that victory's price would be the same as defeat's.  Every
child knew the stories; everyone could sing the sad, defiant {\aemott}
songs.  By early Citrine, the only question was, who would strike
first?

The answer came on Bluesday, Citrine 5, when a young potato carver
named Roopertti\footnote{His mother's name was unknown: like many
young people in Ruuda at the time, he was an orphan.} drove a
magically-hardened icicle into Tellervo's Maarit's side as she left a
puppet show.  Her bodyguards cut him down on the spot, and then, when
the method of the attack became clear, led a mob to the Uwsian
embassade, where they demanded the head (and other body parts) of
Coronel Nitisza.  The weapon had been magical; Coronel Nitisza was a
magician; Uws favored the Diplomats---that was as far as the angry
crowd cared to reason.

Coolly, Coronel Nitisza ordered her household staff to barricade the
doors and windows, take up arms, and defend the north wing of the
embassade.  Bricks, cobblestones, torches, and (for reasons never made
clear) a sack full of kittens were hurled at the Uwsians, but to no
effect.  As runners roused the city militia, the coronel cast spell
after spell to incinerate every scrap of paper in a quarter-gallop
circle that could possibly incriminate her.

The siege went on half the night.  Understandably reluctant to start a
civil war, the militia duty officer ordered his gens to seal off the
area, but not to attempt to drive the mob away from the embassade.
His calculations were no doubt influenced by the fact that a fifth of
his troops were Red Knees; in retrospect, the fact that he kept them
in ranks when their commander had just been murdered was probably
accomplishment enough.

Reinforcements began arriving around the second hour, when the militia
capitan, Uoleva's Yrj\"{o}, took charge of the scene.  Fiddlebaker
Street, on the north side of the embassade, was cleared with a
linked-arm march, during which dozens of gens were ``arrested'', only to
immediately ``escape''.  After the same procedure was repeated two more
times, the embassade's environs were cleared of all except a few
drunkards, whom the militia left to wake or freeze as Fate chose.

Back in their quarters, the Red Knees quickly elected an interim
commander, who just as quickly declared that the troop would withdraw
from public affairs while mourning T.'s Maarit and ``waiting for the
full process of justice to be carried through.''  Civil war had been
averted, as had the war with Uws that would inevitably have followed
the death of its ambassador at the hands of the mob, but for how long?

In the face of growing unrest, the leaders of the major factions moved
the opening of the Third Debate forward a week to Citrine 24.
Independent debaters immediately protested that some of their
colleagues had not yet arrived, or that they were not done preparing
their arguments.  Once again, Clear Dawn changed course: at the
extraordinary session on the 24th, its members voted against the
motion they had sponsored just four days earlier, defeating it by a
slim margin.  In the end, the only effect of the flip-flop was to make
the independent delegates even more suspicious of the larger factions
than they already were.

The Third Debate opened on Purplesday, Citrine 30, with a solemn
parade down Lamplighters Street.  Its first act was to confirm
Hannele's Kaarina as moderator, and to observe a hundred heartbeats of
silence in memory of T.'s Maarit.  Those were the last quiet moments
for many weeks.  ``I must make my throat raw to hear myself,'' Perguuran
complained in a letter to his sister.  ``From the moment we greet the
sun until well past the moon's turning, every voice is raised in a
cart-driver's lament.''

It was clear from the outset that no one wanted a repeat of the
previous year's impasse.  Appointing five deputy moderators (all
independents), H.'s Kaarina allowed half a dozen debates to proceed in
parallel during the morning sessions.  The afternoons---and on most
days, the evenings as well---brought all three hundred and eleven
delegates together to discuss whichever issue had made the most
progress.

The debating groups were organized around the traditional six
ministries of Chernese government: Revenue, Justice, Public Works,
Magic, Education, and War.  The topics ranged from the pedestrian to
the nebular; on the 12th of Topaz, for example, the following were
just some of the subjects moved for debate:

\begin{itemize}

\item that the tax on untreated pine shall be two rings the
foulterweight;

\item that monuments erected in memory of the fallen shall be
catalogued, and taken into care of this government;

\item that the inscriptions on said monuments shall be inspected for
propriety of language;

\item that ambassadors shall be sent to Seyferte and Leyselle to
govern all dealings with them;

\item that the practice of flensing condemned prisoners be itself
condemned;

\item that the efficacy of said practice in treating wounds received
during the recent war be first further studied;

\item that a road be constructed in the ancient manner from
Jalkelainen to the fishing port of Loupiniema;

\item that the funds for such construction shall come from the people
of Loupiniema;

\item that the funds for such construction shall come from this
government;

\item that funding for construction or improvement of public works
shall attend upon an audit of this government's finances;

\item that the practice of greeting the sun shall be observed by all
militias not hotly engaged with an enemy;

\item that the practice of greeting the sun, though strongly
encouraged, shall remain a matter of personal conscience;

\item that this gathering shall debate no other matter until the
matter of a road to the fishing port of Lopuiniema be resolved;

\item that all persons taking part in these debates shall publicly
declare past and present dealings with the governments of other
states;

\item that persons taking part in these debates shall declare past and
present \emph{financial} dealings with the governments other states;

\item that the tax on untreated pine shall be used to fund a college
for the training of examination inspectors; and

\item that no one in this gathering gives a damn about building a road
to Loupiniema.

\end{itemize}

The Justice debates were the most important, as questions related to
the governance of Ruuda itself took place there.  These were therefore
the melee into which Ryutaanan and her spouses threw themselves.  In
her first speech, Ryutaanan declared her goal was, ``To ensure that all
who govern in Ruuda are chosen by public division, to advocate the
will of those who have selected them.''  Everything else, she implied,
was negotiable, even the question of what to do about Bell Prison.
``Give us only representation by consent,'' she wrote in a
hurriedly-scribbled note to a Bright Reflection partisan, ``And we will
give you what else you desire, for we are certain that so long as
consent be required to govern, the people shall never fail in their
strength in that government.''

It took nine weeks.  R.'s Terhenaar collapsed from exhaustion, and
E.'s Juuso threatened more than once to divorce from their marriage if
Ryutaanan would not compromise, but in the end, they carried the Third
Debate with them.  Bright Reflection, the smallest of the five major
factions, was solidly on their side; when it became clear how deeply
Pure Light was opposed, Clear Dawn, the Vaardians, and the few
diasporans who retained their lecterns threw their weight behind it as
well.

The price paid was higher than Ryutaanan would have liked.  After a
series of impassioned speeches by Perguuran, the Debate moved to
support an ``early and all out'' assault on Bell Prison.  As the one who
had moved the motion, Perguuran was appointed to organize the effort.
Everyone present understood what that meant: Ruuda was going to turn
itself into a major naval power, regardless of what its neighbors
thought, and that power would rest in the hands of ``the purest of the
pure''.  Ryutaanan later recalled, ``I felt some small disquiet at that,
in the wake of our 'famous victory', but nothing more until we
reboarded the \ship{Cloud} for the journey home.''  As they waited in
harbor for a favorable wind, she watched as the three
\word{taistelaso}---which Perguuran had already christened \ship{Sun's
Vengeance}, \ship{Light's Justice}, and \ship{Bright Sword's
Edge}---taking shape in the yards half a gallop away.

\begin{quotation}
They were well tarred and truly masted [she wrote], as straight and
merciless as the eagle's stoop for gar prey.  Twice I spied A.'s
Perguuran before them, doing nought but watch as others constructed
the vehicles on which his ambition would sail.
\end{quotation}

\chapter{The First Expedition}

Ryutaanan did not return to Ruuda-in-Ruuda until 1108.  Selected by
the East Wall \word{maatila} for the debates in 1106, she declined on
the grounds of pregnancy.  Kylliki's Aatu went in her place, with half
a dozen of Ryutaanan's former students as aides.

Ryutaanan later recalled those three years as being the happiest in
her life:

\begin{quotation}
A newborn child is joy made flesh, even when squalling and puking and
pissing down the back of one's dress.  The same may be said of a
newborn nation.
\end{quotation}

Pohjoinen was a whirlwind of rebuilding.  The Potteleva was dredged
for the first time in almost a century, so that potato barges and
fishing boats could be winched upriver to stand directly beside their
warehouses.  Rubble from the North Wall was used to extend the
breakwater, which increased the usable area of the harbor by almost
half.  Mansions on two sides of the main square were joined together
to create a new hospital; initially staffed by doctresses and
magicians trained in the Regimental Kingdoms, it was the seed from
which the present-day university would grow.  Closer to Ryutaanan's
heart, the city's many orphanages were regulated for the first time.
Conditions at some had been appalling, with children being forced to
work as prostitutes or bilge patchers with no hope of ever learning
how to read.  The ``City Hearths'' that replaced them were still drab,
gray places, but at least their denizens had some chance of one day
passing the examinations and bettering themselves.

During this time, the city was governed by a council with seventeen
members.  Daanimo's Daanima moderated its debates; despite (or perhaps
because of) his frequent drunkenness, he was a fearless orator, and
would heap mounds of humorous abuse on anyone who dared sully ``his''
debating chamber with a boring, self-contradictory, or disingenuous
speech\footnote{D.'s Daanima paid most of the fines he imposed on
himself for intemperate language in notes, as he was invariably
penniless.  He took great care to record the exact words used in each
infraction, often referring to them as his ``little pearls''.
Taverneers accepted them instead of cash; many were copied and
circulated, becoming the basis of dozens of scurrilous ballads.
Thirty years after his death, two hundred were collected and published
in the first edition of Kenaatu's ``Pohjoinen Commonplace''.  To this
day, natives of Pohjoinen take pride in possessing a command of
invective unrivalled by any people north of Barra Bantang.}.

Meanwhile, in the capitol, the three \word{taistelaso} slowly took
shape under Perguuran's watchful eye.  Once the keels and ribs were
laid, he directed the shipwrights to concentrate on completing the
\ship{Sun's Vengeance}.  Its triple-planked construction,
square-rigged sails, and single rudders were new to Ruuda, an attempt
to catch up with two hundred years of steady innovation in the
Regimental Kingdoms.  Perguuran understandably wanted to see how well
one would sail before putting another two of the same design on the
water.  He also suspected that some of the Derwegians and Bruyais that
had been brought in to oversee the work might try to commit sabotage.
His relations with Majeur Callum, the Derwegian ambassador, had gone
from cool to frosty after the Third Debate; with the assassination of
Tellervo's Maarit still unsolved (at least in his mind), he disliked
having to trust anyone who was not a member of Pure Light's inner
core.

Perhaps surprisingly, no effort was made during these years to conceal
the fact of the ships' construction, or the reasons for it.  Perguuran
and his allies may have thought there was no point even trying, given
how much noise had been made about them during the debates.  Bravado
probably also played a part: just a few short years after the end of
Pale rule, many Ruudians wanted the Pale Remainder to know their
intentions, just as Ban Jeevan duellists in Ini Bantang and Barra
Bantang will announce their intended targets days before a public
fight.  A popular puppeteer of the time who went by the curtain name
Arky Barky put on a popular show in which a succession of increasingly
feeble characters---from an Uwsian magician (a thinly-veiled mockery
of the ambassador, Coronel Nitisza) to a blind, crippled porcupine
with a squeaky voice\footnote{The dim-witted porcupine is still
popular in children's shows in Ruuda-in-Ruuda, and still goes by the
name Arky Barky.}---slew one Pale Remainder after another in
ever-more-improbable ways.  A few die-hard {\aemott} railed against
the show as disrespectful of the fallen, but for the most part,
Ruudians were in the mood to laugh.

They were also in the mood to boast, something which Perguuran played
to carefully.  He personally conducted tours of the shipyards,
carefully pointing out to Ruuda-in-Ruuda's merchants where and how
their wares were being used.  He also organized banquets, sometimes
seating a hundred dignitaries and their companions on deliberately
rough-hewn benches so that they could use the ships' ribs as tables.
Lit by beeswax candles and fine, soft dayglass, these meals were Pure
Light's single largest expense during these years.

\begin{quotation}
Oh, say not ``expense'' [he wrote to a grumbling colleague in
Jalkelianen in 1107].  Say rather ``investment'', for I trust that to
see these great vessels shaped does also shape the opinions of many,
bending them to the greater purpose as does a cooper bend gar staves
to form a barrel that may hold all manner of things.
\end{quotation}

In that same year, Perguuran was faced with a difficult decision: who
should captain of the \ship{Sun's Vengeance}?  His opponents accused
him of wanting the command himself\footnote{One went so far as to
claim that Perguuran had taken to consorting with prostitutes while
wearing nothing but an admiral's collar and a nautical hat.  When word
of the accusation reached Pohjoinen, D.'s Daanima is reported to have
rolled his eyes and said, ``\word{Im awa pha ta},'' a Bantangui phrase
meaning, ``Oh no, not this again.''}, but there is no evidence that he
ever seriously considered doing so.  Instead, he put forward the name
of the city's militia commander, Uoleva's Yrj\"{o}, arguing that since
the assault itself would be made on land, a soldier should have
overall command of the expedition.

The question was eventually referred from the Eternal Committee on
Naval Matters (which Perguuran moderated) to the main debating
chamber.  Perguuran had tilled his field well; as soon as the question
was raised, two supposedly-independent debaters proposed that Kalle's
Taavi be made captain of the \word{Sun's Vengeance}.  Forty years old,
he had been a pirate for twenty-five of them in the waters off Cape
Grind before declaring for the Rebellion in 1093.  Like U.'s Yrj\"{o},
he was well-known, much-feared, competent, and apparently free of
factional ties.

The \word{Sun's Vengeance} floated free for the first time at dawn on
Midsummer's Day of 1108\footnote{Rumors circulated at the time that
Perguuran had actually had the ship floated the night before, just to
make sure she was sound, then had her brought back into her drydock
for the official launch.  While it would have been completely in
character, it could only have been accomplished with the aid of
powerful spells, and it appears from contemporary accounts that
everyone able to so much as light a candle was busy with shadow
shows.}.  Thousands of people crowded along the docks and the seawall
to watch the land give birth to her; thousands more paid up to a
quartering each to watch the shadow shows conjured up in every inn and
public square.  A motion in the city debates to rename the ship
\ship{Ruuda's Pride} was defeated, but gives a sense of how the nation
felt.  This was the ``new'' Ruuda: swift, tall, and strong.  As light
from the city's \word{skenren lanses} played over her, K.'s Taavi let
the ghost of the river's current carry her into the middle of the
harbor, where the dozens of carpenters and shipwrights waiting in her
hold set to work caulking and mending.

Work redoubled on the \ship{Light's Justice} and \ship{Bright Sword's
Edge}.  Volunteers (mostly boys) organized themselves into brigades to
fetch and carry, often working several evenings a week by the orange
light of cheap dayglass lanterns just for a chance to be part of the
venture.  The last few critics of the expedition in the debates fell
silent, though a few continued to grumble in private diaries about the
expense.  And when the \word{Sun's Vengeance} finally set sail on the
third of Chalcedony, the city emptied\footnote{Encouraged, no doubt,
by Perguuran emptying Pure Light's coffers to arrange free drinks at
every tavern for ten gallops along the coast.}.  ``I could capture the
capitol today with three cripples and a lackwit,'' Majeur Callum wrote
to the king (in code), ``And hold it for a week by dressing up in
sailor's costume.''

The \word{Sun's Vengeance}'s first real voyage did not take place
until the next spring, when she sailed east along the coast to Vaarda.
The trip took a month; upon her return, K.'s Taavi reported that she
was sound.  Everyone knew this was an understatement: the ship had
covered the eighty-five gallops from Scalpin's Rock to Lekkuu in just
two days, which would have been a respectable time for a post-boat.
In turn, U.'s Yrj\"{o} told the debates that, ``We saw no sign of
sea-bandits, and heard from each village in which we stopped that all
such had fled upon word of our coming.''  In this, he was dutifully
echoing Perguuran's message to the Ruudian merchants who were becoming
Pure Light's strongest supporters: a strong fleet would be good for
\emph{much} more than just finishing off the Pale.

In Chalcedony, the \word{Sun's Vengeance} set out to sail west.
Instead of hugging the coast, K.'s Taavi fitted her for a long voyage
and took her out onto the deep blue.  Dried meat, hard biscuit, and
jars of salt went into her hold, along with two extra sets of canvas,
ten gallops of rope, eighty marines, and half a dozen
horses\footnote{Advocates of horse cavalry had won their never-ending
debates with proponents of camels on the grounds that the thick wool
of the stocky northern camels would be debilitating in the heat of the
Salt Coast.  A commander of camel cavalry had responded by shaving his
mount.  Public reaction at its first parade led to a sternly-worded
directive from U.'s Yrj\"{o} that, ``No member of the expedition shall
demean its honorable purpose by presenting geself in a manner inviting
ridicule.''  The phrase ``a shaved camel'' is still used in Ruuda to mean
something superficially plausible, but intrinsically foolish.}.  They
struck north for the {\/O}ruu Islands, the farthest limit of Ruudian
sovereignty.  The scattered fishing villages nesting on those barren
rocks were home to some of the north's most infamous pirate bands, and
also (as K.'s Taavi well knew) to its best sailors.  While his hopes
of recruiting some of them were disappointed, he was no doubt
flattered when two villages on opposite sides of {\/O}ruunepaalo Sound
fought a poetry duel for the honor of renaming themselves after his
ship\footnote{Both sides claimed victory, and changed their village's
name.  Years later, one confessed to the folklorist Ld.\ Duyni's
Maatenala that they had actually done so in the hope of confusing the
mainland's tax gatherers.}.

From the {\/O}ruu Islands, the \word{Sun's Vengeance} was supposed to
sail west to Cape Grind, then follow the coast back to Ruuda-in-Ruuda.
The day after she set out, however, a sudden storm fell on her.  As
dark clouds raced overhead, trying to outrun the storm, the wind
howled and the waves rose higher and higher.

After six hours, U.'s Yrj\"{o} ordered K.'s Taavi to turn the ship
around.  K.'s Taavi refused: the \word{Sun's Vengeance} would have to
sail through much worse on her way to Bell Prison, so they had best
find out if she could do it.  He also informed U.'s Yrj\"{o} that
since they were at sea and under sail, he was actually in overall
command.

Furious, U.'s Yrj\"{o} retreated to his cabin, where he spent the next
day and a half drafting and revising an increasingly lengthy letter of
complaint about K.'s Taavi's demeanor and seagenship.  ``I did not know
whose rage to fear more,'' one of his aides later wrote, ``The storm's,
or my commander's.''

The storm finally passed on Chalcedony 27.  K.'s Taavi's log entry
reads:

\begin{quotation}
17/Chalc/27\footnote{Like other {\aemott}, K.'s Taavi used ``patriotic
dating'', which counted years from the start of the Fifth Rebellion,
rather than ``Years Since'' the end of the Uncertain Angel's
Disputation.}: winds now 15-18 wrack south of south-east, waves 3
strides, no sign of bottom churn.  Lost larboard gaffsail during the
night \& 3 ropes, with 2 pigs washed overboard during breakfast.
Vessel sound and sturdy, \& good pace.  I am confident now that we can
make the journey asked of us, so far as the sun's pure light is our
guide, and dissension does not make cripple of us.
\end{quotation}

\begin{center}
* * *
\end{center}

The \word{Sun's Vengeance} reached Jalkelainen on the second of
Malachite.  Once again, it seemed that the whole city turned out to
see her.  Its governing council had already named a street in honor of
the ship; when it arrived, the city's mayor, Kalle's Ne\"a, was so
overcome that she named another one after it as well.  The amber
workers' commonalty voted honorary memberships for her entire crew;
not to be outdone, the city's furniture makers made K.'s Taavi their
honorary moderator.

It was therefore something of a shock when U.'s Yrj\"{o} marched down
the gangplank and interrupted the mayor's speech to demand that K.'s
Taavi be placed under arrest.  At first, Ne\"a thought he was drunk;
when she realized he was not, she told her master-at-arms to lock him
in a closet until dinner.  She later claimed that she had meant it as
a figure of speech, but at the time, her order was taken literally.
The man who was supposed to be in overall command of the expedition
fleet therefore spent his first few hours in Jalkelainen in a clothes
cupboard, pounding on its door so hard that he broke a finger.

K.'s Ne\"a managed to patch matters up later that evening, but it was
another bundle of twigs on the fire of U.'s Yrj\"{o}'s resentment.
K.'s Taavi continued to insist that command of the \word{Sun's
Vengeance} was his, and his alone, while the vessel was at sea; he
made an ostentatious show of vacating the captain's cabin as soon as
she tied up, while letting everyone know that he still had a key to
its door.  Most of the ship's crew sided with him; predictably, her
marine contingent---several of whom had served under U.'s Yrj\"{o}
during the Rebellion---took the other side in the dispute.

None of this seems to have reached the ears of a young boy named
Tomonainan's Petta, who later said that he fell in love with the
\word{Sun's Vengeance} the moment he set eyes on her.  Aged eight, he
was the second son of a moderately prosperous marriage that owned two
small squidding smacks.  He had sailed on deep water almost since the
day he was born, and, like other sea-struck gens his age with stars in
their hearts, spent his free time practicing knots and making small
models of the ships he hoped one day to captain.

During the week the \word{Sun's Vengeance} was in port, T.'s Petta
spent almost every waking moment studying her.  He queued for hours
with one of his fathers for a chance to walk her deck, then ducked
lessons the next day to do it again.  For an eighth quartering he
bought a hasty charcoal sketch of the ship coming into the harbor,
which he hung on on the wall next to the bed he shared with his older
and younger brothers.  Thirty years later, that same drawing (much
folded and faded) would hang in his cabin on board the {\UL}.

\begin{center}
* * *
\end{center}

The \word{Sun's Vengeance} was scheduled to spend two weeks in
Jalkelainen, but left after only eight days.  The official reason was
that as she had come through her first storm so well, she didn't need
the time that had been allotted for refitting and repairs.
Unofficially, of course, the whole city knew of the tension between
her two erstwhile commanders.  Both had posted several letters a day
back to the capitol while in port; both were looking forward to
presenting their side of the argument to the debates.

Aided by a strong following wind, and the Ocean's currents, it took
the ship only four days to reach Ruuda-in-Ruuda.  She arrived an hour
after sunset.  Normal practice would have been to stand off and wait
until daylight, but U.'s Yrj\"{o} ordered his marines to deploy the
ship's small \word{skenren lans} so that they could practice
night-time navigation.  This was a new maneuver for everyone involved:
while fishing vessels often worked by lantern light, no one had ever
tried to steer a vessel the size of the \word{Sun's Vengeance} by the
focused light of a \word{skenren lans}.

Perhaps surprisingly, K.'s Taavi agreed to the order.  Some have
suggested that he did so in the hopes that some mishap would occur
that would discredit U.'s Yrj\"{o}, although it seems implausible that
any captain as dedicated as K.'s Taavi would put his vessel at risk.
More likely, Taavi simply wasn't willing to seem a coward in front of
a landlubber like U.'s Yrj\"{o}.

The astonished crew were therefore ordered to lay on bottom sails and
out the sweeps.  As the marines played the light of the \word{skenren
lans} back and forth across the dark waters outside the harbor's
breakwater, K.'s Taavi bellowed orders to the gens scrambling about in
the rigging and straining at the sweepstays on deck.  Under a
half-turned moon, she slid toward her berth.

Word of her arrival had of course already spread through the city.
Hundreds of gens were waiting for her; unknown to K.'s Taavi, dozens
more had taken to small boats to escort her home.  Among them was a
scull with eight cobblers' apprentices on board, all of whom had been
drinking for several hours by the time the \word{Sun's Vengeance}
entered the harbor.  One of them proposed that they should ``board'' the
\ship{Sun's Vengeance} and ``claim'' her for their commonalty.
Shuttering their one lantern, they rowed alongside the warship,
slipped under her great sweeps, and took hold of one of the lines laid
over the side in preparation for making fast.  A heartbeat later, five
of the apprentices clambered onto the ship's deck, seized two startled
marines, and declared that they were taking the \word{Sun's Vengeance}
as a prize.

Unbeknownst to them, U.'s Yrj\"{o} had secretly ordered the marines to
arrest K.'s Taavi while the ship's crew were busy making fast after
docking.  Thinking that their plan had been uncovered, and that the
sailors were launching a preemptive attack, the marines drew their
swords.  Amidst cries of treachery and betrayal, the two sides fell
upon one another.  U.'s Yrj\"{o} and K.'s Taavi both called for calm,
but to no avail.

By the time order was restored, two of the five apprentices who had
boarded the \word{Sun's Vengeance} lay dead on her deck, along with
one of her crew, and a double dozen marines and sailors had been badly
injured.  In full view of a bewildered and horrified city, Yrj\"{o}
and Taavi were both arrested and led away to jail in a collar and
chains.

The trial began two days later.  In accordance with the rules of
procedure passed by the Second Debate---the most liberal in all of
Cherne---neither the accused nor his accusers were put to the
nightmare beforehand.  In addition, both sides were allowed free
access to all of Ruuda's laws: in the absence of an established
nobility (and in the face of bitter opposition from those who wished
to take on that role), the Second Debate had also decided against
Regimental-style rental of laws.

The confusion and contradiction of the proceedings highlighted the
immature state of Ruuda's young judicial system.  K.'s Taavi argued
that he had followed orders by bringing the \word{Sun's Vengeance}
into harbor at night.  \emph{He} had not put armed marines on the
ship's deck, ``{\ldots}for what purpose one may only surmise,'' he added
darkly, alluding to the claims flying through the streets that U.'s
Yrj\"{o} had been organizing a mutiny, or (more preposterous still)
planning to steal the \word{Sun's Vengeance} and raise the red flag of
piracy.

Ah, came the response, but if his excuse was that he was only
following orders, then was he not acknowledging that U.'s Yrj\"{o} was
in fact his commander, even when the ship was under way?  In which
case, was not his earlier mutiny the incident's true cause?  But then,
if U.'s Yrj\"{o} was in fact K.'s Taavi's commander, then the
apprentices' deaths \emph{were} his fault after all, were they not?

At this point, three days into the proceedings, the hapless
judge\footnote{The Ld.\ Jaarko Villems, who was no doubt aware that he
had been chosen in part because his mixed ancestry would allow either
side to dress him as a villain.  Nearly bankrupt in the wake of the
trial, and unable to find other cases to try, Villems left
Ruuda-in-Ruuda for Derway, where, more than thirty years later, he was
was arrested for urinating on the dock where the {\UL} was berthed.}
suspended the trial.  ``As public order is put at risk by these
proceedings,'' he wrote:

\begin{quotation}
{\ldots}and as both parties have evidenced the essential weakness of
their arguments by stooping to the indignity of rhetorical
questioning, this matter shall be placed in abeyance until the laws
pertaining thereunto shall be clarified.
\end{quotation}

In effect, the judge had ruled that the Recurrent Debate (as
Ruuda-in-Ruuda's governing body now styled itself) would have to
decide what exactly what division of powers it had intended.  Since
the Debate had already recessed for the long Ruudian winter, that left
K.'s Taavi, U.'s Yrj\"{o}, and their respective supporters in limbo
for five months.

They were the busiest of Perguuran's life.  All his political credit
was in the holds of the \word{Sun's Vengeance} and her sister ships;
if the expedition sank (physically or metaphorically), so would his
career.  He therefore spent the winter in a virtuoso whirlwind of
lobbying, cajoling, threatening, blustering, begging, and bargaining.
He tightened his grip on the capitol's chapter of Pure Light, which in
effect became little more than a mount for his political will.  With
that secure, he temporarily set aside its longstanding opposition to
Vaardian autonomy in exchange for its delegation's support for a
military academy on Regimental lines.  None but the naive were
surprised when U.'s Yrj\"{o} was appointed its first superintendent, a
post which automatically gave him a lectern in the Debate.

The fifth session of Ruuda's Recurrent Debate opened on Peridot 7,
1109, a rainy, wind-lashed Redsday.  With the cries of fishmongers
faintly audible in the distance, two hundred and seventy
three\footnote{Three hundred and seven were supposed to be there, but
several debaters from outlying islands and mountain \word{maatilaso}
had been delayed by bad weather.  Several of these later paid to have
themselves added to official portraits of the Debate's first session.}
solemn gens ascended the Sunlit Steps and entered the country's newly
refurbished debating chamber for the first time.  Once a theater, it
still smelled of the pine scaffolding that had been cleared away the
night before.

Wearing a pure white wool coat and kilt, knee-high leather boots
polished to a mirror-like gleam, and a rich bearskin cloak, A.'s
Perguuran took his place at the principal lectern and welcomed the
assembled debaters.  His opening speech was rousing, and occasionally
ribald; several of those present recorded in diaries that while he
didn't actually say anything, he did so in great style.

He most particularly didn't say anything about resuming the trial of
K.'s Taavi, because by this point there was no need.  A month before,
three seagens had risen in front of a carefully picked judge and
testified that their vessels had been attacked by the \ship{Circular
Key} between YS 1093 and 1095.  The ship's name was important: she had
been K.'s Taavi's.  So too was the fact that the attacks had happened
after the start of the Fifth Rebellion, when (according to the judge)
``{\ldots}all patriotic persons should have felt a duty to rally to the
cause of the living.''  A literal interpretation of that ruling would
make most Ruudians over the age of thirty traitors, but that was
unimportant: all that mattered was that it made K.'s Taavi a pirate in
the eyes of the law.

``A captain's first responsibility is to give orders,'' K.'s Taavi wrote
in a letter published later.  ``Gar second is to know which way the
wind is blowing, and stay off the rocks.''  No warrant had yet been
issued for him, but it would clearly not be long in coming.  Some time
in Chrysoprase, he slipped out of the city on board a west-bound
\word{laiva} to return to the village of his birth.  Wisely, A.'s
Perguuran did not pursue, or even proclaim that where there was
flight, there must be guilt.  With the expedition firmly back under
his control, he could afford to be magnanimous.

\begin{center}
* * *
\end{center}

The \ship{Light's Justice} left harbor for the first time in the first
week of Topaz, 1109.  The \ship{Bright Sword's Edge} joined her at the
end of Chalcedony: too late for a long inaugural cruise, but close
enough to her planned launch date to give Perguuran and the resurgent
Admirals a boost.  Tattoos of the three ships sailing side-by-side
were briefly fashionable, and when a brewer named Dutta's Naameda gave
birth to triplets, there was no question what names they would be
given\footnote{One of the three later became a seaman, and was
reportedly spared from slavery by Bantangui pirates when they learned
that he had been named after a ship.}.

It later seemed to many of the Debaters who stayed in Ruuda-in-Ruuda
that winter that outfitting the small fleet was the city's major
business that winter.  Fights between sailors for the honor of being
in their crews, and between soldiers anxious for a place in their
berths, became so common that Ruuda-in-Ruuda's governing council
reserved two afternoons a week for them in the city's gymnasia.  Tons
of supplies were donated by well-wishers; to everyone's surprise, it
appeared that almost none was sold out a side door.  In a rare moment
of brilliance, U.'s Yrj\"{o} ordered that everything that was not
going to be taken with the expedition be brought aboard the ships at
least once before being redistributed to charity, so that even the
city's poorest could proudly claim to have worn, eaten, slept under,
or bathed with something ``from the fleet''.

``Let Ruuda's strength be a light to the world'' was everyone's toast as
the tenth anniversary of the end of the Fifth Rebellion approached.
Timbers brought into the city for the fleet were already being sawn in
preparation for another busy season of shipbuilding; plans were
already being drawn for a whole fleet of \word{taistela} that would
make Ruuda the preeminent naval power of the age.  Meanwhile, in the
city archives, scores of foreign and native-born scholars kept
searching the records of the preceding six centuries for any clues
they might contain about the nature of the Pale Remainder.

Throughout all this, anyone who questioned the wisdom of making plans
and preparations so publicly was wise to keep their doubts to
themselves.  ``I heard this day a stranger in a tavern say that an
expedition alone was insufficient, and that Ruuda should have its goal
to plant a colony where now stands Bell Prison,'' J.'s Maatenala
grumbled in his diary.  ``In answer came only cheers, which did quickly
become a brawl as one ginger-haired drunkard asked why Bell Prison,
and not Vaarda?''

Everyone took 1110's early spring as a good omen---everyone, that is,
except U.'s Yrj\"{o}.  An early spring meant stronger storms; it could
also push the Ocean's major clockwise current closer to the mainland,
and (most worryingly) closer to Sullair.  His captains proposed route
after route, each basing gar arguments on the maps that suited gen
most.  ``I have as more need an \word{{\ae}lfwif} than navigators,'' he
confessed in a private letter to Perguuran, groaning aloud and tearing
at his hair when the city's constabulary showed up at the docks the
next morning with a double dozen such fortune tellers for him to
choose from.

At last the day came: Peridot 9, YS 1110.  Each ship carried a crew of
one hundred and twenty, eighty marines, and thirty horses, plus tons
of supplies: salted fish and hard cheese, cured apples brought in by
the barrelful from Derway, linen for bandages, a forge complete with a
ton of charcoal, timbers for making repairs, and of course, a
\word{skenren lans}.  Each towed a single-masted cutter capable of
carrying fourteen gens, which was to be used for reconnaissance, and
for travel between ship and shore.  Each also had four oared
longboats, a single catapult, and two heavy ballistae capable of
throwing an iron-headed quarrel weighing twenty lard more than a
gallop (though with middling accuracy).

With drums pounding and pipes skirling, the three ships' crews marched
on board.  Resplendent in spotless white uniforms at the bow of the
\word{Sun's Vengeance}, U.'s Yrj\"{o} and his staff gravely returned
the salutes of the Debaters and citizens who had crowded onto the
docks (which a forward-thinking harbormaster had reinforced over the
winter for exactly this moment).  Heaving at their oars, the harbor
yardies in their sculls pulled the ships one by one into the main
current of the Kypsyva, which carried them slowly to the Ocean.  The
First Expedition was under way.

\chapter{Upuliaq}

% \chapter{Launched at Dawn}

% \chapter{Swing Wide 'Round the Dragon}

% \chapter{That Most Complacent Empire}

% \chapter{Among Pirates}

% \chapter{The Overlanders}

% \chapter{The Attack on Bell Prison}

% \chapter{The Flight Westward}

% \chapter{No End to Disaster}

% \chapter{The Return}

% \chapter{Epilog: Afterward}

\end{document}

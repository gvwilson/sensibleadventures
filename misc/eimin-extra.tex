
\begin{center}* * *\end{center}

Five years later, Razi sat cross-legged in front of his tent beside
the lemon grove, weighing feathers in the tin pans of a little
balance.  ``Have you eaten?''  he asked without looking up.

``Yes.  And I brought date bread for lunch.  \emph{Umma} made it yesterday.''

He nodded.  ``Good.''  He took one of the sand-brown feathers from the
balance and trimmed it with his knife.  ``Canteens are in the tent.
We'll need one each, and a spare.''  The balance leveled.  Razi took
the feathers from them, then removed two more from the bag beside him
and set them on the scales.

By the time I had filled the canteens and returned, Razi had trimmed a
dozen feathers to be exactly the same weight.  He laid them flat in a
folded square of paper so that they wouldn't be crushed.  ``Finish
those later,'' he said, standing and stretching.  He stooped and pulled
his bow case from his tent.  It was as tall as I was, a long tube of
hard brown leather.  A second tube, shorter and fatter, was his
quiver.  Any feather on any arrow in that quiver would level the
balance with any other.

The last thing he took from his tent was a dark green bottle wrapped
in woven leather plaits.  Its stopper was a sword hilt with a plain
leather grip.  Razi slipped his belt through a loop in the leather so
that the bottle hung over his left hip.  ``All set?''  he asked me.

I nodded.  My bow was slung over one shoulder in the padded cloth case
my mother had made for it.  I had fifteen arrows, each with a barbed
steel head the size of my thumb, and my machete.

``Good,'' said Razi.  He clapped me on the shoulder.  ``Let's go
hunting.''

Razi led us south, skirting the grounds of the old palace and then
turning inland toward the watchtower.  We played chess as we walked,
each telling the other a move every fifty paces.  It's harder than it
sounds.  Not only do you have to keep the position in your head, you
have to think of a plan while you're counting paces.  It was fun, but
it was also another of Razi's exercises, training me to keep track of
time and distance while I was worrying about other things.

Time and distance are what the desert is about.  How far do I have to
travel?  When am I going to run out of water?  That morning, after he
said, ``Checkmate,'' and we were just walking, they defined my thoughts,
too.  How much longer would I have to stay in Medef?  How far could I
go when I left?  The furthest I'd ever been was Uef.  It was a tenth
the size of Medef, just thirty families living around a spring three
days away.  I wanted to see Ossisswe, and Ensworth, and Ruuda, and Ini
Bantang, and, oh, the whole world.  I wanted to see what a city looked
like, a city full of \emph{nif} like me.  I wanted to have adventures, so
that I could come back to Medef to tell stories and have girls like
Hediyeh blush and giggle when I walked by, the way they blushed and
giggled around the \emph{nif} traders who came for the examination fair
every autumn.

I put one foot in front of the other, taking four steps for every
three of Razi's.  The sun was half-way to noon when Razi turned east.
We were circling around the big hill so that we could come up to the
watchtower from the side closest to the glass.  I tried to figure out
his plan while we walked.  We would reach the tower mid-afternoon.
That didn't make sense: at that time of day, the breeze would carry
our scent straight to the gargoyles.  Unless...

``\emph{La}, am I to be the bait?''  I asked.

Razi grinned without looking at me.  ``No, I am.''

We reached the beach an hour later.  It was an eerie place.  A
thousand years of dry wind had scoured most of the sand away, leaving
a glass shelf that rippled with the shape of ancient waves.  Pieces of
that shelf had cracked and broken over the years, littering the rock
below with sharp-edged shards ready to cut through even the sturdiest
boot sole.

Razi pointed.  ``Down that way there's a path, by three rocks that look
like a mother goat and two kids.  Do you know it?''  I nodded.  ``Follow
it all the way up, then work your way back down behind them.  I'll
come at them from below.''

I hesitated.  ``How will I know when you're ready?''

He grinned again.  ``You'll know.  \emph{La}, you'll know.''

It took me half an hour to make my way up the hill, and another half
an hour to work my way back down.  I moved from rock to bush, never
taking more than a dozen steps at once, scanning the sky for
slow-circling wings before slipping to the next bit of cover.  I tried
to keep my mind clear, to think only about the next few steps.  I was
just a pocket mouse---no, not even that.  I was nothing more than a
wind-blown scrap of tumbleweed.

I was about five hundred paces from the watchtower when the first
gargoyle launched itself into the air.  She didn't start flying right
away.  Instead, she plummeted earthward down the side of the tower for
a moment to build up speed, then snapped her wings open.  I heard the
faint \emph{crack} a double heartbeat later.  By then, she was already
rising on a warm updraft with her wings spread wide.

I waited until it was moving away from me along the shoreline before
moving again.  I crouched behind the waist-high remains of an old wall
that might once have been the side of a house and slipped my bow out
of its case.  I had waxed the bowstring that morning when I was unable
to sleep; there, in the day, I cursed the faint honey smell that still
lingered.  I slipped one loop into the catch at the bottom end of the
bow, braced it, then bent it and hooked the other end of the string in
place in one motion, just as Razi had taught me.

I swallowed a last mouthful of water and set my canteen down in the
wall's narrow shadow.  I still had half a piece of date bread left,
but I couldn't have eaten it then for its weight in pewter.  I pulled
the drawstrings at my wrists and ankles tight and knotted them, so
that my sleeves and pantlegs wouldn't catch on anything.

The second female launched herself from the tower.  As she rose to
follow the first one, I slipped around the end of the ruined wall and
sprinted for the candelilla bush I'd chosen as my next patch of cover.
I had hunted jackrabbits and sand gulls this way, watching for the
moments when my prey looked away, inching closer a few paces each
time.

Suddenly one of the gargoyles stooped, yipping at the other.  I held
my breath.  It must be Razi.  I slipped two arrows from my quiver, put
one between my teeth, and slowly raised my head.

The third gargoyle, the male, took wing just as I pressed myself
against the candelilla's dry leaves.  His wingspan was easily twice
that of the other two, and when he answered their call, its voice
boomed instead of shrieking.  He flapped its wings twice to gain
height.  The two females circled him as he climbed, all three calling
to each other.

Razi broke cover.  He ran straight for the tower, his bow in both
hands like a spear for balance.  One of the females screeched and
plummeted down toward him, arms outstretched.  I notched my arrow.
Razi threw himself sideways at the last moment, stumbled, rolled over,
and somehow was up on one knee to let fly an arrow I hadn't even seen
him draw.  It punched through the skin stretched tight across the ribs
of the gargoyle's wing.

The other female was right behind her, herding Razi toward the waiting
male who would make the kill.  I sighted, leading him, time and
distance made reflex by long hours of practice.  Razi was running
again, trying to get to the base of the tower.  The second female
swooped across his path to cut him off.  He checked himself and
doubled back three quick paces so that she overshot, but the male was
waiting for him.  Razi jabbed upward with his bow as he went down on
one knee again.  The male stumbled in the air as it hit his wing, but
one taloned hand raked Razi's scalp.

I let my first arrow fly just as the male started to climb.  It passed
harmlessly behind him.  I pulled another from my quiver.  I could
taste the one between my teeth, my snapshot arrow.  I could see blood
on Razi's face.

He had two arrows out himself, one in his teeth, one strung.  As the
first female dove, he fired, drew, fired again.  The second hit her
shoulder, cutting some vital muscle or tendon.  Her wing crumpled, and
she tumbled sideways to crash into the ground.

The male bellowed, circling five stories above Razi's head.  I drew,
exhaled, and let fly.  My arrow hit him square in the chest.  The
impact knocked him sideways in the air.  My snapshot arrow struck his
leg a moment later, but I think he was already dead.  He turned over
once as he fell.

The second female shrieked and banked away.  I saw Razi aim, then
lower his bow.  She was too far away, moving too fast.  I ran down the
hillside.

He was wiping blood from his face with his sleeve when I reached him.
``Are you all right?''  I asked.

He glared at his sleeve, then at me.  ``\emph{La}, do I look all right?
Stupid...  Stupid of me.  Where's the spare canteen?''

``You have it,'' I said, pointing.  He glared at me again, then at the
canteen hanging at his hip.  His look softened.  Suddenly he laughed.

``So I do.  \emph{La}, so I do.''  He handed me his bow, unstoppered the
canteen, and splashed some water onto his face.  ``Gaaah... I'm a
mess.''

Something rustled behind me.  I whirled around.  The female that Razi
had shot was struggling to her feet.  Her right wing hung bent at her
side.  She chittered, clutching feebly at the arrow that had passed
through the complicated double joint where arm and wing joined body.

Razi wiped some more blood from his face.  ``Here, give me that.''  He
took back his bow and drew an arrow.  The female just stood there on
her short, bandy legs, one wing folded, the other hanging, watching
with all-too-human eyes while Razi sighted carefully and put an arrow
through her heart.

\begin{center}* * *\end{center}

We walked back to the lemon grove in somber silence, leaving the
gargoyles' bodies where they were to feed the desert.  They had been
taking Grappa Uzman's goats; he had suggested building a fire in the
base of the tower to drive them away, but Razi had said no, they would
just come back when the smoke was gone.  And anyway, how would we fuel
the fire?  Nothing near the tower would burn---would \emph{sri} Uzman like
to spend the next month carrying dried camel dung?

If someone else had said that, Grappa Uzman would have puffed up his
chest and said yes, he would do it.  Then one of his daughters would
put her hand on his arm and gentle him, the way he gentled his goats
when storms came.  But somehow, coming from Razi, the words made the
idea seem so silly that they could both laugh at it.  I had tried to
figure out how he did it, but it was proving a harder thing to learn
than the bow.

The sun was down by the time we reached the hill overlooking the
palace.  The moon hung low over the Sea of Glass, a mottled diamond
whose light turned the frozen waves into irregular bands of light and
shadow.  Beneath us, the ancient palace of the Uncertain Angels
glistened as if it had been painted with teardrops.  Music still rose
faintly into the night from its walls, a thousand years and more after
its builders had destroyed themselves.

``Look,'' Razi said, pointing.  At first I didn't see it, but then
motion caught my eye.  A sandy desert hare stood on his haunches,
sniffing the wind, while his tribe snuffled and nibbled among the
roots of the eucalyptus trees that grew next to the palace walls.  I
saw a splash in the aqueduct behind them as a fish jumped for a night
fly.  The hares froze, ready to scatter, then returned to their search
for dinner, the moment already forgotten.

``Do you want help?'' I asked, gesturing at the bandage wrapped around
Razi's head.  He nodded, and we started down the hill.

The aqueduct ran from the center of the palace all the way to the
village.  It was rough-edged sandstone, human work, and seemed
half-finished compared to the smooth curved sparkle of the palace.
Every hundred strides or so, a square dipping pool bulged out of the
aqueduct's side.  I filled our canteens at the nearest of these as
Razi unwrapped his bandage and handed it to me.  Splashing some water
on its end, I dabbed at the gash the gargoyle had given him, jumping
only slightly when another little fish went 'ploop!' in beside us.

``That'll do,'' Razi eventually said.  He fingered the gash's edges
gingerly.  ``How does it look?''

``Like someone tried to open up your head.''  I rolled up the
bandage---we'd boil it clean later---stretched, and yawned.

Razi grinned.  ``Past your bedtime.''

I nodded.  It had been a long day.  We picked up our gear and trudged
up to Razi's tent in the lemon grove.  Their blossoms gave the air a
sharp, clean edge.

A clay pot held closed with twine sat waiting in front of his tent.
Undoing the knot, Razi lifted the lid and sniffed.  ``Mm.''  Goat stew
with chilis and pomegranates---one of Grappa Uzman's many daughters or
granddaughters would have brought it as payment for the day's work.
Razi held it out to me.

I shook my head.  ``\emph{Umma} will be waiting.''  My mother always waited
up when I was out with Razi.  I knew because I had heard our neighbors
complain that her worrying kept them awake.

Razi set the pot down and faced me, raising his palms.  ``You did well
today.''

``Not that well.''  The whole way home, I'd been thinking about missing
my first shot.

Razi laughed and clapped me on the shoulder.  ``Well enough.  Now go
on, get some sleep.  We'll talk tomorrow.''  He pushed me gently toward
the village.

I read once that the name ``Aphsi'' comes from an old Thindi phrase \emph{aph
e sessi}, meaning, ``the quiet people''.  If you were to walk through
Medef after dark, you would understand where the name came from.  Most
of the conversation people surround themselves with doesn't really
convey any information.  Its purpose is reassurance: yes, you are
safe, your friends are nearby.

Aphsi don't need that.  They can \emph{drome} that all is well.  There was
a drum in the village square to sound the alarm in case of bandits or
fire, but most of the villagers would never think to use it.  The only
sounds that night as I walked home were the goats, the birds, and
Jephel playing the flute.

He and Mureya were sitting on an old palm log people used as a bench.
He broke off when he caught sight of me and waved.  I waved back, not
breaking stride, but then he asked, ``How did it go?''

I sighed under my breath.  ``All right.''  He shuffled his weight
sideways to make room for me, but I ignored the invitation, and the
uncomfortable look on Mureya's face.  She could \emph{drome} that I didn't
want to stop and talk---not with Jephel, anyway.

But Jephel was \emph{nif} like me, and he wanted to know the details.
There had been three?  And we'd killed them all?  Oh, but Razi was
hurt?  He spoke just slowly enough that I couldn't help but interrupt
every third sentence.  He didn't seem to mind---Jephel never seemed to
mind anything, not his chores, or the teasing of the other children,
or the names they had called him when his parents apprenticed him to
the village butcher.  The only time I'd ever seen him angry was when
Hediyeh and some of the other girls teased Mureya for holding his
hand.  I don't know what they \emph{dromed}, but they never did it again.

I finally ran out of patience.  ``I'm sorry,'' I said, ``But it's been a
long day, and \emph{umma} will be wondering where I am.''

``Of course, of course.''  He ducked his head.  ``I'm sure everyone will
be pleased that you took care of it.''  He laughed and patted his
barrel of a belly.  ``It's not like I could have.''

``Yeah.''  I raised a hand and hurried away.  Behind me, he started
playing again, a slow love song that the girls always asked for.

My mother was waiting for me at the door with a soft smile on her
face.  ``We got them,'' I said unnecessarily.  She nodded and held up my
festival \emph{deel} to show me that she had finished sewing new cuffs on
the sleeves, then set it aside to give me a quick hug.

I dropped my gear and sat on a pillow on the floor, reaching up to
take the bowl of rice and peppers she handed me.  Ten minutes later I
was asleep on the slat bed in the curtained alcove that had been
``mine'' since I turned ten.  My mother had put fresh sprigs of mint and
lemon under my pillow; I think I dreamed of desert hares, and of
moonlight on the Sea of Glass, but after so many years, it is hard to
remember.

\begin{center}* * *\end{center}

People compare the desert to many different things.  The pirates of
Ini Bantang and Barra Bantang say it is like the ocean: broad,
trackless, and completely indifferent to the fates of those who travel
across it.  The tribesmen of Darp say it is just a hot, barren version
of their own great plains, while the Avauntois who look down on the
Karaband from their flying mountain simply shrug and say, ``It's brown,
and boring.''  The silver-masked priests and princesses of Thind think
it is like their own country, full of lost cities, caravans cursed to
wander forever, monsters left over from the Angels' Disputation, and
revenge.

All of these things are true, but none of them are right.

\begin{center}* * *\end{center}

My mother shook me awake.  ``Razi wants you.''  Her voice was soft and
urgent.  ``Now.''

It was an hour before dawn.  I dipped my fingers in the washbowl and
ran them through my hair, slipped my deel over my head, and hurried
out the door, my sandals still in my hand.  A moment later I doubled
back to grab my bow and the pack I had carried the day before.

A dayglass lantern cast dull orange light over two dozen or more
people in the village square.  I didn't need to be able to \emph{drome} to
know they were worried.  I hurried over to Razi.  He had taken the
bandage off his head; the gash was already scabbed over.  Not for the
first time, I wondered about how quickly he healed.  ``What's
happening?''

He pointed at a bald man in the center of the crowd, someone I had
never seen before.  ``Rode in a few minutes ago.  Says he's from a
caravan, coming in early for the examination fair.  Got hit by bandits
last night, lost half their livestock and most of their water.''

``The lion?''

Razi nodded, a quick, angry jerk of his head.  ``Probably.  They need
help rounding up what's left of their animals.''

``Which would empty out the village.''

``\emph{La}, just so.''

Grappa Uzman broke off talking to the stranger and beckoned us over.
Jephel was there, and Yassen the butcher, and Yassen's brother Kirash
who did whatever chores people would give him and drank too much---all
of us \emph{nif}.  ``Here is Eben \'e Mogen \'a Dudir,'' Grappa Uzman said,
introducing the stranger to us in the formal way: Dudir, the son of
Eben the father and Mogen the mother.

``I am Razibendra uy-Ossisswe,'' Razi replied, raising his palms.  Up
close, the stranger was much younger than I had thought---no more than
eighteen.  His head was shaved, not bald, and his bare arms were as
big around as my thighs.

Dudir raised his palms in return, then reached forward to grasp Razi's
hand in the city way.  ``My father told me to look for you.  He said
you could help.''

``If I can.''  Razi turned to Grappa Uzman, raising his voice so
everyone could hear.  ``With your leave, grappa, I will take Kirash and
half a dozen to find their animals.  Everyone else should stay here,
and keep watch.''

``Half a dozen?''  Dudir's dismay made his voice sound angry.  ``We lost
two hundred goats last night!  We'll never find them all with just
half a dozen people!''

Razi grinned.  ``Half a dozen \emph{Aphsi},'' he corrected.

Two minutes later they were on their way out of the village, and
Yassen, Jephel, and I were on the roof of the bakery.  Yassen had
fetched his bow; like me, he had been trained how to use it when he
was young.  Jephel had been trained too, but had never amounted to
more than a danger to himself.  There was nothing wrong with his eyes,
though, or with his patience.  If the lion and her bandits came within
a hundred strides of any villager, they'd know.  Our job was to spot
them before then.

Yassen nudged me with his elbow and passed me a stoppered jug of lemon
water.  I took a sip and passed it to Jephel, my eyes on the hills
south and east of the village.  The sun was just coming up, casting
long autumn shadows tinged green-blue at the edges.  If anything
moved, we would---

``There,'' Yassen said flatly, pointing.  I looked north.  Nothing,
nothing...I saw it, a dark outline that changed shape, then changed
shape again.  I count three, four, and a fifth, larger, on four legs
instead of two.  It was her.

``Pass me a rag,'' Yassen ordered.  Silently, Jephel handed him a scrap
of cotton cloth.  As Yassen wound it around the head of an arrow,
Jephel uncorked the brown bottle Yassen had ``borrowed'' from his
brother.  The sharp smell that came from it was the evil twin of
lemon's tang.

Yassen splashed some of the alcohol on the rag, then cursed.  ``Didn't
bring any matches.''

I had some, wrapped in waxed paper along with my sewing needle and a
toothpick.  A scratch and a touch later, the rag on the arrow was
alight.

Yassen set it carefully on the string, drew back, and let it go.  The
flame fluttered and died before the arrow arced down to strike an
innocent patch of candelilla.

``Do you think they saw it?'' Jephel asked.

The morning shadows were already shorter than they had been, but
knowing where to look, I could still see the bandits slipping away.
``She saw it,'' I said.  Lions only attack herds when they're desperate,
or have the element of surprise.  We'd stay on the roof and watch,
just in case, but the excitement was over.

\begin{center}* * *\end{center}

The caravan straggled into Medef a few hours later, raising an ochre
plume of dust that rolled over gently at the top like one of the
frozen waves on the Sea of Glass.  From my post on the bakery roof, I
counted ten carts and thirty people: eleven on camels, six in the
carts, and the rest on foot.  A short string of pack camels trailed
along behind the last cart, their brightly-dyed harnesses so dusty as
to be almost invisible.  As I watched, the knots holding a
round-cornered box on one camel's back came loose.  The box tumbled to
the ground and broke open, spilling dark cloth bags holding buttons or
spices onto the trail.  A woman in a red-and-gray \emph{deel} rushed over
and began beating the camel's side with her palms to get it to stop,
cursing when it didn't even break stride.  The camel could smell
water---it would have taken a strong arm or another lion to stop it.

One cart stood out from the rest.  It was drawn by a pair of burros,
instead of by one of the humpless camels the other carters used, and
its spoked wheels were as tall as me.  The cart itself was a plain
box; whatever occupied it was securely wrapped in dark blue canvas
held in place with enough rope to rig a dozen tents.  Its driver sat
on the front lip of the box with his back against his cargo and his
burros' reins loose in his hand.  He wore darkly-patterned \emph{deel} and
a bright yellow turban, which meant either that he was from Ossisswe,
or that he was a scholar, or that he liked the color yellow.

I nudged Yassen with my elbow to wake him.  ``Mmph?''  He sat up and
scratched his beard with both hands.  Jephel, who was sitting on the
other side of the roof, glanced at him, then at me, then went back to
watching the hills where we had seen the lion and her men.

``Sorry looking bunch, aren't they?'' Yassen said sourly.  He was right:
when Razi had said 'caravan', I had expected fifty carts or more, with
camels trailing behind them halfway to the horizon.  Medef saw two of
those each year, one for the examination fair, the other three months
later, when merchants from the Salt Coast who were willing to risk the
terrors of the deep desert came through on their way to Ossisswe and
Coriandel.  They brought dayglass, chocolate, silk, mace, ground
coral, fine leather, and books---above all else, books.  Looking at
the tired faces below me, I wondered if anyone in this ``caravan'' could
even read.

Yassen stood and stretched.  The sun was a handspan past noon; the
afternoon heat made the bakery roof feel like the ovens beneath us.
``Keep your eyes wide,'' he ordered, lifting the palmed roof-hatch that
covered the ladder leading down into the cool dark below.

``Can we get some water?'' Jephel asked.  He held up his canteen and
shook it to show it was empty.

``I'll send some up.''  The closing hatch muffled his words.

Jephel sighed.  ``He'll forget, you know,'' he said philosophically,
long accustomed to people forgetting about him.

``Here.''  I tossed my canteen at him.  Being thirsty would be better
than listening to him not complain.

As the caravan came down the last length of road leading into Medef, a
handful of villagers busied themselves in the posting yard.  Palm tree
logs stood in a rough grid, twelve strides to a side.  Ropes were
quickly strung on the pulleys hanging from their tops.  Squares of
canvas, each big enough to cover the roof I was sitting on, were laid
across the ropes and hauled into place to form a single great awning
over the fairground.  The southern and eastern sides were blocked by
more sheets of canvas to keep out the hot afternoon wind.

A hundred strides away, four men strained to raise the sluice gate at
the end of the aqueduct.  Anywhere else in the Karaband, they would
have chanted, ``\emph{Heph...ho!}'' to stay in time.  But they were Aphsi, so
they worked in silence.

The sluice gate finally gave in with a screech of stone on stone.
Water poured from the covered aqueduct into the pipe that led to the
fairground.  There, it splashed into a short trough, which in turn
spilled into a longer one.  People would wash their hands and food in
the first; animals would drink from the second.

``Catch.''  Jephel tossed my canteen back to me.  I shook it---he'd
drunk half---and tucked it under my \emph{deel}.  The sound of water
gurgling into the troughs below me had woken my thirst, but for some
reason, I didn't want Jephel to see that.

Two sharp whistles cut through the noise below me.  I shielded my eyes
with my hand and watched as Razi and the villagers who had gone with
him came over the horizon in the midst of a hundred or more goats.
There wouldn't be any stragglers: if a goat was alive, an Aphsi would
have \emph{dromed} it and brought it in.

Six houses away and two stories below, the caravan's lead cart reached
the stone gate that marked the edge of Medef.  Its driver was the bald
man who had ridden in for help that morning.  Beside him sat a younger
man who had to be his son: he had the same brawny arms, the same
eagle-beak nose, and the same bald head.

Grappa Uzman and Granna Efiyeh raised their palms to the pair in
welcome as the cart rolled past.  A few strides further on, Hediyeh
and some of the other village girls stood in a silent cluster, their
hands tucked politely in their sleeves and their faces shadowed by the
hoods of their \emph{deels}.  I watched Hediyeh's head turn to follow the
cart as it passed her.  The carter's son's head turned in time to look
back.  Something prickled on my spine.

\begin{center}* * *\end{center}

Thind has been an empire for a thousand years.  At times, its masked
rulers have governed everything from the Wise River in the north to
Ini Bantang in the south, and from Cap di Per\,calle in the west to
the East Pole itself---everything, of course, except the dragon's
isles.  It has twice been conquered by the Darpani, and twice more
torn itself to pieces with magic in a human echo of the Angels'
Disputation.

Today, people say of the Thindi that they never do anything for the
first time.  They have a ritual for everything: a new year's first cup
of tea, the beginning or end of marriage negotiations with a business
acquaintance, the death of a favorite pet.  They have rules for how
one should dress, too, rules so precise that they leave almost nothing
to whim.  If you see a man wearing a \emph{dhoti} folded two finger widths
below his navel, a turban knotted in the rear, and rope sandals with
unbleached tie strings, you may know that he is the unlettered
youngest son of a peasant family that does not own its own hippo.  Put
a plain shawl on his shoulders, and you may know that he can read;
knot his turban differently, and you proclaim that he is the oldest
son, or a middle child (whom the Thindi consider unlucky, since they
cause so much trouble).

In the Karaband, on the other hand, people are proud of the fact that
no two \emph{deels} are exactly alike.  We lengthen them, shorten them, add
pockets in common or unlikely places, reinforce the shoulders if we
frequently carry heavy loads... Over time, every \emph{deel} learns from
its wearer how to do its job better, and every Karabandi learns how to
read a person's life from their \emph{deel}.

The caravaneer in the yellow turban wore a \emph{deel} like none I had ever
seen.  The pattern on it was actually black and blue stains.  A wide
belt was sewn around the waist to support the weight of whatever
jingled and clanked in the double pockets on each side.  There were
more small pockets on the sleeves, though these seemed empty.  A long
tear in the side had been carefully stitched shut.  I knew Razi would
ask me later what his profession was, but I had no idea.

He had chosen a spot in the furthest corner of the tented market.
Whatever was in his cart was obviously heavy: he had tied it securely
right over the axle to stop its weight from tipping the cart forward
or backward.  As he scratched his burros' ears and fed them handfuls
of barley he noticed me studying him.  ``\emph{La}, young sir,'' he said,
tying the feed bag shut and tossing it onto the cart seat, ``Will you
help a traveler?''  He spoke in a pleasant tenor, rolling his R's and
slanting his Th's as people from Ossisswe do.

As I walked over, he raised his voice.  ``Dudir?  A moment, if you
would?''  The young man who had come for help waved to show that he had
heard.  He was two stalls away, arguing with another man whose arms
were just as thick, and whose head was just as bald.  His father, I
decided, or his uncle.  Judging from the size of his belly and the
rasp in his voice, he liked his drink almost as much as Kirash.

The man in the yellow turban turned back to me.  ``Here is Suresh
ul-Ossisswe,'' he said, raising his palms.

I raised mine.  ``Here is Eimin ul-Medef,'' I replied.  He raised an
eyebrow at that---at my age, I should have introduced myself as
someone and someone's---but made no comment.  Instead, he reached into
the cart and began passing me cloth-wrapped packages an arm's length
on a side to stack on the shelves behind me.

I knew their wait instantly.  They were books, dozens of them.  My
stomach suddenly rumbled.  ``Would you like me to unwrap them?'' I
asked.

Suresh shook his head.  ``\emph{La}, thank you, but no.  I just want them
out of the way so we can get the press off the cart.''

``The press?'' I repeatedly stupidly.

He pointed at the irregular shape squatting in his cart.  ``My printing
press.''  The look on my face made him laugh good-naturedly.

``So you're another one, are you?'' I hadn't heard Dudir come over.  His
amusement didn't seem as friendly as Suresh's.

``Another what?'' I asked.

He snorted.  ``Never mind.  Let's get this weeping thing down on the
ground.''

Getting it down meant chocking the wheels, undoing several dozen
knots, and looping the ropes around the press again.  With Dudir's
weight on one harness post, and mine and Suresh's on the other, we
tilted the cart back, then paid out the ropes to let the press slide
to the ground.  The whole process took about half an hour, by which
time the sun was down, and the smell of barbecued goat had turned my
stomach into a hard knot of hunger.

``Phah!''  Dudir stretched until something in his back made a popping
sound.  ``Glad that's done.''  He undid the broad belt of his \emph{deel} and
slipped it over his head to reveal a sleeveless shift and long, loose
pants.

I heard giggles behind me, but didn't turn around. Hediyeh and the
other girls were supposed to be helping make the welcome dinner, but
they had taken advantage of the confusion to slip away for an early
look at what the caravaneers had brought.  With so many \emph{nif} about,
it would be a while before their parents noticed they were gone.

Dudir grinned past me as he shook out his \emph{deel} and put it back on.
He was as muscular as an ox.  \emph{And probably as smart,} I thought
sourly.

Suresh clapped him on the shoulder.  ``Thank you again,'' he said.  ``And
you as well,'' he continued, turning to me.  He gestured at the bundles
of books sleeping patiently in the stall beside his cart.  ``If you'd
like to borrow anything for the night...''

I heard another giggle behind me.  I'm sure the girls could \emph{drome}
how much I wanted to---how much I wanted to take them all back to my
little alcove and pore through them, sounding out the words I did not
know and wondering about the people who had written them.  ``\emph{La},
thank you, but it is late,'' I said.  ``Perhaps tomorrow.''

\begin{center}* * *\end{center}

 * Discover that Eimin's father left years ago.
 * The water stone
  * Used to supply the palace gardens
  * Now sustains the village
 * Mention that other villages (Aphsi and nif) have failed
  * The desert is getting dryer

\begin{center}* * *\end{center}

 * Razi reveals the Other Hand to Eimin
  * Aphsi-dominated secret society that protects the deep-desert villages
  * Eimin is being trained to join it
 * Central moral issue: is acceptance worth your soul?

\begin{center}* * *\end{center}

 * Eimin's sixteenth birthday
   * He's apprenticing to be a carpenter
   * This is where he changes path
 * Suresh sows doubts about Razi
 * Need to see what Aphsi society is like, so that Eimin's isolation is clear

\begin{center}* * *\end{center}

 * Eimin interrupts Suresh and Hediyeh
  * Suresh is asking questions about the water stone
  * Eimin finds his ``still center''
  * Hediyeh is shocked that he can creep up on her, then frightened by his feelings
  * Suresh is just embarrassed

\begin{center}* * *\end{center}

Suresh came to find me the next morning.  I should have been
practicing, but instead I was sitting under my favorite lemon tree,
sketching the rest of the grove.  I thought about leaving when he came
over the hill---I didn't feel like talking to anyone, especially
him---then was immediately angry at myself.  I was done with going
around other people.  Let them all stay out of my way for a change.

He trudged up the long rise to the foot of the lemon grove, stopped to
wipe his brow with his sleeve, and waved at me.  I raised one hand
carelessly in reply.  The sun had just cleared the big hill beside the
palace.  It gave the lemon trees long shadows, as if someone had drawn
them, then smeared his hand across the page.  My sketch didn't look
anything like that.  I snapped the book shut and leaned back against
my tree.

I should say that it wasn't actually \emph{my} tree, not like the nameday
trees you plant here.  The grove belonged to the whole village.
People kept track of the number of days they worked in it, then took
their share of lemons and bark at harvest time.

The tree I was under stood apart from the rest.  Someone, years ago,
had spat out a seed, and happened to hit good soil.  It wasn't as tall
or as straight as the ones in the grove, and its lemons were so small,
thick-skinned, and sour that even Grappa Uzman wouldn't eat them.  But
it had a better view than any of the others.

``Fair morning,'' Suresh said.

I opened my eyes.  ``Hi.''

He cleared his throat uncomfortably.  ``Listen, about last night.  I
didn't realize you and Hediyeh were...''  His voice trailed off.

I snorted.  ``We're not.''  And then, because I'd grown up among people
who could \emph{drome}, and there's no point not asking a question when
everyone around you knows you want to ask it, I said, ``Did you...?''

He hesitated, then nodded.  ``I kissed her, if that's what you're
asking.''

It was.  I shrugged.  ``\emph{La}, good for you.''  I picked up a pebble and
tossed it down the hillside.

Suresh sat down a few feet away from me.  ``So what is it?  Why are you
so mad?''

I scowled.  ``You're not my listener.''

He raised his hands, palms facing me, meaning, ``I've heard what you
said,'' then lowered them again.

We sat there in silence for a few moments, looking down at the lemon
trees.  ``It's just...''  I shook my head.  I could feel tears waiting
to happen---tears, at sixteen.  I opened my eyes wide to dry them and
tried to finish my sentence.  ``It's just that you're not Aphsi, so
being \emph{nif} doesn't matter.  But for me...''  I spread my hands.

``Ah.''  He nodded his understanding.  ``So when Hediyeh---''

``It's not about Hediyeh!''  I interrupted angrily.  ``It's---it's just
everything.''

We sat in silence for a minute or more.  There was a breeze, a light
breath of wind off the glass.  The leaves rustled over my head.

Suresh cleared his throat.  ``Is that yours?''  he asked, nodding his
head at the sword leaning against the lemon tree.

``Yeah.  I'm supposed to be practicing.''

He stood up and brushed his hands on his shirt.  ``May I?''

I shrugged.  ``If you want.''

He took the scabbard in one hand, hefting it for weight, then drew the
blade.  It wasn't much---two fingers wide at the base, narrowing
slowly to a point, sharp on both edges.  The grip was wrapped with
plain goat leather, with seed-sized grains of sand underneath to give
it texture.  The hilt was a simple crosspiece, slightly shorter on one
side than the other.

Suresh swung it back and forth, testing the balance, then dropped his
left foot back and folded his left arm behind his back.  ``Not bad,'' he
said.  He lunged at the air, stepped, lunged again, cut high, parried
his invisible opponent, then straightened up.  ``How long have you had
it?''

I got to my feet so that I wouldn't have to look up his nose while I
was talking to him.  ``About a year.  I haven't really used it
much---I'm a lot better with the bow.''

``Mm.''  He glanced at me.  ``But a bow's just a hunter's weapon, isn't
it?  And they want to see if you can be more than just a hunter.''

I leaned against the lemon tree.  ``What do you mean?''

He sheathed the sword and handed the scabbard back to me.  ``I know
about the Other Hand, Eimin.''  His eyes searched my face.  ``I know who
they are, and what they're doing to you.''

I looked down at the scabbard in my hands to hide my confusion.  ``I
don't know what you mean,'' I said.  On impulse, I drew the sword and
fell into the stance Razi had taught me, turned sideways like Suresh
had been, but with my left hand up and out for balance, or to grab at
my opponent.

He pointed.  ``I mean like that.  That's not for dueling.  And no
soldier would ever stand like that, not with another man at each
shoulder.  That's a brawler's stance.  That's for bandits and
nightwork, that is.  Eimin, do you know who Razi really is?''

I cut left, cut right, stabbed, stepped back, cut right, stabbed,
stepped back, cut left.  Never the same pattern twice in a row, Razi
had told me, over and over again.  Watch for your opponents' patterns,
but never give them any.

``Eimin!''  Suresh stepped in front of me and knocked my blade sideways
with a sweep of his forearm.  ``Do you know who Razi really is?''

``I don't want to talk to you about this,'' I said.  I felt---I don't
know the words for what I felt.  Angry?  Panicked?  Like I was
standing on the edge of a cliff, looking down, with the wind tugging
at me?

``Eimin, he works for Lady Kembe, in Ossiswe.  The magician.  He's her
hands in the world.  Do you know that phrase?  He does the things she
can't do herself.  He spies on people for her, and steals from them,
and---''

``What's any of that got to do with you?''  I turned away from him and
started another exercise, a fast one, but I kept listening.

``Eimin, stop.  Stop! Listen to me.  It isn't my business, not that
part, but it is yours.  Or it ought to be.  How do you think she pays
him back?''

``Magic.''  I feinted and lunged, over and over again, trying to make it
one smooth motion, growing angrier by the second.  ``In case you hadn't
noticed, there aren't any Aphsi magicians.''

``Yes, but do you know what kind of magic?  Eimin, she makes \emph{nif}.''

It took a moment for his words to sink in.  I turned on him, the sword
still in my hand.  ``What?''

He raised his hands, palms facing me.  ``She makes people \emph{nif}, Eimin.
People like you.  That's the deal.  That's always been the deal.  The
Other Hand does her work in the world, and she---''

Rage boiled up inside me.  ``Shut up!''  I took a step toward him,
raising the sword.

He didn't flinch.  He didn't even look at the sword.  He just shook
his head.  ``No, Eimin, I won't shut up.  I won't.  Think about it.
The Aphsi need to have \emph{nif}.  They need people who can butcher and
hunt and, and kill, like Razi sometimes kills.  Maybe there are a few
natural \emph{nif}, but most of you are made that way, before you're born.
That's what the Aphsi get from magicians.''

``You're lying,'' I said coldly.  I sheathed the sword and stooped to
pick up my sketchbook.  ``I don't want to hear any more.''

``Eimin---Eimin, I used to work for a magician, in Ossisswe.''  His
voice was almost pleading.  I straightened up, clutching the
sketchbook tight against my chest, but didn't look at him.  ``I was
mending some old books for him, and he asked me if I would take
everything from his storeroom and put it on the shelves.  I dropped
one of the boxes, and while I was picking up the papers that had
fallen out, I saw some notes he'd written down about a spell to make
people \emph{drome}.''

``Hah!''  I glared at him.  ``Now I know you're lying.  Magic can't make
people \emph{drome}.''

He nodded.  ``I know, I know.  He hadn't been able to make the spell
work.  The point is, what he'd started with was an opposite spell, to
make people \emph{nif}.  His notes said he'd learned it from Lady Kembe,
who used it regularly.  That's what the page said, Eimin.
'Regularly'.''

``So you came running out here to tell us, right?''

He hesitated, then shook he head.  ``No.  That was three years ago.  I
just put the pages back in the box, and put the box on the shelf.  I
didn't think it was any of my business.  I don't even think I'd met
any Aphsi back then.  But...''  He shook his head.  ``I just couldn't
let go of it.  Or it couldn't let go of me.  I started asking
questions, and reading.  About a year ago I met a caravaneer who told
me about the Other Hand, about what happened to bandits who tried to
raid the Aphsi.  And then I went to Razi's trial, and---''

``What?  What trial?''

He searched my face again.  ``You don't know that either?  Someone
stole a glass turtle from Lord Mezeri.  He's another magician in
Ossisswe, Lady Kembe's main rival.  The city guard took Razi for it,
but couldn't prove it was him.  That's why he's been here so much for
the last year---Ossisswe just wouldn't be healthy right now.''

``I don't believe you.''  I pushed past him.  ``I don't believe any of
this.''

``Eimin.  Eimin!''  he called to my back.  ``It doesn't matter whether
you believe it or not, it's true.  You're not an accident, Eimin.
None of this is an accident.  They made you \emph{nif}, and now they're
making you into a sword.''  He fell silent.  I didn't look back.

\begin{center}* * *\end{center}

I hiked out to Tin Point the next day.  I kept my pack light: three
canteens, my tent, some dry date bread, my knife, and my climbing
rope.  I don't know why I took the climbing rope.  Or the knife, for
that matter.  I took my sketchbook as well, and my box of charcoals.

I skirted the goat pasture so that I wouldn't have to explain where I
was going to Grappa Uzman.  It's funny---when I was little, I loved to
pet and feed his goats, and listen to his stories about chasing away
gargoyles, or the time he went half-way to Ossisswe to sell one that
had blue eyes.  Now that I'm older, I'd love to hear those stories
again.  At sixteen, though, I thought the stories were as boring as
the grownups who told them.  Besides, he would probably ask me where I
was going, and since he wasn't \emph{nif}, I wouldn't be able to lie.

The path ran through the lemon grove, along the hill overlooking the
palace, then down past the watchtower where I'd killed the gargoyle to
the old shoreline.  The Sea of Glass was dazzling.  I put on my sun
goggles as I walked.  The secret to traveling long distances, I'd
discovered, was to create a rhythm in my head.  My feet were its
backbone: left, right, left, right.  I imagined a little wordless tune
on top of it, da-dum-di-dum-di-diddle, da-dum-di-dum-di-day.  If I had
my hands hooked through the shoulder straps of my pack, I tapped
another little pattern with my fingers:
first-fourth-second-third-thumb, then fourth-second-third-first-thumb.
I could go a hundred paces or more with a single pattern.  When I grew
tired of that, I tried to see how many different words I could spell
using just the letters from a word like ``eucalyptus'' or ``catastrophe''.

And I had conversations, lots of them.  Sometimes they were ``should
have saids''---things I didn't think of saying at the right time that
would have been funny, or cutting, or made me seem wise.  A lot of
times I imagined myself rescuing someone from lions, or climbing down
a sheer cliff to save a child who'd slipped and fallen.  I fought
duels with bandits in the middle of gritstorms, beat off whole flocks
of gargoyles, and snuck in and out of magicians' palaces without
leaving so much as my breath behind.

Oh, you're laughing, but we've all had daydreams like that.  I would
never have told anybody about them when I was sixteen, any more than I
would have stopped to talk to Grappa Uzman.  They were just a way to
pass the time while I was walking, and more fun than my real life.

I stopped for a drink and a mouthful of date bread when I reached the
shore.  On a whim, I dropped my pack and walked out fifty paces onto
the glass.  The three fish Razi had once showed me were still there,
frozen in the glass a double arm's length below me.  The one in front
was bent slightly, as if it had been about to turn left when things
changed.  I'd never noticed before, but a shallow depression had been
worn into the glass above the fish.  People had been coming and
looking at them for a thousand years.  They were long gone, but the
fish were still there.  I felt a little foolish after a moment, so I
walked back to the beach and kept going.

I stayed on the trail after that.  It would actually have been easier
to walk on the glass---it was flatter, and there weren't any
potholes---but I was going to spend the next two days doing that.  I
pulled a stubby leaf off a raspberry cactus as I walked and chewed it
to a pulp to kill my thirst.

There is life all around in the desert, if you know where to look.
There was a spider web on that cactus, for example.  The spider might
have been hiding from the sun down among the cactus's roots, but she
was there.  I saw finger lizards sunning themselves on rocks, and an
S-curve across the road that some snake had left the night before.  A
shirrup followed me for half an hour or more, flitting from rock to
bush to cactus, sometimes ahead of me, sometimes behind, hoping for a
crumb of date bread, or maybe just waiting to see if I would scare
something tastier than finger lizards into breaking cover.

I slept beneath an overhanging rock during the middle of the day.  I
allowed myself two mouthfuls of water before I closed my eyes---you
sweat less when you're not moving, so my body would have a chance to
soak it all up.  I kept my knife in my hand as I dozed, as Razi had
taught me.  I'd been worried the first few times that I would cut
myself in my sleep, but I never had.  I still didn't do it when I was
at home in my own hammock, though.

I reached the cache at the end of Tin Point about an hour before
sundown.  It was marked by a pile of rocks that came up to my waist.
There were water bottles underneath them, and sealed jars of honeyed
fruit, medicines, bandages, even some firewood.  I didn't take
anything---it was only for emergencies.  But it was a good place to
camp, so I did.

I washed down a thick slice of date bread with what was left in my
first canteen and pulled my sketchbook out of my bag.  The sun had
just touched the horizon, and the whole Sea of Glass had turned red
and orange, as if it were on fire.  I drew the lines of the frozen
waves as quickly as I could, smudging them with my thumb to make
shadows.  The sea had been calm here, when the great spell came.  I
thought it must have been a beautiful day.

I stopped sketching when the creeping shadows reached me.  I yawned
and rolled my shoulders to get out the kinks, tucked my sketchbook
back in my pack, and wrapped myself in my traveling robe for warmth.
Finally, as the small creatures of the evening began to stir, I
couldn't avoid thinking about what Suresh had said any longer.

There were a hundred reasons why it couldn't be true.  Something like
that could never be kept secret.  And besides, the Aphsi didn't have
magicians.  No-one knew why, but people who could \emph{drome} never had a
talent for magic.

But maybe it wasn't a secret.  Maybe grownups just didn't talk about
it around children.  Maybe they were waiting until they could \emph{drome}
that I was ready to accept it.  Maybe that was another of the Other
Hand's tests.

I realized suddenly that I was playing finger-patterns on the ground,
one-four-three-two-thumb, thumb-four-one-three-two.  I took a deep
breath and relaxed.  What if---just supposing it was true.  Just
supposing.  Did my mother know?  Had someone come to her and my father
and said, ``It has been decided''?  Or was there some sort of lottery?
Or had they volunteered?  I thought about my mother, heavy and round
like Hediyeh's mother had been when she was carrying the twins.  A
dark figure bowed its head and put its hand on her belly.  The air
tingled with power, and---

I shook my head.  It couldn't be true.  Being \emph{nif} was just bad luck,
that was all.  And anyway, if I'd been made \emph{nif}, then Jephel must
have been made that way too.  I snorted.  FIXME: something
disparaging.

The stars had come out.  I closed my eyes and dreamed of Jephel
wearing a magician's embroidered robe, turning himself into a goat by
mistake.

I woke up as the sky started to pale.  It only took me a moment to
make water and shoulder my pack.  Dawn is the best time to walk on
glass.  The glare is bad, but you miss the worst of the heat.  I put
on my goggles and set off.

Once I was on it, the sea didn't seem as calm as it had the evening
before.  The frozen waves were a couple of feet high, which meant that
the surface was never exactly flat.  My ankles always felt like they
wanted to roll over.  Every few paces, I had to step up onto the crest
of the next wave.  My thighs started to burn after a few hundred
steps, but the feeling passed and I found my rhythm, step step step
and up, over and over again.

I waited until the sun was halfway to noon before putting up my
umbrella.  I was quite proud of it---it was made out of Bantangui
bamboo, and I was always careful to roll the cloth, rather than
letting it fold, so it wasn't cracked or creased.  Best of all, the
handle slipped into a sock I'd sewn onto the side of my pack, so that
I had my hands free while I walked.

Nothing lives on the glass.  Parents tell their children stories about
glass turtles, and glass crabs, and even glass squid, but they're just
stories.  Even the birds don't go very far from shore.  Underneath is
a different story: there are places where springs well up, and just
enough light gets through the glass for life to bloom.  Grappa Uzman
told me once that there were places where the glass had cracked, and
trees had sprouted, but he never said where that was, and from the
smiles on other children's faces, I guessed they could \emph{drome} he was
just telling a story.

I hiked until early in the afternoon, then pitched my tent with the
silver side out to reflect away the sun.  Lying inside was like being
in an oven, but walking would have been worse.  I drank some water and
tried to take shallow breaths.  I had been turning Suresh's words over
and over in my head while I walked.  What if?  What if it was true?
What if that was why my father had left?

I closed my eyes and did something I hadn't done since I was little.
I tried to \emph{drome}.  For a while, I believed that if only I wanted it
enough, if I really tried, somehow I'd be able to.  Then I laughed at
myself.  The nearest living things were half a day's walk behind me.
Even if I could suddenly \emph{drome}, how would I tell?

I crawled out of my tent late in the afternoon and set off again.  If
I walked until midnight, slept for three or four hours, then pushed
myself hard, I'd be at the fishing fleet by mid-day.  I wished that I
had saved some of the raspberry cactus---I was going to have enough
water, but I was thirsty whenever I let myself think about it.  So I
tried to think about other things, which meant about Suresh kissing
Hediyeh, and Jephel dancing with Mureya, and how long it would be
before someone found me if I slipped and broke my ankle on the glass.

The sun went down.  The stars came out.  My thoughts chased themselves
around and around in my head.  It could be true, I decided.  That
didn't mean it was, but it could be.  Only a few Aphsi would have to
know.  My mother probably didn't.  Even Razi might not.  The Other
Hand, or somebody else, could just decide, and have their magician
make the spell, without telling anyone else.  Then they could wait to
see who worked out, and who didn't.  Somebody like Jephel would just
grow up to be a butcher.  Someone like me would be taught how to hunt
and climb, how to move silently, how to become so still inside that
even the most sensitive Aphsi in the world couldn't---

That was it.  All of the exercises Razi had given me, teaching me to
quiet my thoughts and let go of my feelings.  They weren't really to
teach me how to hide my \emph{deroma} from Aphsi.  What would be the point
of that?  They were trying to see if I could control my anger enough
to be told that being \emph{nif} wasn't just bad luck.  Suresh was right.

I stopped in my tracks.  I had never felt anything like the cold,
clear rage that filled me at that moment.  I wanted to hurt somebody.
I wanted to drag them out into the village square and tell the whole
world what they'd done.  I wanted to blind them, put out their ears,
make them \emph{nif} like me, then leave them lying there, weeping.  I
didn't know who they were, but I was going to find out.

I almost turned around to head straight back to Medef, but instead I
stayed on my original course.  ``A fatal mistake can sometimes save you
a few minutes of thought,'' Razi told me once.  If I went back to Medef
then, everyone in the village would be able to \emph{drome} how angry I
was.  I needed a plan, and I needed to be as clear inside as the glass
beneath my feet.

I didn't stop for sleep that night.  I ate and drank as I walked, and
reached the fishing fleet at dawn.

There was nothing left of them above the glassline, not after a
thousand years, but their hulls still lay perfectly preserved in the
glass.  Some of the fishermen had been thrown from deck onto the glass
when the spell came and froze the sea beneath them.  One had been up
to his ankles in water-I found the two holes that showed where his
friends had chipped him out of the glass.  The holes' edges had been
worn smooth by time, but I could still see where each toe had been.

I looked around for a while to see if I could find any fish bones for
good luck, but they were long gone.  It felt eerie being there on my
own, surrounded by silence, looking at the moment when everything
changed.  I knelt and ran my hand over a frozen bow wave, then brushed
my fingers along the worn top of the ship's hull.  There was grit
there, carried by the wind.  One day it would wear the waves
themselves away.  I slipped my pack from my shoulders and pitched my
tent.

I slept through the heat somehow.  When I woke, I tore a page out of
my sketchbook, wrote, ``Eimin the Nif was here'' on it, and left it on
the glass underneath an empty canteen.  My shadow walked tall beside
me until the sun went down.

\begin{center}* * *\end{center}

 * Suresh tries to get Eimin into his plans
 * Aphsi know something is wrong: Hediyeh guesses that Eimin is going
   to run away
 * Confrontation with Razi
   * Eimin's mother convinces him that it's not true
   * His ``father'' left because of her affair with Razi

\begin{center}* * *\end{center}

I ran.  I ran like a jackrabbit runs, like a hawk flies.  I ran like I
had never run before.  Cold sick fear lay heavy in my gut.  The moon
was already half-way up the sky.  I'm not going to make it, I thought,
then ran even faster.

I stumbled just before the top of the hill, scraping my hands and
knees as I fell.  I knelt there panting for a moment, then struggled
back to my feet and kept going.  The only sounds were my breathing and
the steady crunch of my sandals, left, right, left, right.  Razi's
sword bumped against my hip every second step.  I had to find Suresh,
and stop him.

I reached the top of the hill.  The palace shone in the moonlight.  I
could smell lemon trees and moist earth.  There! Someone was standing
in the shadows by the palace gate.  No, wait, there were two people,
and they were kissing.

It was Suresh and Hediyeh.  She pushed him away, then grabbed his
wrists and pulled him back.  They kissed again.  Good, I thought
coldly.  As long as they're doing that, I don't have to worry about
being seen.  I cleared my throat, spat, and started running again.

The path dipped behind a shoulder of the hill.  When I could see the
palace again, the gate was open, and they were gone.  Then I was
running through rows of eucalyptus, neat and square.

The trees stopped two dozen strides from the gate.  I slowed to a
walk, drawing Razi's sword.  The quicksilver blade took shape with a
soft gurgle, like water being poured from a jug.  Heart pounding, I
stepped through the gate.

The palace music played softly around me.  I let out my breath with a
whoosh.  It would have stopped if the spell had already been broken.
There was still time.  I closed my eyes.  My heart was pounding in my
chest.  I imagined calming it, soothing it, as if it were a frightened
pocket mouse.  Suresh couldn't \emph{drome}---at least, I hoped he
couldn't---but Hediyeh certainly could.  Sh, be still, be still, I
thought.  I imagined myself petting my heart gently, rubbing its ears,
brushing its soft fur.  I imagined it growing still under my hand,
until it did.

I opened my eyes.  My moonshadow was just two fingers wide.  There
wasn't much time left.  I trotted across the outer courtyard on the
balls of my feet and took the stairs two at a time.  I was calm.  I
was certain.  I was as clear as your breath, empty, a quiet stream
flowing without a ripple or a wave.

I pressed myself against the wall next to the open door, then craned
my head to look into the greeting room.  Empty.  I slipped inside.
They had probably gone through the inner courtyard to the pool.  I ran
the other way, to the audience room.  A side door opened onto a narrow
stairway that led up to a window set just below the roof.  When the
palace was in use, a servant would have stood there to watch for
guests arriving.  It didn't overlook the pool, but it was big enough
for me to squeeze through.

I pushed Razi's sword back into its bottle, corking it with the
handle.  The window was waist-high.  I put both hands on the sill and
hopped up onto it, tucking my feet underneath me.  It wasn't quite
tall enough for me to stand, so I leaned out, one hand on the wall for
balance, the other reaching up for the edge of the roof.  The stone
felt cool against my skin.  I took a breath, let go of the wall, and
stepped out.

My whole weight hung from one hand for a heartbeat until my other hand
found purchase.  I pulled myself up onto the roof and drew Razi's
sword once again.  The stones were dark and discolored where the brief
puddles left behind by occasional storms had dried.  I hunched down
and crept forward.

Suresh and Hediyeh were already in the pool, laughing and splashing.
Their clothes lay at the pool's edge.  Suresh's body and legs were
pale in the moonlight, though his hands and feet and face were bark
brown.  Hediyeh---Hediyeh was all one color.  That's all I want to
say.

Suresh lunged forward, catching Hediyeh's foot.  ``Stop it! Stop it!''
she squealed as he tickled her.  He let go, then dove, swam beneath
her, popped up behind her, and flicked water off his fingers into her
eyes as she turned around.

``Stop!''  she giggled.  She wiped her face.  ``You swim so well.  You're
like a fish.''

``Thank you.  I practically grew up in the water back home.''  He
slipped around her, heading for his clothes.  ``Used to spend pretty
much every summer in the lake, looking for treasure.''

``Treasure?  Really?''  Hediyeh dog-paddled after him.  I shifted my
weight.  What was he doing?

``Well, no, I mean, there wasn't really any treasure.''  He glanced up
at the moon, then back at Hediyeh.  He was at the edge of the pool
now.  ``But I was, I don't know, seven, maybe eight, and there was this
story about a boy who found a gold necklace and a helmet once.  It was
a true story, too.''  He kicked himself up a few inches so that he
could reach out and pull his clothes closer.

``You're not getting out already, are you?''  Hediyeh pouted.

``What?  No, no, I'm just...  Anyway, the boy sold them in Ossisswe,
then bought a whole herd of goats and lived happily ever after.''
Suresh pulled something out of the pocket of his trousers and turned
back to Hediyeh.

``So did you ever find anything?''  she asked.

``Mm hm.''  He held up the ring.  It sparkled in the moonlight.

``Oh! Let me see!''  He held it out, then pulled it away just before she
took it.

``Come and get it,'' he grinned.  He wriggled away from her, drawing her
toward him, always out of reach.  He was leading her closer to the
waterstone, directly beneath me.  I tensed.

``Oh, stop teasing!''  Hediyeh paddled forward.  I slipped my feet out
of my sandals.  Suresh was almost close enough.  Almost...

Suddenly he stopped backing away.  He looked up at the sky.  ``It's
time,'' he said, and he slipped the ring onto his finger.

Hediyeh screamed in shock as the ancient haunt flooded into Suresh.
He lunged forward and grabbed her arm.  ``Let go! Let go!''  she
shrieked.  She tried to pull away, but he was too strong.  He pulled
her close, spinning her around so that he could get his other arm
around her neck.  The ring throbbed with red light, making the water
look like blood had already been spilled in it.

I splashed into the pool an arm's length away.  My feet hit the bottom
hard.  I pushed off, so that I shot halfway back out of the water
again.  Hediyeh screamed again.  I raised my arm.  The quicksilver
blade of the sword licked the night above us.  ``Let go of her!''  I
ordered.

``Eimin!''  Suresh tightened his grip on Hediyeh's neck, choking off her
scream.  ``What are you---''

``Let go!''  I twisted in the water and feinted at him.  The sword blade
snapped like a whip just inches from his head.

``Eimin, no, wait, listen to me.  Listen! You have to let me do this.
Eimin, it's for both of us.''  He glanced up at the moon again.
``Please, Eimin, think! Think about what they did to you.  Think about
what they kept from you! Once I have the power that's in that stone, I
can undo that.  I can make you like everyone else.  We can stop them
from ever doing it to anyone ever again.  Please, Eimin, I know what
you're thinking, but it's the only way.''

``You don't know what I'm thinking,'' I said.  I flicked the sword at
him again to try to back him away from the stone.  He didn't flinch
this time.  The blade caught his shoulder, leaving a thin red line
that immediately started to drip.  His ring brightened.

``So is that it?  Are you going to be a pawn all your life, when you
could be a king?''  Hediyeh pulled at his arm.  Her mouth was open, her
eyes bulged.  She couldn't breathe.  He was choking her while he
talked to me.

``I don't know what I'm going to be,'' I told him.  I flicked the sword
again.  ``All I know is that you're a liar.  Nobody did this to me.
Nobody made me this way.  My mother wouldn't let them.''

His face hardened.  He took a deep breath.  He was going to dive.
Hediyeh had almost stopped struggling.  He was going to dive and give
her death to the stone.  I was out of time.

I lunged forward.  Razi's sword went right through the joint of
Hediyeh's shoulder and into Suresh's ribs.  She tried to scream, but
couldn't.  He could.  He did.  His scream was rage and pain and a
thousand years of being trapped in a little shard of glass.

He shouted a word and slapped the sword blade with the hand that wore
the ring.  The blade snapped like a dry twig.  He rolled in the water
and dove.  I dropped the broken sword and went after him.

His ring was so bright now that it cast red-tinged shadows.  Hediyeh
was limp in his arms.  Dark blood made the water murky.  He grabbed
the corner of the waterstone with one hand to steady himself and held
her down against it with the other.  Air bubbled out of his mouth as
he began the spell.

I frog-kicked over to him and grabbed his arm.  He tried to beat me
away, but the water took the force of the blow away.  I was already
running out of air.  The ring was blinding me.  I let go of him,
grabbed Hediyeh's arm, and planted my feet on the waterstone to try to
pull her away.

Cold... I have never felt cold like that.  It was as if ice water was
flooding into my veins.  The waterstone wanted.  The cold rose through
my legs, into my body.  I pulled weakly at Hediyeh's arm.  It was too
late.  The cold was in my heart, sucking all my strength away.  I
couldn't---

I twisted, and with the last of my strength I let go of Hediyeh and
jammed my fingers into Suresh's mouth to stop the spell.  He bit them
in surprise, then tried to pull his head away.  He couldn't let go of
the waterstone.  He couldn't let go of Hediyeh.  He couldn't get away
from me, so he tried to spit them out, but I hung on.  I was actually
holding on to his bottom teeth for an instant.

The red glow started to fade.  He shook his head from side to side
frantically.  I was so cold.  I couldn't feel my legs any more.  The
glow flickered and died.  In the last of its light I caught a glimpse
of Suresh's face.  Whatever was in there looked back at me full of
hate and despair and longing, and then it was gone.

The next thing I remember I was lying on my back looking up at the
stars.  I didn't know where I was.  I knew I was cold, but I didn't
have the energy to fetch a blanket.  My mother would look in on me in
a while.  She could always \emph{drome} when I was awake, when I was too
hot or too cold.  She'd take care of me.

I blinked and turned my head as Suresh dragged Hediyeh out of the
pool.  I must have made a sound, because he looked at me.  Our eyes
met.  He shook his head slowly and started to sob.  I closed my eyes
again.

\begin{center}* * *\end{center}

When I woke again, Razi was beside me.  He had cut his shirt into
strips, and was bandaging Hediyeh's shoulder.  Suresh was gone.

``Cold...''  I whispered.  He nodded without looking at me, intent on
his work.

``I'll build a fire as soon as I can.  She's lost a lot of blood.''  He
knotted another strip of shirt under her armpit, then sat back and
rubbed his ankle.  ``What happened?''

``Suresh...?''

Razi shook his head.  ``Gone.  But he left this.''  He held up a small
glass ring.  It was just glass, now.  Whatever had been in it had
missed its chance.

I tried to sit up, but I was too weak.  It was like one of those bad
dreams where you're a puppet whose strings have been cut, so that you
can see and hear, but you can't move.  ``Can't ...''  I gasped.  For
some reason it seemed funny to me.  I started to chuckle, but it
turned into a wheeze.

Razi frowned.  ``Well, then who fished you out of the---oh.  Hm.
Interesting.''  He turned the ring over in his hand thoughtfully, then
tucked it into his pocket and struggled to his feet.  ``And where's my
damn sword?  If there's even a scratch on it...''

In the end, I went for help, so that Razi could stay and look after
Hediyeh.  It wasn't a good idea.  I only had to walk as far as the
goat pasture, but it seemed to take as long as my trek out to the
fleet.  I stumbled lightheadedly straight into the arms of Grappa
Uzman.  I don't know how much sense he made of the story I told him,
but it frightened him enough for him to send one of his grandsons back
to Medef for help.  I fell asleep, shivering, as the sun came up.

It was a month before my strength returned.  I spent most of that time
sleeping, or sitting on the roof, watching the world go by.  A pair of
sand pigeons built a nest in the eucalyptus tree beside our house in
that month.  Little Ashwara took her first steps.  And somehow,
Hediyeh became a hero.

I'm still not quite sure how that happened.  The waterstone had taken
even more out of her than me, and for a few days the wound in her
shoulder---the wound I'd given her---looked like it was going to go
bad.  Once she was well enough to sit up and talk, though, her room
was never empty.  Everyone wanted to hear the story from her, over and
over, grownups and children alike.  Or maybe they could just \emph{drome}
how much she enjoyed telling it.

Razi came to see me every day.  One morning he brought three people
with him, two men and a woman.  The two men asked me questions, over
and over, patiently, drawing every little detail of the story out of
me.  The woman just listened and frowned.

Razi came back into my room after they left and turned his left hand
over, palm up, palm down, meaning, ``The Other Hand''.  I nodded.  I had
guessed as much.  They didn't look like Razi, any more than any other
Aphsi, but there was something in their eyes, sadness and patience and
acceptance and watchfulness all at once.

``They still haven't found your friend,'' he said.  ``Someone took water
and food from the cache at Tin Point, but there's no trail.  We asked
Lady Kembe to try a spell, but even she couldn't find him.''

``Do you think he's dead?''  I asked.

Razi shook his head.  ``Dead men can't hide that well.  Oh, but here, I
almost forgot.  This is for you.''  He brought his arm out from behind
his back and handed me a chessboard in a leather sleeve, and a cloth
bag full of pieces.

``Thanks.  Thanks, Razi.''  I slipped the board out of its sleeve.  ``Do
you have time for a game?''

He grinned.  ``Always.''  He pulled the little lopsided table I'd made
over beside my hammock, then sat on the stool on the other side.  Razi
took the bag from me and poured the pieces onto the board.  A moment
later he held up two fists.  ``Choose.''

I laughed.  ``Either.''  He laughed too, then opened his hands to show
me a pair of white pawns.

\end{document}

% Characters
% 
%  * Hediyeh
%    * Alpha female in her age cohort
%    * She and Eimin might actually be a couple if:
%     * Eimin wasn't so defensive
%     * She didn't enjoy manipulating people so much
%  * Jephel the butcher's apprentice (the other \emph{nif} Eimin's age)
%    * Slow but not stupid
%    * Has a girlfriend already (Mureya) who doesn't mind him being nif
%    * His maturity will later shame Eimin (who believes as the Aphsi do that
%     being nif makes you worthless, and who has taken out his resentment on
%     Jephel his whole life)
%  * Suresh: claims to be collecting stories for a book
%    * Paying his way as a bookmender and scribe
%    * Not much older than Eimin (18 vs. 15)
%    * Hediyeh is interested: Suresh defeats Dudir with wit rather than muscle
%  * Dudir, the cartwright's son (Eben is the cartwright)
%  * Yassen and Kirash
%  * Grappa Uzman and Granna Efiyeh
%  * Ashwara (new-born)
